\documentclass[../../../interview-questions.tex]{subfiles}

\begin{document}

\subsection{分库分表新问题}

\paragraph{分布式一致性问题}

当更新内容同时分布在不同库中,不可避免会带来跨库事务问题。跨分片事务也是分布式事务,没有简单的方案,一般可使用"XA协议"和"两阶段提交"处理\footnote{参考:\url{https://mp.weixin.qq.com/s?__biz=MzU3OTc1MDM1Mg==&mid=2247500792&idx=1&sn=7d6d688ce9cdb4f1a60b43f97163efe1&chksm=fd63d347ca145a51cbf42519a256ecbeafaaec40cc74c5ac96e481d747eaf094105e1db3fd72&scene=21\#wechat_redirect}}。

分布式事务能最大限度保证了数据库操作的原子性。但在提交事务时需要协调多个节点,推后了提交事务的时间点,延长了事务的执行时间。导致事务在访问共享资源时发生冲突或死锁的概率增高。随着数据库节点的增多,这种趋势会越来越严重,从而成为系统在数据库层面上水平扩展的枷锁。


对于那些性能要求很高,但对一致性要求不高的系统,往往不苛求系统的实时一致性,只要在允许的时间段内达到最终一致性即可,可采用事务补偿的方式。与事务在执行中发生错误后立即回滚的方式不同,事务补偿是一种事后检查补救的措施,一些常见的实现方法有:对数据进行对账检查,基于日志进行对比,定期同标准数据来源进行同步等等。事务补偿还要结合业务系统来考虑。

\paragraph{跨节点关联表}

切分之前,系统中很多列表和详情页所需的数据可以通过sql join来完成。而切分之后,数据可能分布在不同的节点上,此时join带来的问题就比较麻烦了,考虑到性能,尽量避免使用join查询。

\paragraph{跨节点分页、排序、函数计算}
\paragraph{全局主键不唯一}
\paragraph{数据库扩容、数据迁移问题}

\end{document}