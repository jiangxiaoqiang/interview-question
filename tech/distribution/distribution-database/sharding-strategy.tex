\documentclass[../../../interview-questions.tex]{subfiles}

\begin{document}

\subsection{Sharding-JDBC分片策略}

\paragraph{标准分片策略}

SQL 语句中有>,>=, <=,<,=,IN 和 BETWEEN AND 操作符,都可以应用此分片策略。标准分片策略(StandardShardingStrategy),它只支持对单个分片键(字段)为依据的分库分表,并提供了两种分片算法 PreciseShardingAlgorithm(精准分片)和 RangeShardingAlgorithm(范围分片)。

\paragraph{复合分片策略}

SQL 语句中有>,>=, <=,<,=,IN 和 BETWEEN AND 等操作符,不同的是复合分片策略支持对多个分片键操作。

\paragraph{行表达式分片策略}

行表达式分片策略(InlineShardingStrategy),在配置中使用 Groovy 表达式,提供对 SQL语句中的 = 和 IN 的分片操作支持,它只支持单分片键。

行表达式分片策略适用于做简单的分片算法,无需自定义分片算法,省去了繁琐的代码开发,是几种分片策略中最为简单的。


Hint分片策略

\paragraph{一致性Hash(Consistent Hashing)}

一致性Hash是麻省理工的David Karger教授在一篇论文中提出来的,现在被用在很多分布式系统中。简单来说,一致性哈希将整个哈希值空间组织成一个虚拟的圆环,如假设某哈希函数H的值空间为$0-2^{32}$(即哈希值是一个32位无符号整形)。

https://juejin.cn/post/7096005660654960670

\end{document}