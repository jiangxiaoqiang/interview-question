\documentclass[../../../interview-questions.tex]{subfiles}

\begin{document}

\subsection{Sharding-JDBC分片策略}

\paragraph{标准分片策略}

SQL 语句中有>,>=, <=,<,=,IN 和 BETWEEN AND 操作符,都可以应用此分片策略。标准分片策略(StandardShardingStrategy),它只支持对单个分片键(字段)为依据的分库分表,并提供了两种分片算法 PreciseShardingAlgorithm(精准分片)和 RangeShardingAlgorithm(范围分片)。

\paragraph{复合分片策略}

SQL 语句中有>,>=, <=,<,=,IN 和 BETWEEN AND 等操作符,不同的是复合分片策略支持对多个分片键操作。

\paragraph{行表达式分片策略}

行表达式分片策略(InlineShardingStrategy),在配置中使用 Groovy 表达式,提供对 SQL语句中的 = 和 IN 的分片操作支持,它只支持单分片键。

行表达式分片策略适用于做简单的分片算法,无需自定义分片算法,省去了繁琐的代码开发,是几种分片策略中最为简单的。


Hint分片策略

\paragraph{一致性Hash(Consistent Hashing)}

一致性Hash是麻省理工的David Karger教授在一篇论文中提出来的,现在被用在很多分布式系统中。简单来说,一致性哈希将整个哈希值空间组织成一个虚拟的圆环,如假设某哈希函数H的值空间为$0-2^{32}$(即哈希值是一个32位无符号整形)。当然一致性哈希算法的实现不同语言有不同的实现方式,其中较为有名的一种实现叫Ketama算法,该算法最初是由Last.fm的程序员实现的并得到了广泛的应用,一些开源框架譬如spymemcached,twemproxy等都内置了该算法的实现。具体的实现例子可以参考https://github.com/jiangxiaoqiang/java-memcached-client项目的KetamaNodeLocator.java类。在实际生产中,使用了一致性Hash算法后,如果需要增加一台服务器X,只需要将其插入位置后面的服务器C的内容拷贝到服务器X上一份即可。更精细化的操作可以将服务器C的数据按照新的Hash输出结果分摊到服务器C和服务器X这两台服务器上,这样可以避免数据冗余,但操作相对繁杂一些。假如服务器C被缩容摘除,则对数据对象A、B、D不会产生任何影响。只有C对象被重定位到服务器D上。而事实上,C对象对应的Node C不存在了,因此无论如何C都会受到影响。同时,会对服务器D产生影响,因为原本分配到服务器C上的流量会被转到它的身上,使其压力增大。在具体生产操作中,如果需要缩容服务器C,则需要将该服务器C的内容拷贝到其后面的服务器D上。这样,由服务器C转接过来的流量还可以访问到对应的数据。

https://juejin.cn/post/7096005660654960670

\end{document}