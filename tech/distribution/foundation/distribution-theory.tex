\documentclass[../../../interview-questions.tex]{subfiles}

\begin{document}

\subsection{ACID/CAP/BASE}

ACID(原子性Atomicity, 一致性Consistency, 隔离性Isolation, 持久性Durability):是RDBMS中遵循的事务处理基本原则,但是也是影响其性能的原因,在NoSQL是分布式数据库一般不保证ACID原则。

CAP (一致性Consistency, 可用性Availability, 分区容错性Partition Tolerance):CAP理论是针对分布式系统而言的。CAP理论的核心是:一个分布式系统不可能同时很好的满足一致性(Consistency),可用性(Availability)和分区容错性(Partition Tolerance)这三个需求,最多只能同时较好的满足两个。

BASE(基本可用Basically Available, 软状态Soft State, 最终一致性Eventual consistency):与ACID是RDBMS强一致性的四个要求对应,BASE是NoSQL通常对可用性及一致性的弱要求原则,它们的意思分别是,BASE:Basically Available(基本可用), Soft-state(软状态/柔性事务。 "Soft state" 可以理解为"无连接"的), Eventual

\end{document}