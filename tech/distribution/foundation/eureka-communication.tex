\documentclass[../../../interview-questions.tex]{subfiles}

\begin{document}

\subsection{Eureka通信原理}

再来看看 Eureka 集群的工作原理。我们假设有三台 Eureka Server 组成的集群,第一台 Eureka Server 在北京机房,另外两台 Eureka Server 在深圳和西安机房。这样三台 Eureka Server 就组建成了一个跨区域的高可用集群,只要三个地方的任意一个机房不出现问题,都不会影响整个架构的稳定性。

从图中可以看出 Eureka Server 集群相互之间通过 Replicate 来同步数据,相互之间不区分主节点和从节点,所有的节点都是平等的。在这种架构中,节点通过彼此互相注册来提高可用性,每个节点需要添加一个或多个有效的 serviceUrl 指向其他节点。

如果某台 Eureka Server 宕机,Eureka Client 的请求会自动切换到新的 Eureka Server 节点。当宕机的服务器重新恢复后,Eureka 会再次将其纳入到服务器集群管理之中。当节点开始接受客户端请求时,所有的操作都会进行节点间复制,将请求复制到其它 Eureka Server 当前所知的所有节点中。

另外 Eureka Server 的同步遵循着一个非常简单的原则:只要有一条边将节点连接,就可以进行信息传播与同步。所以,如果存在多个节点,只需要将节点之间两两连接起来形成通路,那么其它注册中心都可以共享信息。每个 Eureka Server 同时也是 Eureka Client,多个 Eureka Server 之间通过 P2P 的方式完成服务注册表的同步。

Eureka Server 集群之间的状态是采用异步方式同步的,所以不保证节点间的状态一定是一致的,不过基本能保证最终状态是一致的。所以Eureka满足AP。

\paragraph{Eureka 保证 AP}

CAP (一致性Consistency, 可用性Availability, 分区容错性Partition Tolerance):CAP理论是针对分布式系统而言的。CAP理论的核心是:一个分布式系统不可能同时很好的满足一致性(Consistency),可用性(Availability)和分区容错性(Partition Tolerance)这三个需求,最多只能同时较好的满足两个。
Eureka Server 各个节点都是平等的,几个节点挂掉不会影响正常节点的工作,剩余的节点依然可以提供注册和查询服务。而 Eureka Client 在向某个 Eureka 注册时,如果发现连接失败,则会自动切换至其它节点。只要有一台 Eureka Server 还在,就能保证注册服务可用(保证可用性),只不过查到的信息可能不是最新的(不保证强一致性)。所以Eureka满足可用性Availability, 分区容错性Partition Tolerance。

\paragraph{Eurka 工作流程}

了解完 Eureka 核心概念,自我保护机制,以及集群内的工作原理后,我们来整体梳理一下 Eureka 的工作流程:

1、Eureka Server 启动成功,等待服务端注册。在启动过程中如果配置了集群,集群之间定时通过 Replicate 同步注册表,每个 Eureka Server 都存在独立完整的服务注册表信息

2、Eureka Client 启动时根据配置的 Eureka Server 地址去注册中心注册服务

3、Eureka Client 会每 30s 向 Eureka Server 发送一次心跳请求,证明客户端服务正常

4、当 Eureka Server 90s 内没有收到 Eureka Client 的心跳,注册中心则认为该节点失效,会注销该实例

5、单位时间内 Eureka Server 统计到有大量的 Eureka Client 没有上送心跳,则认为可能为网络异常,进入自我保护机制,不再剔除没有上送心跳的客户端

6、当 Eureka Client 心跳请求恢复正常之后,Eureka Server 自动退出自我保护模式

7、Eureka Client 定时全量或者增量从注册中心获取服务注册表,并且将获取到的信息缓存到本地

8、服务调用时,Eureka Client 会先从本地缓存找寻调取的服务。如果获取不到,先从注册中心刷新注册表,再同步到本地缓存

9、Eureka Client 获取到目标服务器信息,发起服务调用

10、Eureka Client 程序关闭时向 Eureka Server 发送取消请求,Eureka Server 将实例从注册表中删除




\end{document}