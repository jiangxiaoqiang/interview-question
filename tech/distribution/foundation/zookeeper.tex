\documentclass[../../../interview-questions.tex]{subfiles}

\begin{document}

\subsection{什么是zookeeper}

Zookeeper最早起源于雅虎研究院的一个研究小组。在当时,研究人员发现,在雅虎内部很多大型系统基本都需要依赖一个类似的系统来进行分布式协调,但是这些系统往往都存在分布式单点问题。所以,雅虎的开发人员就试图开发一个通用的无单点问题的分布式协调框架,以便让开发人员将精力集中在处理业务逻辑上。关于“ZooKeeper”这个项目的名字,其实也有一段趣闻。在立项初期,考虑到之前内部很多项目都是使用动物的名字来命名的(例如著名的Pig项目),雅虎的工程师希望给这个项目也取一个动物的名字。时任研究院的首席科学家RaghuRamakrishnan开玩笑地说:“在这样下去,我们这儿就变成动物园了!”此话一出,大家纷纷表示就叫动物园管理员吧一一一因为各个以动物命名的分布式组件放在一起,雅虎的整个分布式系统看上去就像一个大型的动物园了,而Zookeeper正好要用来进行分布式环境的协调一一于是,Zookeeper的名字也就由此诞生了。ZooKeeper 是一个典型的分布式数据一致性解决方案,分布式应用程序可以基于 ZooKeeper 实现诸如数据发布/订阅、负载均衡、命名服务、分布式协调/通知、集群管理、Master 选举、分布式锁和分布式队列等功能。

\end{document}