\documentclass[../../../interview-questions.tex]{subfiles}

\begin{document}

\subsection{分布式锁实现方式}

实现分布式锁一般是通过MySQL等数据库、ZooKeeper、Redis和自研分布式锁:如谷歌的Chubby。

\paragraph{Redis实现-基于单Redis节点的分布式锁}

获取分布式锁时,注意set操作和设置过期时间操作要保证原子性,不能分2步完成,在设置过期时间失败后,锁不会过期,将会成为死锁\footnote{\url{https://wudashan.cn/2017/10/23/Redis-Distributed-Lock-Implement/}}。

\begin{lstlisting}[language=Java]
public class RedisTool {
    private static final String LOCK_SUCCESS = "OK";
    private static final String SET_IF_NOT_EXIST = "NX";
    private static final String SET_WITH_EXPIRE_TIME = "PX";
    /**
     * 尝试获取分布式锁
     * @param jedis Redis客户端
     * @param lockKey 锁
     * @param requestId 请求标识
     * @param expireTime 超期时间
     * @return 是否获取成功
     */
    public static boolean tryGetDistributedLock(Jedis jedis, String lockKey, String requestId, int expireTime) {
        String result = jedis.set(lockKey, requestId, SET_IF_NOT_EXIST, SET_WITH_EXPIRE_TIME, expireTime);
        if (LOCK_SUCCESS.equals(result)) {
            return true;
        }
        return false;
    }
}
\end{lstlisting}

同理,解锁时也要保证操作的原子性:

\begin{lstlisting}[language=Java]
public class RedisTool {
    private static final Long RELEASE_SUCCESS = 1L;
    /**
     * 释放分布式锁
     * @param jedis Redis客户端
     * @param lockKey 锁
     * @param requestId 请求标识
     * @return 是否释放成功
     */
    public static boolean releaseDistributedLock(Jedis jedis, String lockKey, String requestId) {
        String script = "if redis.call('get', KEYS[1]) == ARGV[1] then return redis.call('del', KEYS[1]) else return 0 end";
        Object result = jedis.eval(script, Collections.singletonList(lockKey), Collections.singletonList(requestId));
        if (RELEASE_SUCCESS.equals(result)) {
            return true;
        }
        return false;
    }
}
\end{lstlisting}

存在问题:如果在并发极高的场景下,比如抢红包场景,可能存在UnixTimestamp重复问题,另外由于不能保证分布式环境下的物理时钟一致性,也可能存在UnixTimestamp重复问题,只不过极少情况下会遇到。以上实现仅在单实例的场景下是安全的,针对如何实现分布式Redis的锁,antirez提出了分布式锁算法Redlock\footnote{\url{http://zhangtielei.com/posts/blog-redlock-reasoning.html}}。

\paragraph{Redis实现-基于多Redis节点的Redlock算法实现分布式锁}

基于Redis的Redisson红锁RedissonRedLock对象实现了Redlock介绍的加锁算法。当然更好的方案是避免使用分布式锁,采用消息队列替代。

\paragraph{看门狗(Watch Dog)方式解决锁过期问题}

如果业务代码没执行完,锁却过期了,这时候其他线程又能抢锁了,线程不安全啦。所以Redisson内部有个看门狗的机制,意思是定时监测业务是否执行结束,没结束的话你这个锁是不是快到期了(超过锁的三分之一时间,比如设置的9s过期,现在还剩6s到期),那就重新续期。这样防止如果业务代码没执行完,锁却过期了所带来的线程不安全问题。看门狗方式可能会存在性能问题。

\end{document}