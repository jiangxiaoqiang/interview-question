\documentclass[../../../interview-questions.tex]{subfiles}

\begin{document}

\subsection{分布式算法}

\paragraph{2阶段提交(2 Phase Commit)}

阻塞,无法满足高并发。二阶段提交算法的成立是基于以下假设的:

在该分布式系统中,存在一个节点作为协调者(Coordinator),其他节点作为参与者(Participants),且节点之间可以进行网络通信;
所有节点都采用预写式日志,日志被写入后被保存在可靠的存储设备上,即使节点损坏也不会导致日志数据的丢失;
所有节点不会永久性损坏,即使损坏后仍然可以恢复。
两阶段提交中的两个阶段,指的是 Commit-request 阶段和 Commit 阶段

\paragraph{3阶段提交(3 Phase Commit)}

阻塞,无法满足高并发。为了解决二阶段协议中的同步阻塞等问题,三阶段提交协议在协调者和参与者中都引入了超时机制,并且把两阶段提交协议的第一个阶段拆分成了两步:询问,然后再锁资源,最后真正提交。

三阶段中的 Three Phase 分别为 CanCommit、PreCommit、DoCommit 阶段。

\paragraph{柔性事务TCC(Try、Confirm、Cancel)}

「柔」主要是相对于「传统」ACID的刚而言,柔性事务只需要遵循BASE原则。而TCC是柔性事务的一种实现。TCC是三个首字母,Try-Confirm-Cancel,具体描述是将整个操作分为上面这三步。两个微服务间同时进行Try,在Try的阶段会进行数据的校验,检查,资源的预创建,如果都成功就会分别进行Confirm,如果两者都成功则整个TCC事务完成。如果Confirm时有一个服务有问题,则会转向Cancel,相当于进行Confirm的逆向操作。Atomikos公司在商业版本事务管理器Extreme Transactions中提供了TCC方案的实现,但是由于其是收费的,因此相应的很多的开源实现方案也就涌现出来,如:tcc-transaction、ByteTCC、spring-cloud-rest-tcc。

\textbf{允许空回滚}.事务协调器在调用TCC服务的一阶段Try操作时,可能会出现因为丢包而导致的网络超时,此时事务协调器会触发二阶段回滚,调用TCC服务的Cancel操作;TCC服务在未收到Try请求的情况下收到Cancel请求,这种场景被称为空回滚;TCC服务在实现时应当允许空回滚的执行;

\textbf{防悬挂控制}.事务协调器在调用TCC服务的一阶段Try操作时,可能会出现因网络拥堵而导致的超时,此时事务协调器会触发二阶段回滚,调用TCC服务的Cancel操作;在此之后,拥堵在网络上的一阶段Try数据包被TCC服务收到,出现了二阶段Cancel请求比一阶段Try请求先执行的情况;用户在实现TCC服务时,应当允许空回滚,但是要拒绝执行空回滚之后到来的一阶段Try请求;

\textbf{幂等控制}.无论是网络数据包重传,还是异常事务的补偿执行,都会导致TCC服务的Try、Confirm或者Cancel操作被重复执行;用户在实现TCC服务时,需要考虑幂等控制,即Try、Confirm、Cancel 执行一次和执行多次的业务结果是一样的;

\textbf{业务数据可见性控制}.TCC服务的一阶段Try操作会做资源的预留,在二阶段操作执行之前,如果其他事务需要读取被预留的资源数据,那么处于中间状态的业务数据该如何向用户展示,需要业务在实现时考虑清楚;通常的设计原则是“宁可不展示、少展示,也不多展示、错展示”;

分布式事务测试集合:http://springcloud.cn/view/374

\paragraph{Paxos}

\paragraph{ZAB}

\paragraph{Raft}

\paragraph{Gossip}

\paragraph{QuorumNWR}

\paragraph{PBFT}

\paragraph{POW}

\end{document}