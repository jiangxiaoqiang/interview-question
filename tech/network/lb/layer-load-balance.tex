\documentclass[../../../interview-questions.tex]{subfiles}

\begin{document}

\subsection{四层负载均衡和七层负载均衡}

所说的“四层”、“七层”,指的是经典的OSI 七层模型中第四层传输层和第七层应用层。真正大型系统的负载均衡过程往往是多级的。譬如,在各地建有多个机房,或机房有不同网络链路入口的大型互联网站,会从 DNS 解析开始,通过“域名” → “CNAME” → “负载调度服务” → “就近的数据中心入口”的路径,先将来访地用户根据 IP 地址(或者其他条件)分配到一个合适的数据中心中,然后才到稍后将要讨论的各式负载均衡。在 DNS 层面的负载均衡与前面介绍的 DNS 智能线路、内容分发网络等,在工作原理上是类似的,其差别只是数据中心能提供的不仅有缓存,而是全方位的服务能力。

所谓四层负载均衡,也就是主要通过报文中的目标地址和端口,再加上负载均衡设备设置的服务器选择方式,决定最终选择的内部服务器。
以常见的TCP为例,负载均衡设备在接收到第一个来自客户端的SYN 请求时,即通过上述方式选择一个最佳的服务器,并对报文中目标IP地址进行修改(改为后端服务器IP),直接转发给该服务器。TCP的连接建立,即三次握手是客户端和服务器直接建立的,负载均衡设备只是起到一个类似路由器的转发动作。在某些部署情况下,为保证服务器回包可以正确返回给负载均衡设备,在转发报文的同时可能还会对报文原来的源地址进行修改。

所谓七层负载均衡,也称为“内容交换”,也就是主要通过报文中的真正有意义的应用层内容,再加上负载均衡设备设置的服务器选择方式,决定最终选择的内部服务器。

https://server.51cto.com/article/703456.html

\end{document}