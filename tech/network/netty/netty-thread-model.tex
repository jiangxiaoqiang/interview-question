\documentclass[../../../interview-questions.tex]{subfiles}

\begin{document}

\subsection{Netty线程模型}

https://juejin.cn/post/6999225355374428168。
Netty通过Reactor模型基于多路复用器接收并处理用户请求,内部实现了两个线程池,boss线程池和work线程池,其中boss线程池的线程负责处理请求的accept事件,当接收到accept事件的请求时,把对应的socket封装到一个NioSocketChannel中,并交给work线程池,其中work线程池负责请求的read和write事件,由对应的Handler处理。

单线程模型:所有I/O操作都由一个线程完成,即多路复用、事件分发和处理都是在一个Reactor线程上完成的。既要接收客户端的连接请求,向服务端发起连接,又要发送/读取请求或应答/响应消息。一个NIO 线程同时处理成百上千的链路,性能上无法支撑,速度慢,若线程进入死循环,整个程序不可用,对于高负载、大并发的应用场景不合适。

多线程模型:有一个NIO 线程(Acceptor) 只负责监听服务端,接收客户端的TCP 连接请求;NIO 线程池负责网络IO 的操作,即消息的读取、解码、编码和发送;1 个NIO 线程可以同时处理N 条链路,但是1 个链路只对应1 个NIO 线程,这是为了防止发生并发操作问题。但在并发百万客户端连接或需要安全认证时,一个Acceptor 线程可能会存在性能不足问题。

主从多线程模型:Acceptor 线程用于绑定监听端口,接收客户端连接,将SocketChannel 从主线程池的Reactor 线程的多路复用器上移除,重新注册到Sub 线程池的线程上,用于处理I/O 的读写等操作,从而保证mainReactor只负责接入认证、握手等操作;

\end{document}