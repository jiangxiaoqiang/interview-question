\documentclass[../../../interview-questions.tex]{subfiles}

\begin{document}

\subsection{Netty零拷贝}

https://juejin.cn/post/6999225355374428168。
介绍完传统 Linux 的零拷贝技术之后,我们再来学习下 Netty 中的零拷贝如何实现。Netty 中的零拷贝和传统 Linux 的零拷贝不太一样。Netty 中的零拷贝技术除了操作系统级别的功能封装,更多的是面向用户态的数据操作优化,主要体现在以下 5 个方面:

\begin{itemize}
    \item {堆外内存,避免 JVM 堆内存到堆外内存的数据拷贝。}
    \item {CompositeByteBuf 类,可以组合多个 Buffer 对象合并成一个逻辑上的对象,避免通过传统内存拷贝的方式将几个 Buffer 合并成一个大的 Buffer。}
    \item {通过 Unpooled.wrappedBuffer 可以将 byte 数组包装成 ByteBuf 对象,包装过程中不会产生内存拷贝。}
    \item {ByteBuf.slice 操作与 Unpooled.wrappedBuffer 相反,slice 操作可以将一个 ByteBuf 对象切分成多个 ByteBuf 对象,切分过程中不会产生内存拷贝,底层共享一个 byte 数组的存储空间。}
    \item {Netty 使用 FileRegion 实现文件传输,FileRegion 底层封装了 \url{FileChannel.transferTo()} 方法,可以将文件缓冲区的数据直接传输到目标 Channel,避免内核缓冲区和用户态缓冲区之间的数据拷贝,这属于操作系统级别的零拷贝。}
\end{itemize}

\newpage

问题列表:

\begin{enumerate}
    \item {能简单的介绍一下你自己吗}
    \item {对比搜索引擎,你的优势在哪里}
    \item {能详细说明一下你从接收到问题到回答的整个工作流程吗}
    \item {请介绍一下你的特点有哪些}
    \item {你的训练数据最新时间是}
    \item {mac下常用的视频剪辑软件有哪些}
    \item {有开源的剪辑软件可供选择吗}
    \item {身体里的结石会自己清除吗}
    \item {结石平时有哪些注意事项}
    \item {你知道Gamma Skies吗}
    \item {我指的是一位叫Gamma Skies的歌手}
    \item {韩国的女权主义为什么比较严重}
    \item {中国男足有可能冲出亚洲吗}
\end{enumerate}


理解语义、能关联上下文、中文的支持很好


\newpage


\end{document}