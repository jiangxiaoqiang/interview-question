\documentclass[../../../interview-questions.tex]{subfiles}

\begin{document}

\subsection{PostgreSQL高可用方案}

\paragraph{流复制(streaming replication)}

流复制是PostgreSQL自带的一种高可用集群方案。它通过将主节点上的写操作同步到一个或多个从节点,以实现数据的备份和故障恢复。在流复制中,从节点会与主节点建立一个流式复制的连接,并不断地从主节点接收写操作,并将它们应用到本地的数据库。流复制的主要优点是可靠性高、易于设置和维护,适用于高可用性要求不是特别高的场景。

\paragraph{Pgpool-II}

Pgpool-II是一个PostgreSQL的连接池和负载平衡器,它可以用来实现高可用性和负载均衡。Pgpool-II可以将客户端连接分发到多个PostgreSQL服务器上,并自动处理主节点故障时的故障转移。Pgpool-II的主要优点是可靠性高、易于设置和维护,适用于需要高可用性和负载均衡的场景。

\paragraph{Patroni + Etcd 方案}

使用Patroni管理本地库,并结合Etcd作为数据存储和主节点选举,具有以下优势:

健壮性: 使用分布式key-value数据库作为数据存储,主节点故障时进行主节点重新选举,具有很强的健壮性。
支持多种复制方式: 基于内置流复制,支持同步流复制、异步流复制、级联复制。
支持主备延迟设置: 可以设置备库延迟主库WAL的字节数,当备库延迟大于指定值时不做故障切换。
自动化程度高: 1)支持自动化初始PostgreSQL实例并部署流复制; 2)当备库实例关闭后,支持自动拉起; 3)当主库实例关闭后,首先会尝试自动拉起; 4)支持switchover命令,能自动将老的主库进行角色转换。
避免脑裂: 数据库信息记录到 ETCD 中,通过优化部署策略(多机房部署、增加实例数)可以避免脑裂。

https://postgres.fun/20200529182600.html

https://github.com/zalando/patroni

\end{document}