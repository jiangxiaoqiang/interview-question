\documentclass[../../../interview-questions.tex]{subfiles}

\begin{document}

\subsection{MySQL集群高可用方案}

https://docker.shujuwajue.com/mysql-ji-qun/chang-jian-mysql-ji-qun-fang-an

\paragraph{MMM}

Multi Master Replication Manager。MMM(Multi-Master Replication Manager)是一组灵活的脚本,用于执行 MySQL 主-主复制配置的监视/故障转移和管理(任何时候只有一个节点可写)。

\paragraph{Percona XtraDB Cluster }

Percona XtraDB Cluster (简称 PXC) 是 Percona 公司开源的实现 MySQL 高可用的解决方案。它将 Percona Server 和 Percona XtraBackup 与 Galera 库集成,以实现多主同步复制。和 MySQL 传统的异步复制相比,它能够保证数据的强一致性,在任何时刻集群中任意节点上的数据状态都是完全一致的,并且整个架构实现了去中心化,所有节点都是对等的,即允许你在任意节点上进行写入和读取,集群会把数据状态同步至其他所有节点。但目前 PXC 集群只支持 InnoDB 存储引擎,并具有以下限制:

\begin{enumerate}
    \item {添加新节点时,必须从现有节点之一复制完整数据集。如果是 100GB,则复制 100GB。为了减少网络开销,建议在搭建集群前使用备份的方式将所有节点初始的数据状态调整至一致。}
    \item {不支持 LOCK TABLES ,在多主设置的情况下也不支持 UNLOCK TABLES。}
    \item {不支持锁定功能,如 GET\_LOCK(),RELEASE\_LOCK() 等。}
    \item {由于可能在提交时回滚,因此也不支持 XA 事务 (分布式事务) 。}
    \item {所有表必须具有主键。}
    \item {由于节点是对等的,所以整个集群的写吞吐量受限于性能最差的节点,如果一个节点变慢,则整个群集都会变慢。因此应该保证所有节点的硬件配置一致,并避免单个节点超负载运行。}
    \item {允许的最大事务大小由 wsrep\_max\_ws\_rows 和 wsrep\_max\_ws\_size 参数共同定义,因此超大型事务会被拆分为一系列小型事务,如加载大数据集 LOAD DATA INFILE。}
    \item {由于在集群级别采用乐观锁进行并发控制,所以事务在 COMMIT 阶段仍然有被中止的可能。如两个事务在不同的集群节点上提交对相同的行的写入,此时只有其中一个可以成功提交,另一个将被中止。}
\end{enumerate}

\paragraph{Master High Availability Manager and Tools for MySQL}

MHA(Master High Availability)由原MySQL团队(Sun时代)的Yoshinori Matsunobu开发。可以实现故障切换和主从提升功能。在 MySQL 故障切换过程中,MHA 能做到在 0~30 秒之内自动完成数据库的故障切换操作,并且在进行故障切换的过程中,MHA 能在最大程度上保证数据的一致性,以达到真正意义上的高可用。MHA(Master High Availability Manager and Tools for MySQL)目前在MySQL高可用方面是一个相对成熟的解决方案。它是日本的一位MySQL专家采用Perl语言编写的一个脚本管理工具,该工具仅适用于MySQL Replication环境,目的在于维持Master主库的高可用性。

MHA是基于标准的MySQL复制(异步/半同步)。

MHA是由管理节点(MHA Manager)和数据节点(MHA Node)两部分组成。

MHA Manager可以单独部署在一台独立机器,也可以部署在一台slave上。

https://tech.meituan.com/2017/06/29/database-availability-architecture.html

MHA曾经非常流行,但随着MySQL官方的高可用方案不断推出,作者已经意识到,曾经MHA所解决的问题,已经逐渐被官方的解决方案所代替,因此,从MySQL8.0开始,作者已经不在对MHA进行开发和维护。

\paragraph{MGR: MySQL Group Replication}


\paragraph{Gelera Cluster}

Galera Cluster 是一个基于 MySQL 的高可用性解决方案,可以提供同步多主节点的数据复制和自动故障转移功能。Galera Cluster是由Codership开发的MySQL多主结构集群,这些主节点互为其它节点的从节点。不同于MySQL原生的主从异步复制,Galera采用的是多主同步复制,并针对同步复制过程中,会大概率出现的事务冲突和死锁进行优化,就是复制不基于官方binlog而是Galera复制插件,重写了wsrep api。

异步复制中,主库将数据更新传播给从库后立即提交事务,而不论从库是否成功读取或重放数据变化。这种情况下,在主库事务提交后的短时间内,主从库数据并不一致。

同步复制时,主库的单个更新事务需要在所有从库上同步 更新。换句话说,当主库提交事务时,集群中所有节点的数据保持一致。

对于读操作,从每个节点读取到的数据都是相同的。对于写操作,当数据写入某一节点后,集群会将其同步到其它节点。

\end{document}