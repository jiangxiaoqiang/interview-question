\documentclass[../../../interview-questions.tex]{subfiles}

\begin{document}

\subsection{MySQL存储引擎}

MySQL存储引擎主要有: \paragraph{MyISAM} 

如果数据表主要用来插入和查询记录,则MyISAM(但是不支持事务)引擎能提供较高的处理效率

\paragraph{Mrg\_MyISAM} 

MERGE存储引擎把一组MyISAM数据表当做一个逻辑单元来对待,让我们可以同时对他们进行查询。Merge表有点类似于视图。使用Merge存储引擎实现MySQL分表,这种方法比较适合那些没有事先考虑分表,随着数据的增多,已经出现了数据查询慢的情况。这个时候如果要把已有的大数据量表分开比较痛苦,最痛苦的事就是改代码。所以使用Merge存储引擎实现MySQL分表可以避免改代码。Merge引擎下每一张表只有一个MRG文件。MRG里面存放着分表的关系,以及插入数据的方式。它就像是一个外壳,或者是连接池,数据存放在分表里面。merge合并表的要求:

\begin{enumerate}
\item{合并的表使用的必须是MyISAM引擎}
\item{表的结构必须一致,包括索引、字段类型、引擎和字符集}
\item{对于增删改查,直接操作总表即可。}
\end{enumerate}


\paragraph{Memory}

MEMORY存储引擎将表中的数据存储到内存中,为查询和引用其他表数据提供快速访问。使用存在内存中的内容来创建表。每个MEMORY表只实际对应一个磁盘文件。MEMORY类型的表访问非常得快,因为它的数据是放在内存中的,并且默认使用HASH索引。但是一旦服务关闭,表中的数据就会丢失掉。 HEAP允许只驻留在内存里的临时表格。驻留在内存里让HEAP要比ISAM和MYISAM都快,但是它所管理的数据是不稳定的,而且如果在关机之前没有进行保存,那么所有的数据都会丢失。在数据行被删除的时候,HEAP也不会浪费大量的空间。HEAP表格在你需要使用SELECT表达式来选择和操控数据的时候非常有用。


\paragraph{Blackhole}

MySQL在5.x系列提供了Blackhole引擎–“黑洞”。其表现就像一个黑洞,只进不出,进来就消失。换句话说,任何往其中写的数据都将丢失,有点像Linux的/dev/null。
比如一个表test的引擎是Blackhole,任何对这个表的insert都将丢失,
对它的select永远返回空集,对应的数据目录下只有一个test.frm文件,且没有其他文件与之关联。关键在于,虽然其不保存数据,但对数据库的操作仍旧记录在binlog日志中。
这就带来一个好处,可以将其作为主从复制的中介,将原来从主库中同步的操作变为从作为中介的BlackHole引擎数据库中同步。众所周知,当从库比较多的时候,所有从库都从主库load数据将加重主库的负担。但如果是从BlackHole的伪主库中同步就可以减轻主库的负担。


\paragraph{CSV}

CSV存储引擎可以将csv文件作为MySQL的表进行处理。存储格式就是普通的csv文件。    适合做为数据交换的中间表(能够在服务器运行的时候,拷贝和拷出文件,可以将电子表格存储为CSV文件再拷贝到MySQL数据目录下,就能够在数据库中打开和使用。同样,如果将数据写入到CSV文件数据表中,其它web程序也可以迅速读取到数据。


\paragraph{Performance\_Schema}

MySQL Performance Schema  用于监视MySQL服务器,且运行时消耗很少的性能。Performance Schema 收集数据库服务器性能参数,并且表的存储引擎均为PERFORMANCE\_SCHEMA,而用户是不能创建存储引擎为PERFORMANCE\_SCHEMA的表。Performance Schema 具有以下特征:

Performance Schema 提供了一种在服务器运行时检查服务器的内部执行的方法。它使用PERFORMANCE\_SCHEMA存储引擎和performance\_schema数据库实现。性能模式主要关注性能数据。这与用于检查元数据的INFORMATION\_SCHEMA不同。

Performance Schema 事件特定于MySQL服务器的给定实例。 Performance Schema 表被视为本地服务器,并且对其进行的更改不会被复制或写入二进制日志。

Performance Schema 中的表是内存表,不使用磁盘存储,在 datadir 的 performance\_schema 目录下,只有.frm表结构文件,没有数据文件。表内容在服务器启动时重新填充,并在服务器关闭时丢弃。

数据收集是通过修改服务器源代码来实现的。 不同于其他功能(如复制或Event Scheduler),不存在与Performance Schema相关联的单独线程。

服务器监控持续不中断地进行,花费很少。 开启Performance Schema不会使服务器不可用。

 
\paragraph{Archive}

如果只有INSERT和SELECT操作,可以选择Archive,Archive支持高并发的插入操作,但是本身不是事务安全的。Archive非常适合存储归档数据,如记录日志信息可以使用Archive

\paragraph{Federated}

Federate存储引擎也是mysql比较常用的存储引擎,使用它可以访问远程的mysql数据库上的表,这种引擎的作用类似于oracle数据库的dblink,以mysql5.5为例默认是不启用federated引擎的,可以使用INSTALL PLUGIN plugin\_name SONAME 'shared\_library\_name'语句动态加载。

\paragraph{InnoDB}

\end{document}