\documentclass[../../../interview-questions.tex]{subfiles}

\begin{document}

\subsection{MySQL索引覆盖}

覆盖索引(covering index)指一个查询语句的执行只用从索引中就能够取得,不必从数据表中读取。也可以称之为实现了索引覆盖。如果一个索引包含了(或覆盖了)满足查询语句中字段与条件的数据就叫做覆盖索引。当一条查询语句符合覆盖索引条件时,sql只需要通过索引就可以返回查询所需要的数据,这样避免了查到索引后再返回表操作,减少I/O提高效率。
使用覆盖索引Innodb比MyISAM效果更好----InnoDB使用聚集索引组织数据,如果二级索引中包含查询所需的数据,就不再需要在聚集索引中查找了(一般的查询流程是从二级索引找到主键,再回表查询,索引覆盖就是直接从二级索引中获取到需要的数据)。聚集索引就是存放的物理顺序和列中的顺序一样。一般设置主键索引就为聚集索引。每次给字段建一个新索引, 字段中的数据就会被复制一份出来, 用于生成索引。 因此, 给表添加索引,会增加表的体积, 占用磁盘存储空间。通过聚集索引可以一次查到需要查找的数据, 而通过非聚集索引第一次只能查到记录对应的主键值 , 再使用主键的值通过聚集索引查找到需要的数据。聚集索引一张表只能有一个,而非聚集索引一张表可以有多个。https://stackoverflow.com/questions/1251636/what-do-clustered-and-non-clustered-index-actually-mean

\end{document}