\documentclass[../../../interview-questions.tex]{subfiles}

\begin{document}

\subsection{组合索引的好处}

https://www.jianshu.com/p/XgXfhf

\paragraph{减少磁盘IO}如果采用组合索引,那么当需要更新索引时,只需要更新组合索引即可。如果是独立的索引,那么就需要更新多个索引,会增加磁盘IO。

\paragraph{覆盖索引}同样的有复合索引(a,b,c),如果有如下的sql: select a,b,c from table where a=1 and b = 1。那么MySQL可以直接通过遍历索引取得数据,而无需回表,这减少了很多的随机io操作。减少io操作,特别的随机io其实是dba主要的优化策略。所以,在真正的实际应用中,覆盖索引是主要的提升性能的优化手段之一。

\paragraph{高效筛选}

多列索引时,SQL只能应用其中的一个索引,而不是应用多个索引\footnote{\url{https://stackoverflow.com/questions/2349817/two-single-column-indexes-vs-one-two-column-index-in-mysql}}。索引列越多,通过索引筛选出的数据越少。有1000W条数据的表,有如下sql:select * from table where a = 1 and b = 2 and c = 3,假设假设每个条件可以筛选出10\%的数据,如果只有单值索引,那么通过该索引能筛选出1000W*10\%=100w 条数据,然后再回表从100w条数据中找到符合b=2 and c= 3的数据,然后再排序,再分页;如果是复合索引,通过索引筛选出1000w *10\% *10\% *10\%=1w,然后再排序、分页,哪个更高效,一眼便知。

\paragraph{减少索引占用空间}

相比单列索引,组合索引可以用较少的索引来达到更好的查询性能,因此可以减少索引占用的空间。

\end{document}