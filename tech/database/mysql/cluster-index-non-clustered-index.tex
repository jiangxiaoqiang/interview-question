\documentclass[../../../interview-questions.tex]{subfiles}

\begin{document}

\subsection{聚集索引和非聚集索引}

聚簇索引是对磁盘上实际数据重新组织以按指定的一个或多个列的值排序的算法。特点是存储数据的顺序和索引顺序一致。一般情况下主键会默认创建聚簇索引,且一张表只允许存在一个聚簇索引(理由:数据一旦存储,顺序只能有一种)。聚簇索引(innodb)的叶子节点就是数据节点 而非聚簇索引(myisam)的叶子节点仍然是索引文件,只是这个索引文件中包含指向对应数据块的指针。在《数据库原理》一书中是这么解释聚簇索引和非聚簇索引的区别的:聚簇索引的叶子节点就是数据节点,而非聚簇索引的叶子节点仍然是索引节点,只不过有指向对应数据块的指针。
https://stackoverflow.com/questions/1251636/what-do-clustered-and-non-clustered-index-actually-mean

\end{document}