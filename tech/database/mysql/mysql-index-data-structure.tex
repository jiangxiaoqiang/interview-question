\documentclass[../../../interview-questions.tex]{subfiles}

\begin{document}

\subsection{MySQL索引的数据结构}

B+树。B+树是一种数据结构,是一个N叉排序树,每个节点通常有多个孩子,一棵B+树包含根节点(Root Node)、内部节点(Internal Node)和叶子节点(Leaf Node)。根节点可能是一个叶子节点, 也可能是一个包含两个或两个以上孩子节点的节点。B+树通常用于数据库和操作系统的文件系统中。NTFS、ReiserFS、NSS、XFS、JFS、ReFS和BFS等文件系统都在使用B+树作为元数据索引。为什么要用B+树呢?B+树的特点是能够保持数据稳定有序,其插入与修改拥有较稳定的对数时间复杂度(插入、删除、检索平均复杂度O($\log N$))。B+树元素自底向上插入\footnote{参见:\url{https://ivanzz1001.github.io/records/post/data-structure/2018/06/16/ds-bplustree}}。B+树的磁盘读写代价更低。B+树的查询效率更加稳定。阶数(Degree)表示此树的节点最多有多少个孩子结点(子树),一般用字母 M 表示阶数。总结B+树的优势如下:

1、B+树的层级更少,查询速度更快:相较于B树B+每个非叶子节点也就是索引节点也叫内部节点存储的键(Key)数更多,树的层级更少所以查询数据更快;

2、B+树查询速度更稳定:B+所有关键字数据地址都存在叶子节点上,所以每次查找的次数都相同所以查询速度要比B树更稳定;

3、B+树天然具备排序功能:B+树所有的叶子节点数据构成了一个有序链表,在查询大小区间的数据时候更方便,数据紧密性很高,缓存的命中率也会比B树高。

4、B+树全节点遍历更快:B+树遍历整棵树只需要遍历所有的叶子节点即可,而不需要像B树一样需要对每一层进行遍历,这有利于数据库做全表扫描。

B树相对于B+树的优点是,如果经常访问的数据离根节点很近,而B树的非叶子节点本身存有关键字其数据的地址,所以这种数据检索的时候会要比B+树快。

\end{document}