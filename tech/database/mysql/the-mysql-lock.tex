\documentclass[../../../interview-questions.tex]{subfiles}

\begin{document}

\subsection{\color{red}{MySQL锁}}

MySQL的锁属于基础类型的知识,必须要掌握。在MySQL当中,关于InnoDB的锁类型总共可以分为四种,分别是:

\begin{enumerate}
    \item{基本锁 - [ 共享锁(Shared Locks:S锁)和排它锁(Exclusive Locks:X锁)]}
    \item{意向锁 - [ intention lock,分为意向共享锁(IS锁)和意向排他锁(IX锁)]}为什么要用意象锁?如果真要到用表锁的时候,那表锁和行锁之间不是会冲突的吗?如果表里面已经加了行锁怎么办?得一条记录一条记录遍历过去找行锁吗?这确实是一种实现方式,但是性能太差了,假设数据库里有上千万的数据,这加个表锁将会非常慢。所以有了个叫意向锁(Intention Locks)的东西。这两个锁是表级别的锁,当需要对表中的某条记录上 S 锁的时候,先在表上加个 IS 锁,表明此时表内有 S 锁。当需要对表中的某条记录上 X 锁的时候,先在表上加个 IX 锁,表明此时表内有 X 锁。这样操作之后,如果要加表锁,就不需要遍历所有记录去找了,直接看看表上面有没有 IS 和 IX 锁\footnote{内容来源:\url{https://zhuanlan.zhihu.com/p/388691518}}。
    \item{行锁 - [ record Locks、gap locks、next-key locks、Insert Intention Locks ]}记录锁需要加到记录上,但是如果要给此时还未存在的记录加锁怎么办?也就是要预防幻读的出现!这时候间隙锁就派上用场了,它是给间隙加上锁。间隙锁的唯一目的就是防止其他事务插入数据到间隙中 ,所以即使两个间隙锁要锁住相同的间隙也没有关系,因为它们的目的是一致的,所以不冲突。插入意向锁,即 Insert Intention Locks,它也是一类间隙锁,但是它不是锁定间隙,而是等待某个间隙。比如上面举例的 id = 4 的那个事务 C ,由于被间隙锁给阻塞了,所以事务 C 会生成一个插入意向锁,表明等待这个间隙锁的释放。间隙锁和行锁合称next-key lock,每个next-key lock是前开后闭区间。
    \item{自增锁 - [ auto-inc locks ] }Auto-Inc Lock 是一个特殊的表级锁,用于自增列插入数据时使用。 在插入一条数据的时候,需要在表上加个 Auto-Inc Lock,然后为自增列分配递增的值,在语句插入结束之后,再释放 Auto-Inc Lock。在 MySQL 5.1.22 版本之后,又弄了个互斥量来进行自增减的累加。互斥量的性能高于 Auto-Inc Lock,因为 Auto-Inc Lock是语句插入完毕之后才释放锁,而互斥量是在语句插入的时候,获得递增值之后,就可以释放锁,所以性能更好。但是我们还需要考虑主从的情况,由于并发插入的情况,基于 statement -based binlog 复制时,自增的值顺序无法把控,可能会导致主从数据不一致。所以 MySQL 有个 innodb\_autoinc\_lock\_mode 配置,一共有三个值:
    
    0,只用 Auto-Inc Lock。
    1,默认值,对于插入前已知插入行数的插入,用互斥量,对于插入前不知道具体插入数的插入,用 Auto-Inc Lock,这样即使基于 statement -based binlog 复制也是安全的。
    2,只用互斥量。
    \item {空间索引的谓词锁 - [Predicate Locks for Spatial Indexes]}InnoDB 是支持空间数据的,所以有空间索引,为了处理涉及空间索引的操作的锁定,next-key locking 不好使,因为多维数据中没有绝对排序的概念,因此不清楚“下一个” key 在哪。所以为了支持具有空间索引的表的隔离级别,InnoDB使用谓词锁。空间索引包含最小边界矩形(MBR)值,因此 InnodB 通过在用于查询的 MBR 值上设置谓词锁定,使得 InnoDB 在索引上执行一致性读, 其他事务无法插入或修改与查询条件匹配的行。
\end{enumerate}

基本锁如下:

\begin{enumerate}
    \item{读锁(共享锁)}「共享锁(S)」:又称读锁 (read lock),是读取操作创建的锁。其他用户可以并发读取数据, 但任何事务都不能对数据进行修改(获取数据上的排他锁),直到已释放所有共享锁。当如果事务对读锁进行修改操作,很可能会造成死锁。注意平时使用的select语句默认是没有加读锁的,实际是快照读(snapshot read),利用的是MVCC(Multiversion concurrency control)机制\footnote{参见:\url{https://www.wikiwand.com/en/Multiversion_concurrency_control}},所以此时其他事务是可以修改对应的数据的。但我们读到的数据可能是历史数据,是不及时的数据,不是数据库当前的数据\footnote{参见:\url{https://tech.meituan.com/2014/08/20/innodb-lock.html}}。
    \item{写锁(排他锁或互斥锁)}「排他锁(X)」:exclusive lock(也叫writer lock)又称写锁。 若某个事物对某一行加上了排他锁,只能这个事务对其进行读写,在此事务结束之前, 其他事务不能对其进行加任何锁,其他进程可以读取,不能进行写操作,需等待其释放。 「排它锁是悲观锁的一种实现」。
\end{enumerate}

按数据操作的粒度来分:

\begin{enumerate}
    \item{表锁}
    \item{页锁}
    \item{行锁}
\end{enumerate}

\end{document}