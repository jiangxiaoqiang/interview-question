\documentclass[../../../interview-questions.tex]{subfiles}

\begin{document}

\subsection{MySQL锁}

按数据操作的类型来分\footnote{参见:\url{https://www.cnblogs.com/myitnews/p/13698029.html}}:

\begin{enumerate}
    \item{读锁(共享锁)}「共享锁(S)」:又称读锁 (read lock),是读取操作创建的锁。其他用户可以并发读取数据, 但任何事务都不能对数据进行修改(获取数据上的排他锁),直到已释放所有共享锁。当如果事务对读锁进行修改操作,很可能会造成死锁。
    \item{写锁(排他锁或互斥锁)}「排他锁(X)」:exclusive lock(也叫writer lock)又称写锁。 若某个事物对某一行加上了排他锁,只能这个事务对其进行读写,在此事务结束之前, 其他事务不能对其进行加任何锁,其他进程可以读取,不能进行写操作,需等待其释放。 「排它锁是悲观锁的一种实现」。
\end{enumerate}

按数据操作的粒度来分:

\begin{enumerate}
    \item{表锁}
    \item{页锁}
    \item{行锁}
\end{enumerate}



\end{document}