\documentclass[../../../interview-questions.tex]{subfiles}

\begin{document}

\subsection{\color{red}{MySQL的锁类型}}

MySQL中常见的锁类型有以下几种:

\begin{itemize}
    \item {行锁(row lock):锁定表中某一行数据,只允许一个事务修改该行数据,其他事务需要等待}
    \item {表锁(table lock):锁定整个表,其他事务无法修改该表中的任何数据,只有等待该事务完成之后才能进行操作。}
    \item {共享锁(shared lock):允许多个事务读取同一行数据,但只允许一个事务修改该行数据。}
    \item {排它锁(exclusive lock):禁止其他事务读取或修改被锁定的行数据,只允许当前事务对该行数据进行读写操作。}
    \item {意向锁(intention lock):表示事务在对表中某个范围的行进行操作时,需要对该范围进行锁定,以避免其他事务同时对范围内的行进行操作。}
    \item {记录锁(record lock):锁定行数据的一个部分,只允许一个事务修改该部分数据,但不会锁定整个行数据。}
    \item {间隙锁(gap lock):锁定两个记录之间的间隙,防止其他事务向该间隙中插入数据,以避免幻读(phantom read)的问题。}
\end{itemize}



基本锁如下:

\begin{enumerate}
    \item{读锁(共享锁)}「共享锁(S)」:又称读锁 (read lock),是读取操作创建的锁。其他用户可以并发读取数据, 但任何事务都不能对数据进行修改(获取数据上的排他锁),直到已释放所有共享锁。当如果事务对读锁进行修改操作,很可能会造成死锁。注意平时使用的select语句默认是没有加读锁的,实际是快照读(snapshot read),利用的是MVCC(Multiversion concurrency control)机制\footnote{参见:\url{https://www.wikiwand.com/en/Multiversion_concurrency_control}},所以此时其他事务是可以修改对应的数据的。但我们读到的数据可能是历史数据,是不及时的数据,不是数据库当前的数据\footnote{参见:\url{https://tech.meituan.com/2014/08/20/innodb-lock.html}}。
    \item{写锁(排他锁或互斥锁)}「排他锁(X)」:exclusive lock(也叫writer lock)又称写锁。 若某个事物对某一行加上了排他锁,只能这个事务对其进行读写,在此事务结束之前, 其他事务不能对其进行加任何锁,其他进程可以读取,不能进行写操作,需等待其释放。 「排它锁是悲观锁的一种实现」。
\end{enumerate}

锁在MySQL 8.0源代码storage/innobase/include/lock0types.h中的定义如下:

\begin{lstlisting}[language=C]
/* Basic lock modes */
enum lock_mode {
    LOCK_IS = 0,          /* intention shared */
    LOCK_IX,              /* intention exclusive */
    LOCK_S,               /* shared */
    LOCK_X,               /* exclusive */
    LOCK_AUTO_INC,        /* locks the auto-inc counter of a table
                        in an exclusive mode */
    LOCK_NONE,            /* this is used elsewhere to note consistent read */
    LOCK_NUM = LOCK_NONE, /* number of lock modes */
    LOCK_NONE_UNSET = 255
};
\end{lstlisting}

\end{document}