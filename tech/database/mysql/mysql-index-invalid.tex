\documentclass[../../../interview-questions.tex]{subfiles}

\begin{document}

\subsection{MySQL索引失效场景}

\begin{enumerate}
\item {or语句前后没有同时使用索引。当or左右查询字段只有一个是索引,该索引失效,只有当or左右查询字段均为索引时,才会生效;}
\item {复合索引未用左列字段,即不是使用第一列索引,索引失效;}
\item {like以\%开头,当like前缀没有\%,后缀有\%时,索引有效;}
\item {需要类型转换;}
\item {where中索引列有运算,或者索引列使用了函数;}
\item {where中在索引字段上使用not,<>,!=。}

  (不等于操作符是永远不会用到索引的,因此对它的处理只会产生全表扫描。key<>0 改为 key>0 or key<0。)不等于操作符是永远不会用到索引的,因此对它的处理只会产生全表扫描。优化方法:key<>0 改为 key>0 or key<0。)
\item {如果mysql觉得全表扫描更快时(数据少);}
\item {在索引列上使用 IS NULL 或 IS NOT NULL操作。}
(索引是不索引空值的,所以这样的操作不能使用索引,可以用其他的办法处理,例如:数字类型,判断大于0,字符串类型设置一个默认值,判断是否等于默认值即可。)
\end{enumerate}

\end{document}