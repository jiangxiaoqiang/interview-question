\documentclass[../../../interview-questions.tex]{subfiles}

\begin{document}

\subsection{InnoDB如何保证事务}

数据库通过原子性(A)、隔离性(I)、持久性(D)来保证一致性(C)。其中一致性是目的,原子性、隔离性、持久性是手段。因此数据库必须实现AID三大特性才有可能实现一致性。https://www.51cto.com/article/705237.html

\paragraph{Atomicity原子性}

事务的原子性指一个事务(transaction)中的所有操作,要么全部完成,要么全部不完成,不会结束在中间某个环节。事务在执行过程中发生错误,会被回滚(Rollback)到事务开始前的状态,就像这个事务从来没有执行过一样。

原子性是由undo log日志保证的,它记录了需要回滚的日志信息,也就是说我们的事务还没提交需要回滚,那么事务回滚就是根据undo log日志来撤销已经执行成功的SQL。说白了,undo log其实就是SQL的反向执行,它记录了反向执行的SQL语句,把正向语句回滚回去。

\paragraph{Consistency一致性}

事务的一致性指的是在一个事务执行之前和执行之后数据库都必须处于一致性状态。

如果事务成功地完成,那么系统中所有变化将正确地应用,系统处于有效状态;如果在事务中出现错误,那么系统中的所有变化将自动地回滚,系统返回到原始状态。

一致性是ACID的目的,也就是说,只需要保证原子性、隔离性、持久性,自然也就保证了数据的一致性。

\paragraph{Isolation隔离性}

事务的隔离性指的是在并发环境中,当不同的事务同时操纵相同的数据时,每个事务都有各自的完整数据空间,由并发事务所做的修改必须与任何其他并发事务所做的修改隔离。

事务查看数据更新时,数据所处的状态要么是另一事务修改它之前的状态,要么是另一事务修改它之后的状态,事务不会查看到中间状态的数据。

在MySQL中隔离性是通过MVCC多版本并发控制机制来保证的,它是在事务隔离级别中最最重要的一个概念,那它是怎么实现的呢?

多版本并发控制:读取数据时通过一种类似快照的方式将数据保存下来,这样读锁和写锁就不冲突了,不同事务的session会看到自己特定版本的数据,也就是版本链,通过版本链的概念来达到读和写能够并发进行。

MVCC只在READ COMMITTED(已提交读)和REPEATABLE READ(可重复读)两个隔离级别下工作,其他两个隔离级别和MVCC不兼容,这是因为READ UNCOMMITTED(读未提交)总是读取最新的数据行,而不是符合当前事务版本的数据行,而ZERIALIZABLE(串行化)则会对所有读取的行都加锁,MVCC就没有意义了。

\paragraph{Duration持久性}

事务的持久性指的是只要事务成功结束,它对数据库所做的更新就必须永久保存下来,即使发生系统崩溃,重新启动数据库系统后,数据库还能恢复到事务成功结束时的状态。

持久性意味着事务操作最终要持久化到数据库中,持久性是由 内存+redo log来保证的,MySQL的InnoDB在修改数据的时候,同时在内存和redo log记录这次操作,宕机的时候可以从redo log中恢复数据。

同时,我们都知道MySQL Server的主从同步就是通过binlog来实现的,从服务器通过binlog文件的SQL拿过去执行一遍,保证跟主服务器的数据一致,而binlog和redo log都存储了表中的数据,都可以用来做数据恢复的,那怎么保证binlog和redo log的数据一致呢?下面是InnoDB下redo log的过程:

对redo log进行写盘,写完后事务进入prepare状态。
如果前面prepare成功,马上就会进行binlog写盘,再继续将事务日志持久化到binlog。
如果binlog持久化成功,那么事务则进入commit状态(在redo log里面写一条commit记录)。
这意味着一个事务到底有没有成功,由两方面来保证:第一是redo log里面有没有commit记录,如果有commit记录,那么binlog一定是持久化成功了,也就是说事务成功了。

再者就是redo log最终还会进行刷盘,它的刷盘会在系统空闲时进行,并不是写到redo log时马上进行刷盘。

以上就是数据库的基本特性ACID在MySQL中如何进行保证的方法,ACID就是这样通过InnoDB的几个日志和MVCC来保证原子性、一致性、隔离性、持久化的。

\end{document}