\documentclass[../../../interview-questions.tex]{subfiles}

\begin{document}

\subsection{\color{red}{数据库事务的隔离级别(Database Transaction Isolation Levels)}}

Spring 对 JDBC 的事务隔离级别进行了补充和扩展,并提出了 7 种事务传播行为。InnoDB offers all four transaction isolation levels described by the SQL:1992 standard: READ UNCOMMITTED, READ COMMITTED, REPEATABLE READ, and SERIALIZABLE. The default isolation level for InnoDB is REPEATABLE READ\footnote{参见:\url{https://dev.mysql.com/doc/refman/8.0/en/innodb-transaction-isolation-levels.html}}.

\begin{enumerate}
\item {read uncommitted(读未提交)B事务可以读取到A事务未提交的修改。有脏读(Dirty Read)的问题}脏读又称无效数据的读出,是指在数据库访问中,事务T1将某一值修改,然后事务T2读取该值,此后T1因为某种原因撤销对该值的修改,这就导致了T2所读取到的数据是无效的,值得注意的是,脏读一般是针对于update操作的。
\item{read committed (读已提交)B事务不能读取其他事务未commit的修改。有不可重复读(前后多次读取,数据内容不一致),幻读的问题。}会话A开启一个事务,查询id=1的结果,此时查询的结果name为武汉市。接着会话B把id=1的name修改为温州市(隐式事务Implicit Transaction\footnote{那些语句会导致隐式事务:\url{https://dev.mysql.com/doc/refman/8.0/en/implicit-commit.html}},因为此时的autocommit为1,每条SQL语句执行完自动提交),此时会话A的事务再一次查询id=1的结果,读取的结果name为温州市。会话B再此修改id=1的name为杭州市,会话A的事务再次查询id=1,结果name的值为杭州市,这种现象就是不可重复读(Unrepeated Read)。在当前事务中输入某些语句之后, 会造成当前事务悄悄的提交,那么这种导致事务悄悄提交的情况, 我们称为隐式提交(Implicit Commit)。
\item{repeated read (可重复读)有效地防止了脏读和不可重复读,事务第一次读和第二次读,结果不受其他事务影响,结果是一样的。有幻读(Phantom Read)的问题。事务A在执行读取操作,需要两次统计数据的总量,前一次查询数据总量后,此时事务B执行了新增数据的操作并提交后,这个时候事务A读取的数据总量和之前统计的不一样,就像产生了幻觉一样,平白无故的多了几条数据,成为幻读(前后多次读取,数据总量不一致)。不可重复读与幻读都是读到其他事务已提交的数据,但是它们针对点不同.不可重复读是针对update.幻读是针对delete,insert.}
\item{Serializeble (串行化)串行化,就是顺序执行事务。这个隔离界别要慎用,因为及其影响数据库吞吐量,未能使用到CPU的高性能。}幻读(Phantom)
一个事务先根据某些条件查询出一些记录,之后另一个事务又向表中插入了符合这些条件的记录,原先的事务再次按照该条件查询时,能把另一个事务插入的记录也读出来。(幻读在读未提交、读已提交、可重复读隔离级别都可能会出现)会话A开启一个事务,查询id>0的记录,此时会查到name=武汉市的记录。接着会话B插入一条name=温州市的数据(隐式事务,因为此时的autocommit为1,每条SQL语句执行完自动提交),这时会话A的事务再以刚才的查询条件(id>0)再一次查询,此时会出现两条记录(name为武汉市和温州市的记录),这种现象就是幻读(Phantom Read)。
\end{enumerate}

Spring的5大事务隔离级别比数据库多一个。

\end{document}


