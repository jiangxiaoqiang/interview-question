\documentclass[../../../interview-questions.tex]{subfiles}

\begin{document}

\subsection{InnoDB磁盘架构}

从 MySQL 官方文档中,我们可知 InnoDB 在物理层面上可划分为如下 7 个模块:

\paragraph{系统表空间(System Tablespace)}

The System Tablespace 是Doublewrite Buffer和Change buffer的储存区域,也有用户创建的表和索引数据。该空间的数据文件通过参数innodb\_data\_file\_path控制,默认值是ibdata1:12M:autoextend(文件名为ibdata1,大小略大于12MB,自动扩展)。8.0之后InnoDB将元数据(以前的.frm文件,存表结构)存在该区域的数据字典中(data dictionary)。

\paragraph{双写缓存文件(Doublewrite Buffer Files)}

Doublewrite Buffer位于The System Tablespace。在BP的页数据刷到磁盘真正的位置前,会先将数据存在doublewrite buffer。 这步操作是直接将数据作为顺序块,调用OS的fsync()方法写入到doublewrite buffer。

虽然数据写了两次,但是性能还是比两次IO低的。
此外fsync保证了BP中的数据写到磁盘中,即使数据库挂了,还可以从doublewrite buffer中还原数据。还可以解决页断裂的问题

\paragraph{Undo 表空间(Undo Tablespaces)}

Undo Tablespaces保存的是undo log ,用于回滚事务。
该表空间有rollback segments,rollback segments是用于存 undo log segments, 而undo log segments存的就是undo logs。

mysql启动的时候,默认初始两个undo tablespace。因为sql执行前必须要有rollback segments。而两个undo tablespace才支持automated truncation of undo。

\paragraph{Redo Log 文件(Redo Log)}

\paragraph{独占表空间(File-Per-Table Tablespaces)}

File-Per-Table Tablespaces 默认开启,为每个表都独立建一个.ibd文件。 通过参数innodb\_file\_per\_tabl 可以设置关闭,这样的话所有表数据是都存在The System Tablespace的ibdata。

\paragraph{通用表空间(General Tablespaces)}

General Tablespaces 是通过CREATE TABLESPACE创建的共享表空间。


\paragraph{临时表空间(Temporary Tablespaces)}

InnoDB把 Temporary Tablespaces分为两种,session temporary tablespaces 和global temporary tablespace。
session temporary tablespaces存储的是用户创建的临时表和内部的临时表,一个session最多有两个表空间(用户临时表和内部临时表)。
global temporary tablespace储存用户临时表的回滚段(rollback segments )。


\end{document}