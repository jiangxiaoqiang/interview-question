\documentclass[../../../interview-questions.tex]{subfiles}

\begin{document}

\subsection{MySQL优化思路}

MySQL优化可以从连接配置优化、存储引擎与表结构、SQL分析与优化、架构优化等方面入手。

\paragraph{连接配置优化}

服务端增加可用连接数,修改环境变量max\_connections,默认情况下服务端的最大连接数为151个。及时释放不活动的连接,系统默认的客户端超时时间是28800秒(8小时),我们可以把这个值调小一点。

客户端常见的数据库连接池有DBCP、C3P0、阿里的Druid、Hikari。对于每一个连接,服务端会创建一个单独的线程去处理,连接数越多,服务端创建的线程自然也就越多。而线程数超过CPU个数的情况下,CPU势必要通过分配时间片的方式进行线程的上下文切换,频繁的上下文切换会造成很大的性能开销。Hikari官方给出了一个PostgreSQL数据库连接池大小的建议值公式,CPU核心数*2+1。假设服务器的CPU核心数是4,把连接池设置成9就可以了。这种公式在一定程度上对其他数据库也是适用的。

\paragraph{存储引擎与表结构}

选择存储引擎。优化字段。

\paragraph{SQL分析与优化}

慢查询。EXPLAIN执行计划。SQL与索引优化。

\paragraph{架构优化}

使用缓存。读写分离。分库分表。消息队列削峰。

\paragraph{业务优化}

\end{document}