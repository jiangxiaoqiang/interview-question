\documentclass[../../../interview-questions.tex]{subfiles}

\begin{document}

\subsection{MySQL索引条件下推}

“索引条件下推”,称为 Index Condition Pushdown (ICP),这是MySQL提供的用某一个索引对一个特定的表从表中获取元组”,注意我们这里特意强调了“一个”,这是因为这样的索引优化不是用于多表连接而是用于单表扫描,确切地说,是单表利用索引进行扫描以获取数据的一种方式。


1、根据联合索引查出所有名字是’陈艮’的二级索引数据,然后回表,到主键索引上查询全部符合条件的数据(19 条数据)。然后返回给 Server 层,在 Server 层过滤出手机号码后四位为4087这个人。

2、根据联合索引查出所有名字是’陈艮’的二级索引数据(19 个索引),然后从二级索引 中筛选出手机号码后四位为4087的索引(1 个索引),然后再回表,到主键索引上查询全部符合条件的数据(1 条数据),返回给 Server 层。

很明显,第二种方式到主键索引上查询的数据更少。

注意,索引的比较是在存储引擎进行的,数据记录的比较,是在 Server 层进行的。 而当 phone 的条件不能用于索引过滤时,Server 层不会把 phone 的条件传递 给存储引擎,所以读取了两条没有必要的记录。

这时候,如果满足 name=’陈艮’的记录有 100000 条,就会有 99999 条没有 必要读取的记录。

https://princeli.com/mysql探秘三mysql索引原理


\end{document}