
\documentclass[../../../interview-questions.tex]{subfiles}

\begin{document}

\subsection{InnoDB逻辑存储结构}

InnoDB将所有数据都存放在表空间中,表空间又由段(segment)、区(extent)、页(page)组成。https://juejin.cn/post/6968264298208428046

\paragraph{表空间}

表空间可以看做是InnoDB存储引擎逻辑结构的最高层,所有的数据都存放在表空间中。在默认情况下InnoDB存储引擎有一个共享表空间ibdata1,所有数据都存放在这个表空间内。
如果启用了参数innodb\_file\_per\_table,则每张表内的数据可以单独放到一个表空间内。需要注意的是,这些单独的表空间文件仅存储该表的数据、索引和插入缓冲Bitmap等信息,其余信息还是存放在共享表空间中,例如 undo日志、插入缓冲索引页、系统事务信息、二次写缓冲等。
因此即使在启用了参数innodb\_file\_per\_table之后,共享表空间的大小还是会不断地增加,例如事务中写入了undo日志,就算回滚了,共享表空间的大小也不会缩小。但是会判断这些undo信息是否还需要,不需要的话,就会将这些空间标记为可用空间,供下次重复使用。

\paragraph{段}

从前面B+树的结构知道,B+树分为叶子节点和非叶子节点,最底层的叶子节点才存储了数据,非叶子节点是索引目录。如果将叶子节点页和非叶子节点页混合在一起存储,那在检索数据的时候同样也会有大量的随机I/O。
所以 InnoDB 又提出了段的概念,常见的段有数据段、索引段、回滚段等。段是一个逻辑上的概念,并不对应表空间中某一个连续的物理区域,它由若干个完整的区组成(还会包含一些碎片页),不同的段不能使用同一个区。
存放叶子节点的区的集合就是数据段,存放非叶子节点的区的集合就是索引段。也就是说一个索引会生成2个段,一个叶子节点段(数据段),一个非叶子节点段(索引段)。

\paragraph{区}

在默认情况下,InnoDB存储引擎页的大小为16KB,表空间中的页就太多了。为了更好的管理这些页,InnoDB 将物理位置上连续的64个页划为一个区,任何情况下,每个区的大小都为1MB。
B+树中每一层都是通过双向链表连接起来的,如果是以页为单位来分配存储空间,本来链表中相邻的两个页之间的物理位置就可能离得非常远,那么磁盘查询时就会有大量的随机I/O,随机I/O是非常慢的。所以应该尽量让链表中相邻的页的物理位置也相邻,这样可以消除很多的随机I/O,使用顺序I/O,尤其是在进行范围查询的时候。
所以在表中数据量大的时候,为某个索引分配空间的时候就不再按照页为单位分配了,而是按照区为单位分配,甚至在表中的数据非常多的时候,可以一次性分配多个连续的区。
不论是系统表空间还是独立表空间,都可以看成是由若干个区组成的,每个区64个页,然后每256个区又被划分成一组。

\paragraph{页}

页(Page)是 InnoDB 磁盘管理的最小单位,默认每个页的大小为16KB,也就是最多能保证16KB的连续存储空间。
InnoDB 将数据划分为若干个页,以页作为磁盘和内存之间交互的基本单位,也就是一次最少从磁盘中读取一页16KB的内容到内存中,一次最少把内存中的16KB内容刷新到磁盘中。

\end{document}


