\documentclass[../../../interview-questions.tex]{subfiles}

\begin{document}

\subsection{MySQL分表策略}

\paragraph{range}

RANGE方法优点: 扩容简单,提前建好库、表就好

RANGE方法缺点: 大部分读和写都访会问新的数据,有IO瓶颈,这样子造成新库压力过大,不建议采用。

\paragraph{hash}

HASH取模方法优点: 能保证数据较均匀的分散落在不同的库、表中,减轻了数据库压力

HASH取模方法缺点: 扩容麻烦、迁移数据时每次都需要重新计算hash值分配到不同的库和表

\paragraph{一致性Hash算法}

按照常用的hash算法来将对应的key哈希到一个具有2\^32次方个节点的空间中,即0~ (2\^32)-1的数字空间中。现在我们可以将这些数字头尾相连,想象成一个闭合的环形,这个圆环首尾相连,那么假设现在有三个数据库服务器节点node1、node2、node3三个节点,每个节点负责自己这部分的用户数据存储,假设有用户user1、user2、user3,我们可以对服务器节点进行HASH运算,假设HASH计算后,user1落在node1上,user2落在node2上,user3落在user3上。

\end{document}


