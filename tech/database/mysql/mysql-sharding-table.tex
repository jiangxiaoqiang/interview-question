\documentclass[../../../interview-questions.tex]{subfiles}

\begin{document}

\subsection{MySQL分表策略}

\paragraph{range}

RANGE方法优点: 扩容简单,提前建好库、表就好

RANGE方法缺点: 大部分读和写都访会问新的数据,有IO瓶颈,这样子造成新库压力过大,不建议采用。

\paragraph{hash}

HASH取模方法优点: 能保证数据较均匀的分散落在不同的库、表中,减轻了数据库压力

HASH取模方法缺点: 扩容麻烦、迁移数据时每次都需要重新计算hash值分配到不同的库和表

\paragraph{一致性Hash算法}

一致性哈希算法在 1997 年由麻省理工学院提出,是一种特殊的哈希算法,在移除或者添加一个服务器时,能够尽可能小地改变已存在的服务请求与处理请求服务器之间的映射关系。一致性哈希解决了简单哈希算法在分布式哈希表(Distributed Hash Table,DHT)中存在的动态伸缩等问题 。按照常用的hash算法来将对应的key哈希到一个具有$2^{32}$次方个节点的空间中,即0至$ 2^{32}-1$的数字空间中。现在我们可以将这些数字头尾相连,想象成一个闭合的环形,这个圆环首尾相连,那么假设现在有三个数据库服务器节点node1、node2、node3三个节点,每个节点负责自己这部分的用户数据存储,假设有用户user1、user2、user3,我们可以对服务器节点进行HASH运算,假设HASH计算后,user1落在node1上,user2落在node2上,user3落在user3上。一致性Hash算法也会存在数据倾斜问题。解决数据倾斜问题要使用虚拟节点策略。即,将每个数据库都通过定位算法生成几个在 Hash 环上的位置,每个位置都承担上面节点的功能,区别在于原来每个数据库对应一个节点,现在每个数据库会对应若干个节点,这就是虚拟节点。

\end{document}


