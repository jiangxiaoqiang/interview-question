\documentclass[../../../interview-questions.tex]{subfiles}

\begin{document}

\subsection{MySQL GTID}

全局事务 ID (Global Transaction ID, GTID) 是用来强化数据库在主备复制场景下,可以有效保证主备一致性、提高故障恢复、容错能力。GTID是MySQL 5.6的新特性,其全称是Global Transaction Identifier,可简化MySQL的主从切换以及Failover。GTID用于在binlog中唯一标识一个事务。当事务提交时,MySQL Server在写binlog的时候,会先写一个特殊的Binlog Event,类型为GTID\_Event,指定下一个事务的GTID,然后再写事务的Binlog。主从同步时GTID\_Event和事务的Binlog都会传递到从库,从库在执行的时候也是用同样的GTID写binlog,这样主从同步以后,就可通过GTID确定从库同步到的位置了。也就是说,无论是级联情况,还是一主多从情况,都可以通过GTID自动找点儿,而无需像之前那样通过File\_name和File\_position找点儿了。支持多线程复制(多数据库同时复制才有意义,仅仅复制一个则没有意义),通常,由于读取较大,主负责数据的写入,从负责的读取,可以有多从。
https://segmentfault.com/a/1190000041399202

\paragraph{GTID相较与传统复制的优势}

\begin{itemize}
    \item {主从搭建更加简便,不用手动特地指定position位置。}
    \item {复制集群内有一个统一的标识,识别、管理上更方便。}
    \item {故障转移更容易,不用像传统复制那样需要找 log\_file 和 log\_Pos的位置。}
    \item {通常情况下GTID是连续没有空洞的,更能保证数据的一致性,零丢失。}
    \item {相对于ROW复制模式,数据安全性更高,切换更简单。}
    \item {比传统的复制更加安全,一个GTID在一个MySQL实例上只会执行一次,避免重复执行导致数据混乱或者主从不一致。}
\end{itemize}


\end{document}