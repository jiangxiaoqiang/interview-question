\documentclass[../../../interview-questions.tex]{subfiles}

\begin{document}

\subsection{MySQL语句执行过程}

一个更新操作的流程,这是一个简化的过程(name原值是zhangsan)。

\begin{lstlisting}[language=SQL]
begin;update user set name='penyuyan' where id=1;commit;
\end{lstlisting}

\begin{enumerate}
    \item {事务开始,从内存或磁盘取到包含这条数据的页,返回给Server 的执行器;}
    \item {执行器在缓存页中修改这一行数据的值为penyuyan;}
    \item {记录 name=zhangsan 到 undo log;注意这里是记录到undo log buffer而不是实际的undo log磁盘上的文件}
    \item {记录 name=zhangsan 到redo log;}
    \item {调用存储引擎接口,在内存(Buffer Pool)中修改 name=penyuyan}
    \item {事务提交}
    \item {刷新 redo log buffer 到磁盘,并标记该事务的状态为 prepare。此操作称为 redo log prepare}
    \item {刷新 binlog cache 到磁盘}
    \item {刷新 redo log buffer 到磁盘,并标记该事务的状态为 commit。此操作称为 redo log commit。}刷新log到磁盘的代码在storage/innobase/handler/ha\_innodb.cc文件的方法:
    
\begin{lstlisting}[language=C]
/* Flush the redo log buffer to the redo log file.
Sync it to disc if we are in FLUSH LOGS, or if
innodb_flush_log_at_trx_commit=1
(write and sync at each commit). */
log_buffer_flush_to_disk(!binlog_group_flush ||
                            srv_flush_log_at_trx_commit == 1);
\end{lstlisting}


    \item {向客户端返回事务执行的结果。}
\end{enumerate}

这里些redo log和undo log也是写入到Buffer里,然后再定时刷到磁盘上的。通过innodb\_flush\_log\_at\_trx\_commit参数进行控制。推荐做法是\url{innodb\_flush\_log\_at\_trx\_commit=2},sync\_binlog=500 或1000。redo log 先 prepare,再刷新 binlog ,再 redo log commit 的过程就是一次两段式提交。这种只在 MySQL 内部组件间保障数据一致性的操作,也被称作内部 XA 事务;与之对应的是,保障跨服务器间数据一致性的两段式提交,被称为外部 XA 事务,即分布式事务。注:XA 事务属于分布式事务中两段式提交事务的一种实现在宕机后,重启 MySQL 时,InnoDB 会自动恢复 redo log 中 checkpoint\_lsn 后的,且处于 commit 状态的事务。如果 redo log 中事务的状态为 prepare,则需要先查看 binlog 中该事务是否存在,是的话就恢复,否则就回滚(通过 undo log 回滚。脏页一直在刷,更新了脏页,但事务没提交就宕机了,所以需要回滚)。

MySQL 宕机可能会发生在整个过程中的任意时刻。以刚才的流程为例,假设宕机发生在第 5 步后、第 6 步前。此时服务器还未向客户端返回事务的结果,而 redo log 中可能记录了该事务的 redo log,也可能没有。但是只要该事务没有被标记为 prepare,我们就认为该事务没有执行完,否则 redo log 用于恢复事务的数据可能是不完整的。因此,只要此时我们选择抛弃未 prepare 的 redo log,不会导致任何数据一致性的问题。

\end{document}