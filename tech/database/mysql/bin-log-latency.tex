\documentclass[../../../interview-questions.tex]{subfiles}

\begin{document}

\subsection{binlog延迟如何处理}

从硬件瓶颈和拆分业务2个方面进行考虑。

\begin{enumerate}
    \item {从库的压力大}
    \item {大事务的执行}
    \item {主库的DDL(alter、drop、create)}
    \item {锁冲突}锁冲突问题也可能导致从节点的SQL线程执行慢,比如从机上有一些select .... for update的SQL,或者使用了MyISAM引擎等。
    \item {从库的复制能力}一般场景中,因偶然情况导致从库延迟了几分钟,都会在从库恢复之后追上主库。但若是从库执行速度低于主库,且主库持续具有压力,就会导致长时间主从延迟,很有可能就是从库复制能力的问题。从库上的执行,即sql\_thread更新逻辑,在5.6版本之前,是只支持单线程,那么在主库并发高、TPS高时,就会出现较大的主从延迟。因此,MySQL自5.7版本后就已经支持并行复制了。可以在从服务上设置 slave\_parallel\_workers为一个大于0的数,然后把slave\_parallel\_type参数设置为LOGICAL\_CLOCK。
\end{enumerate}


https://juejin.cn/post/6844903824952393742

如何优化延迟呢?

\begin{itemize}
    \item {降低多线程大事务并发的概率,优化业务逻辑}
    \item {优化SQL,避免慢SQL,减少批量操作,建议写脚本以update-sleep这样的形式完成。}
    \item {提高从库机器的配置,减少主库写binlog和从库读binlog的效率差。}
    \item {尽量采用短的链路,也就是主库和从库服务器的距离尽量要短,提升端口带宽,减少binlog传输的网络延时。}
    \item {实时性要求的业务读强制走主库,从库只做灾备,备份。}
\end{itemize}

https://www.51cto.com/article/661356.html

\end{document}