\documentclass[../../../interview-questions.tex]{subfiles}

\begin{document}

\subsection{binlog延迟如何处理}

从硬件瓶颈和拆分业务2个方面进行考虑。

\begin{enumerate}
    \item {从库的压力大}
    \item {大事务的执行}
    \item {主库的DDL(alter、drop、create)}
    \item {锁冲突}
    \item {从库的复制能力}一般场景中,因偶然情况导致从库延迟了几分钟,都会在从库恢复之后追上主库。但若是从库执行速度低于主库,且主库持续具有压力,就会导致长时间主从延迟,很有可能就是从库复制能力的问题。从库上的执行,即sql\_thread更新逻辑,在5.6版本之前,是只支持单线程,那么在主库并发高、TPS高时,就会出现较大的主从延迟。因此,MySQL自5.7版本后就已经支持并行复制了。可以在从服务上设置 slave\_parallel\_workers为一个大于0的数,然后把slave\_parallel\_type参数设置为LOGICAL\_CLOCK。
\end{enumerate}


https://juejin.cn/post/6844903824952393742

\end{document}