\documentclass[../../../interview-questions.tex]{subfiles}

\begin{document}

\subsection{MySQL Redo Log}

InnoDB 有两块非常重要的日志,一个是undo log,另外一个是redo log,前者用来保证事务的原子性以及InnoDB的MVCC,后者用来保证事务的持久性。

http://mysql.taobao.org/monthly/2015/05/01/


\paragraph{生成Redo log}

InnoDB的redo log都是通过mtr产生的,先写到mtr的cache中,然后再提交到公共buffer中。Mini transaction(简称mtr)是InnoDB对物理数据文件操作的最小事务单元,用于管理对Page加锁、修改、释放、以及日志提交到公共buffer等工作。一个mtr操作必须是原子的,一个事务可以包含多个mtr。每个mtr完成后需要将本地产生的日志拷贝到公共缓冲区,将修改的脏页放到flush list上。生成Redo Log的入口函数在storage/innobase/row/row0ins.cc类的\url{row\_ins\_clust\_index\_entry\_low}方法中。

\begin{lstlisting}[language=C]
dberr_t row_ins_clust_index_entry_low(uint32_t flags, ulint mode,
    dict_index_t *index, ulint n_uniq,
    dtuple_t *entry, que_thr_t *thr,
    bool dup_chk_only) {

}
\end{lstlisting}



\paragraph{Redo log写入Buffer}

\paragraph{Redo log写入磁盘}

有几种场景可能会触发redo log写文件:

\begin{enumerate}
    \item {Redo log buffer空间不足时}
    \item {事务提交}
    \item {后台线程}
    \item {做checkpoint}
    \item {实例shutdown时}
    \item {binlog切换时}
\end{enumerate}

我们所熟悉的参数innodb\_flush\_log\_at\_trx\_commit 作用于事务提交时,这也是最常见的场景:

\begin{itemize}
    \item {当设置该值为1时,每次事务提交都要做一次fsync,这是最安全的配置,即使宕机也不会丢失事务;}
    \item {当设置为2时,则在事务提交时只做write操作,只保证写到系统的page cache,因此实例crash不会丢失事务,但宕机则可能丢失事务;}
    \item {当设置为0时,事务提交不会触发redo写操作,而是留给后台线程每秒一次的刷盘操作,因此实例crash将最多丢失1秒钟内的事务。}
\end{itemize}

对于刷盘操作,有3种类型。

\begin{itemize}
    \item {BUF\_FLUSH\_LRU}从 buffer pool 的 LRU 上扫描并刷新。
    \item {BUF\_FLUSH\_LIST}从 buffer pool 的 FLUSH LIST 上扫描并刷新。
    \item {BUF\_FLUSH\_SINGLE\_PAGE}从 LRU 上只刷新一个 page,会通过 \url{buf\_dblwr\_write\_single\_page()} 来写一个 page 。
\end{itemize}

前两种属于 BATCH FLUSH,最后一种属于 SINGLE FLUSH,在 \url{buf\_flush\_write\_block\_low()} 函数中执行如下逻辑。Single也就是在 \url{buf\_dblwr\_write\_single\_page()} 函数中,用于将一个 page 加入到 double write buffer 中,并完成写操作。需要注意的是MySQL 8移除了以上2个方法。


\end{document}