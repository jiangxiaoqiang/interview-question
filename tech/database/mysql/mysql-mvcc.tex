\documentclass[../../../interview-questions.tex]{subfiles}

\begin{document}

\subsection{MySQL MVCC实现}

MVCC(Multi-Version Concurrency Control)在MySQL InnoDB中的实现主要是为了提高数据库并发性能,用更好的方式去处理读-写冲突,做到即使有读写冲突时,也能做到不加锁,非阻塞并发读。MVCC实现的要点是隐藏列、Undo Log和Read View。隐藏列,一个保存了行的创建时间,一个保存行的过期时间(或删除时间)。当然存储的并不是实际的时间值,而是系统版本号(system version number)。每开始一个新的事务,系统版本号都会自动递增。事务开始时刻的系统版本号会作为事务的版本号,用来和查询到的每行记录的版本号进行比较。

\paragraph{隐藏列}

\paragraph{Undo Log}

\paragraph{Read View}

什么是Read View,说白了Read View就是事务进行快照读(select * from)操作的时候生产的读视图(Read View),在该事务执行的快照读的那一刻,会生成事务系统当前的一个快照,记录并维护系统当前活跃事务(未提交事务)的ID(当每个事务开启时,都会被分配一个ID, 这个ID是递增的,所以最新的事务,ID值越大)。

所以我们知道 Read View主要是用来做可见性判断的, 即当我们某个事务执行快照读的时候,对该记录创建一个Read View读视图,把它比作条件用来判断当前事务能够看到哪个版本的数据,既可能是当前最新的数据,也有可能是该行记录的undo log里面的某个版本的数据。可见性判断在文件storage/innobase/include/read0types.h中,如下代码片段:

\begin{lstlisting}[language=C]
/** Check whether the changes by id are visible.
@param[in]    id      transaction id to check against the view
@param[in]    name    table name
@return whether the view sees the modifications of id. */
[[nodiscard]] bool changes_visible(trx_id_t id,
                                    const table_name_t &name) const {
    ut_ad(id > 0);

    if (id < m_up_limit_id || id == m_creator_trx_id) {
        return (true);
    }

    check_trx_id_sanity(id, name);

    if (id >= m_low_limit_id) {
        return (false);

    } else if (m_ids.empty()) {
        return (true);
    }

    const ids_t::value_type *p = m_ids.data();

    return (!std::binary_search(p, p + m_ids.size(), id));
}
\end{lstlisting}

\begin{itemize}
    \item {首先比较DB\_TRX\_ID < up\_limit\_id, 如果小于,则当前事务能看到DB\_TRX\_ID 所在的记录,如果大于等于进入下一个判断}
    \item {接下来判断 DB\_TRX\_ID 大于等于 low\_limit\_id , 如果大于等于则代表DB\_TRX\_ID 所在的记录在Read View生成后才出现的,那对当前事务肯定不可见,如果小于则进入下一个判断 }
    \item {判断DB\_TRX\_ID 是否在活跃事务之中,trx\_list.contains(DB\_TRX\_ID),如果在,则代表我Read View生成时刻,你这个事务还在活跃,还没有Commit,你修改的数据,我当前事务也是看不见的;如果不在,则说明,你这个事务在Read View生成之前就已经Commit了,你修改的结果,我当前事务是能看见的}
\end{itemize}

\end{document}