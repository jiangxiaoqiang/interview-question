\documentclass[../../../interview-questions.tex]{subfiles}

\begin{document}

\subsection{MySQL主从复制过程}

MySQL 主从复制有三种方式:基于SQL语句的复制(statement-based replication,SBR),基于行的复制(row-based replication,RBR),混合模式复制(mixed-based replication,MBR)。对应的bin-log文件的格式也有三种:STATEMENT,  ROW,  MIXED。

\paragraph{Binlog复制}

初次复制的时候要经历slave和master的认证,slave将binlog的文件名以及pos点告知master。
https://www.51cto.com/article/612454.htmlMySQL复制过程分成三步:

\begin{itemize}
    \item {master将改变记录到二进制日志(binary log)。这些记录过程叫做二进制日志事件,binary log events;}
    \item {slave将master的binary log events拷贝到它的中继日志(relay log);}
    \item {slave重做中继日志中的事件,将改变应用到自己的数据库中。 MySQL复制是异步的且串行化的。}
\end{itemize}

\paragraph{GTID复制}

在传统的复制里面,当发生故障,需要主从切换,需要找到bin-log和pos点(指从库更新到了主库bin-log的哪个位置,这个位置之前都已经更显完毕,这个位置之后未更新),然后将主节点指向新的主节点,相对来说比较麻烦,也容易出错。在MySQL 5.6里面,不用再找bin-log和pos点,我们只需要知道主节点的ip,端口,以及账号密码就行,因为复制是自动的,MySQL会通过内部机制GTID自动找点同步。基于GTID的复制是MySQL 5.6后新增的复制方式.GTID (global transaction identifier) 即全局事务ID, 保证了在每个在主库上提交的事务在集群中有一个唯一的ID.

\begin{itemize}
    \item {master节点在更新数据的时候,会在事务前产生GTID信息,一同记录到binlog日志中。}
    \item {slave节点的io线程将binlog写入到本地relay log中。}
    \item {然后SQL线程从relay log中读取GTID,设置gtid\_next的值为该gtid,然后对比slave端的binlog是否有记录。}
    \item {如果有记录的话,说明该GTID的事务已经运行,slave会忽略。}
    \item {如果没有记录的话,slave就会执行该GTID对应的事务,并记录到binlog中。}
\end{itemize}

https://blog.nowcoder.net/n/b90c959437734a8583fddeaa6d102e43

\end{document}