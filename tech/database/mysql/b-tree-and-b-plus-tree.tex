\documentclass[../../../interview-questions.tex]{subfiles}

\begin{document}

\subsection{B树和B+树是解决什么样的问题的,怎样演化过来,之间区别}

要更深入的明白B树和B+树的特点,首先得从二叉树(Binary Search Tree)说起,一棵高度为h的树上各操作的时间复杂度为O(h),而树的高度与树的平衡性有关,最好情况下对于一颗含n个节点的完全二叉树,操作的复杂度为O(lgn),但是在最坏情况下二叉树退化成线性链,操作复杂度为O(n)。由此可知,在实际使用时应尽量保证数据的随机性,以使树的高度不至于过大,否则会产生退化现象(这也就解释了为什么不适合在性别字段上建立MySQL数据库的索引,为什么索引要求数据尽量分散,不要趋同)。为什么会出现这样的问题?造成这种情况的主要原因就是BST(Binary Search Tree)不够平衡(左右子树高度差太大)。为了避免此问题,提出了平衡二叉树,AVL(Adelson-Velsky and Landis Tree)树是最早被发明的自平衡二叉查找树,AVL树得名于它的发明者G. M. Adelson-Velsky和Evgenii Landis,他们在1962年的论文《An algorithm for the organization of information》中公开了这一数据结构。但是AVL树有什么问题呢?\textbf{AVL树平衡耗时。}AVL树是严格的平衡二叉树,所有节点的左右子树高度差不能超过1;AVL树查找、插入和删除在平均和最坏情况下都是O(lgn)。AVL实现平衡的关键在于旋转操作:插入和删除可能破坏二叉树的平衡,此时需要通过一次或多次树旋转来重新平衡这个树。当插入数据时,最多只需要1次旋转(单旋转或双旋转);但是当删除数据时,会导致树失衡,AVL需要维护从被删除节点到根节点这条路径上所有节点的平衡,旋转的量级为O(lgn)。由于旋转的耗时,AVL树在删除数据时效率很低;在删除操作较多时,维护平衡所需的代价可能高于其带来的好处,因此AVL实际使用并不广泛。AVL树相比,红黑树并不追求严格的平衡,而是大致的平衡:只是确保从根到叶子的最长的可能路径不多于最短的可能路径的两倍长。从实现来看,红黑树最大的特点是每个节点都属于两种颜色(红色或黑色)之一,且节点颜色的划分需要满足特定的规则(具体规则略)。但是红黑树有什么问题呢?\textbf{树太高。}再提出怎样让每一层放的节点多一些来减少遍历高度(所以二叉树、AVL树都只放一个节点,戏称1叉树),引申出m叉树,m叉搜索树同样会有退化现象,引出m叉平衡树,也就是B树,这时候每个节点既放了key也放了value,怎样使每个节点放尽可能多的key值,以减少遍历高度呢(访问磁盘次数),可以将每个节点只放key值,将value值放在叶子结点,在叶子结点的value值增加指向相邻节点指针,这就是优化后的B+树。B+树是B树的一个升级版,相对于B树来说B+树更充分的利用了节点的空间,让查询速度更加稳定,其速度完全接近于二分法查找。由此,B+树与B树相比,有以下优势:

更少的IO次数:B+树的非叶节点只包含键,而不包含真实数据,因此每个节点存储的记录个数比B数多很多(即阶m更大),因此B+树的高度更低,访问时所需要的IO次数更少。此外,由于每个节点存储的记录数更多,所以对访问局部性原理的利用更好,缓存命中率更高。
更适于范围查询:在B树中进行范围查询时,首先找到要查找的下限,然后对B树进行中序遍历,直到找到查找的上限;而B+树的范围查询,只需要对链表进行遍历即可。

更稳定的查询效率:B树的查询时间复杂度在1到树高之间(分别对应记录在根节点和叶节点),而B+树的查询复杂度则稳定为树高,因为所有数据都在叶节点。

B+树也存在劣势:由于键会重复出现,因此会占用更多的空间。但是与带来的性能优势相比,空间劣势往往可以接受,因此B+树的在数据库中的使用比B树更加广泛。
前面说到,B树/B+树与红黑树等二叉树相比,最大的优势在于树高更小。实际上,对于Innodb的B+索引来说,树的高度一般在2-4层。下面来进行一些具体的估算。

树的高度是由阶数决定的,阶数越大树越矮;而阶数的大小又取决于每个节点可以存储多少条记录。InnoDB中每个节点使用一个页(page),页的大小为16KB,其中元数据只占大约128字节左右(包括文件管理头信息、页面头信息等等),大多数空间都用来存储数据。

对于非叶节点,记录只包含索引的键和指向下一层节点的指针。假设每个非叶节点页面存储1000条记录,则每条记录大约占用16字节;当索引是整型或较短的字符串时,这个假设是合理的。延伸一下,我们经常听到建议说索引列长度不应过大,原因就在这里:索引列太长,每个节点包含的记录数太少,会导致树太高,索引的效果会大打折扣,而且索引还会浪费更多的空间。
对于叶节点,记录包含了索引的键和值(值可能是行的主键、一行完整数据等,具体见前文),数据量更大。这里假设每个叶节点页面存储100条记录(实际上,当索引为聚簇索引时,这个数字可能不足100;当索引为辅助索引时,这个数字可能远大于100;可以根据实际情况进行估算)。
对于一颗3层B+树,第一层(根节点)有1个页面,可以存储1000条记录;第二层有1000个页面,可以存储1000*1000条记录;第三层(叶节点)有1000*1000个页面,每个页面可以存储100条记录,因此可以存储1000*1000*100条记录,即1亿条。而对于二叉树,存储1亿条记录则需要26层左右。

\end{document}