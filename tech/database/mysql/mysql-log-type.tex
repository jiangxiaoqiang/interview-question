\documentclass[../../../interview-questions.tex]{subfiles}

\begin{document}

\subsection{MySQL日志类型}

https://www.cnblogs.com/wy123/p/8365234.html。MySQL日志主要包括六种:

\begin{itemize}
    \item {重做日志(redo log)} 重做日志的作用是确保事务的持久性。防止在发生故障的时间点,尚有脏页未写入磁盘,在重启MySQL服务的时候,根据redo log进行重做,从而达到事务的持久性这一特性。事务开始之后就产生redo log,redo log的落盘并不是随着事务的提交才写入的,而是在事务的执行过程中,便开始写入redo log文件中。当对应事务的脏页写入到磁盘之后,redo log的使命也就完成了,重做日志占用的空间就可以重用(被覆盖)。
    \item {回滚日志(undo log)}undo log是实现原子性的关键。保存了事务发生之前的数据的一个版本,可以用于回滚,同时可以提供多版本并发控制下的读(MVCC),隔离性由 MVCC 来保证,也即非锁定读。逻辑格式的日志,在执行undo的时候,仅仅是将数据从逻辑上恢复至事务之前的状态,而不是从物理页面上操作实现的,这一点是不同于redo log的。
    \item {归档日志(binlog)}
    \item {错误日志(errorlog)}
    \item {慢查询日志(slow query log)}
    \item {一般查询日志(general log)}
    \item {中继日志(relay log)}relay-log中继日志是连接master和slave的核心。relay-log的结构和binlog非常相似,只不过他多了一个master.info和relay-log.info的文件。master.info记录了上一次读取到master同步过来的binlog的位置,以及连接master和启动复制必须的所有信息。relay-log.info记录了文件复制的进度,下一个事件从什么位置开始,由sql线程负责更新。从服务器 I/O 线程将主服务器的 Binlog 日志读取过来,解析到各类 Events 之后记录到从服务器本地文件,这个文件就被称为 relay log。然后 SQL 线程会读取 relay log 日志的内容并应用到从服务器,从而使从服务器和主服务器的数据保持一致。中继日志充当缓冲区,这样 master 就不必等待 slave 执行完成才发送下一个事件。
\end{itemize}


\end{document}