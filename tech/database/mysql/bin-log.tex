\documentclass[../../../interview-questions.tex]{subfiles}

\begin{document}

\subsection{binlog文件格式}

MySQL binlog的格式有三种,基于SQL语句的复制(statement-based replication, SBR),基于行的复制(row-based replication, RBR),
混合模式复制(mixed-based replication, MBR)。相应地,binlog的格式也有三种:STATEMENT,ROW,MIXED。它主要用于mysql的复制技术。
MySQL binlog是二进制日志文件,用于记录MySQL的数据更新或者潜在更新(比如DELETE语句执行删除而实际并没有符合条件的数据),在MySQL主从复制中就是依靠的binlog。可以通过语句“show binlog events in 'binlogfile'”来查看binlog的具体事件类型。binlog记录的所有操作实际上都有对应的事件类型的,MySQL binlog的三种工作模式:

(1)Row level(用到MySQL的特殊功能如存储过程、触发器、函数,又希望数据最大化一直则选择Row模式,我们公司选择的是row)
简介:日志中会记录每一行数据被修改的情况,然后在slave端对相同的数据进行修改。
优点:能清楚的记录每一行数据修改的细节
缺点:数据量太大

(2)Statement level(默认)
简介:每一条被修改数据的sql都会记录到master的bin-log中,slave在复制的时候sql进程会解析成和原来master端执行过的相同的sql再次执行。在主从同步中一般是不建议用statement模式的,因为会有些语句不支持,比如语句中包含UUID函数,以及LOAD DATA IN FILE语句等
优点:解决了 Row level下的缺点,不需要记录每一行的数据变化,减少bin-log日志量,节约磁盘IO,提高性能。缺点:容易出现主从复制不一致

(3)Mixed(混合模式)
简介:结合了Row level和Statement level的优点,同时binlog结构也更复杂。


binlog格式分为statement,row以及mixed三种,MySQL 5.5默认的还是statement模式,当然我们在主从同步中一般是不建议用statement模式的,因为会有些语句不支持,比如语句中包含UUID函数,以及LOAD DATA IN FILE语句等,一般推荐的是mixed格式。暂且不管这三种格式的区别,看看binlog的存储格式是什么样的。binlog是一个二进制文件集合,当然除了我们看到的mysql-bin.xxxxxx这些binlog文件外,还有个binlog索引文件mysql-bin.index。如官方文档中所写,binlog格式如下:

\begin{enumerate}
\item {binlog文件以一个值为0Xfe62696e的魔数开头,这个魔数对应0xfe 'b''i''n'。}
\item {binlog由一系列的binlog event构成。每个binlog event包含header和data两部分,event是binlog的最小组成单元。}
\item {header部分提供的是event的公共的类型信息,包括event的创建时间,服务器等等。}
\item {data部分提供的是针对该event的具体信息,如具体数据的修改。}
\item {每个binlog文件以一个Rotate类型的event结束(但也有特殊情况,当数据库出现宕机的情况,重新启动数据库会生成一个新的binlog文件,但是宕机之前的最新binlog文件中,不是以ROTATE\_EVENT结束的}
\end{enumerate}

\end{document}