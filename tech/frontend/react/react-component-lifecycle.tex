\documentclass[../../../interview-questions.tex]{subfiles}

\begin{document}

\subsection{React组件生命周期}

组件的生命周期可分成三个状态:

\begin{enumerate}
\item{Mounting(挂载):已插入真实 DOM}

当组件实例被创建并插入 DOM 中时,其生命周期调用顺序如下:


constructor(): 在 React 组件挂载之前,会调用它的构造函数。

getDerivedStateFromProps(): 在调用 render 方法之前调用,并且在初始挂载及后续更新时都会被调用。

render(): render() 方法是 class 组件中唯一必须实现的方法。

componentDidMount(): 在组件挂载后(插入 DOM 树中)立即调用。


\item{Updating(更新):正在被重新渲染}

每当组件的 state 或 props 发生变化时,组件就会更新。当组件的 props 或 state 发生变化时会触发更新。组件更新的生命周期调用顺序如下:

getDerivedStateFromProps(): 在调用 render 方法之前调用,并且在初始挂载及后续更新时都会被调用。根据 shouldComponentUpdate() 的返回值,判断 React 组件的输出是否受当前 state 或 props 更改的影响。

shouldComponentUpdate():当 props 或 state 发生变化时,shouldComponentUpdate() 会在渲染执行之前被调用。

render(): render() 方法是 class 组件中唯一必须实现的方法。

getSnapshotBeforeUpdate(): 在最近一次渲染输出(提交到 DOM 节点)之前调用。

componentDidUpdate(): 在更新后会被立即调用。

\item{Unmounting(卸载):已移出真实 DOM}

当组件从 DOM 中移除时会调用如下方法:

componentWillUnmount(): 在组件卸载及销毁之前直接调用。

\end{enumerate}

componentDidMount(): 在组件挂载后(插入 DOM 树中)立即调用。render() 方法是 class 组件中唯一必须实现的方法,其他方法可以根据自己的需要来实现。componentDidUpdate(): 在更新后会被立即调用。componentWillUnmount(): 在组件卸载及销毁之前直接调用。

https://projects.wojtekmaj.pl/react-lifecycle-methods-diagram/

\end{document}