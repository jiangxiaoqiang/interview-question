\documentclass[../../../interview-questions.tex]{subfiles}

\begin{document}

\subsection{函数组件和类组件的区别}

在 React 中,类组件就是基于ES6语法,通过继承 React.component 得到的组件\footnote{参见:\url{https://github.com/jappp/Blog/issues/12}}。函数组件也称无状态组件,顾名思义就是以函数形态存在的 React 组件\footnote{参见:\url{https://overreacted.io/zh-hans/how-are-function-components-different-from-classes/}}。

\begin{enumerate}
    \item{类组件有生命周期,函数组件没有,不过有了Hooks之后,函数组件也有了生命周期}
    \item{类组件需要继承 Class,函数组件不需要}
    \item{类组件可以获取实例化的 this,并且基于 this 做各种操作,函数组件不行}
    \item {类组件内部可以定义并维护 state, 函数组件都称为无状态了,那肯定不行。不过有了Hooks之后,函数组件也可以维护state} 
\end{enumerate}

看上去类组件的功能包含了方方面面,大而全,大而全一定好吗?肯定不尽其然,大家也能知道类组件的内部逻辑容易和组件黏在一起,难以拆分和复用;大而全代表学习成本高,容易写出垃圾代码。函数组件相比较类组件,优点是更轻量与灵活,便于逻辑的拆分复用。函数组件更符合 React 团队的设计理念,并且代码易于拆分和复用,用脚投票都知道 React 团队为什么要推出 Hooks 来扩展函数组件的功能,并且倡导大家使用函数组件了。

\end{document}