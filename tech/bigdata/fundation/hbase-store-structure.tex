\documentclass[../../../interview-questions.tex]{subfiles}

\begin{document}

\subsection{HBase底层存储结构}

HBase是Google的BigTable的开源实现,底层存储引擎是基于LSM-Tree数据结构设计的。LSM树(Log-Structured-Merge-Tree)的名字往往会给初识者一个错误的印象,事实上,LSM树并不像B+树、红黑树一样是一颗严格的树状数据结构,它其实是一种存储结构,目前HBase,LevelDB,RocksDB这些NoSQL存储都是采用的LSM树。LSM树的核心特点是利用顺序写来提高写性能,但因为分层(此处分层是指的分为内存和文件两部分)的设计会稍微降低读性能,但是通过牺牲小部分读性能换来高性能写,使得LSM树成为非常流行的存储结构。写入数据时会先写WAL日志,再将数据写到写缓存MemStore中,等写缓存达到一定规模后或满足其他触发条件才会flush刷写到磁盘,这样就将磁盘随机写变成了顺序写,提高了写性能。每一次刷写磁盘都会生成新的HFile文件。


\end{document}






