\documentclass[../../../interview-questions.tex]{subfiles}

\begin{document}

\subsection{谈谈你对MapReduce的理解}

MapReduce是一种计算模型,该模型可以将大型数据处理任务分解成很多单个的、可以在服务器集群中并行执行的任务,而这些任务的计算结果可以合并在一起来计算最终的结果。简而言之,Hadoop Mapreduce是一个易于编程并且能在大型集群(上千节点)快速地并行得处理大量数据的软件框架,以可靠,容错的方式部署在商用机器上。
MapReduce这个术语来自两个基本的数据转换操作:map过程和reduce过程。

映射(Mapping) :对集合里的每个目标应用同一个操作。即,如果你想把表单里每个单元格乘以二,那么把这个函数单独地应用在每个单元格上的操作就属于mapping(这里体现了移动计算而不是移动数据);map操作会将集合中的元素从一种形式转化成另一种形式,在这种情况下,输入的键值对会被转换成零到多个键值对输出。其中输入和输出的键必须完全不同,而输入和输出的值则可能完全不同。

化简(Reducing):遍历集合中的元素来返回一个综合的结果。即,输出表单里一列数字的和这个任务属于reducing。某个键的所有键值对都会被分发到同一个reduce操作中。确切的说,这个键和这个键所对应的所有值都会被传递给同一个Reducer。reduce
过程的目的是将值的集合转换成一个值(例如求和或者求平均),或者转换成另一个集合。这个Reducer最终会产生一个键值对。需要说明的是,如果job不需要reduce过程的话,那么reduce过程也是可以不用的。

\end{document}






