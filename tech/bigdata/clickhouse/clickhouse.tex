\documentclass[../../../interview-questions.tex]{subfiles}

\begin{document}

\subsection{Clickhouse}

Hadoop生态体系解决了大数据界的大部分问题,当然其也存在缺点。Hadoop体系的最大短板在于数据处理时效性。基于Hadoop生态的数据处理场景大部分对时效要求不高,按照传统的做法一般是 T + 1 的数据时效。即 Trade + 1,数据产出在交易日 + 1 天。ClickHouse的产生就是为了解决大数据量处理的时效性。根据官方提供的数据,性能表现大致如下:

\begin{itemize}
    \item {低延迟:对于数据量(几千行,列不是很多)不是很大的短查询,如果数据已经被载入缓存,且使用主码,延迟在50MS左右}
    \item {并发量:虽然ClickHouse是一种在线分析型数据库,也可支持一定的并发。当单个查询比较短时,官方建议100 Queries / second}
    \item {写入速度:在使用MergeTree引擎的情况下,写入速度大概是50 - 200M / s,如果按照1 K一条记录来算,大约每秒可写入50000 至 200000条记录每秒。如果每条记录比较小的话写入速度会更快}
\end{itemize}

\end{document}






