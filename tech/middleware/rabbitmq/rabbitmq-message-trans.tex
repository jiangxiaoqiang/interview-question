\documentclass[../../../interview-questions.tex]{subfiles}

\begin{document}

\subsection{RabbitMQ消息传播模式}

Simple Work Queue (简单工作队列):也就是常说的点对点模式\footnote{参见:\url{https://segmentfault.com/a/1190000040968626}},一条消息由一个消费者进行消费。(当有多个消费者时,默认使用轮训机制把消息分配给消费者)。

Work Queues (工作队列):也叫公平队列,能者多劳的消息队列模型。队列必须接收到来自消费者的手动ack才可以继续往消费者发送消息。

Publish/Subscribe (发布订阅模式):一条消息被多个消费者消费。

Routing(路由模式):有选择的接收消息。

Topics (主题模式):通过一定的规则来选择性的接收消息。

RPC 模式:发布者发布消息,并且通过 RPC 方式等待结果。目前这个应该场景少,而且代码也较为复杂,本章不做细讲。
注意:官网最后有 Publisher Confirms 为消息确认机制。指的是生产者如何发送可靠的消息。


\begin{table}[htbp]
	\caption{队列与Exchange类型对应关系}
	\label{table:queueexchange}
	\begin{center}
		\begin{tabular}{p{6cm}c}
			\hline
			\multirow{1}{*}{消息模式}
			& \multicolumn{1}{c}{交换机} \\			
			\cline{1-2}
			Simple Work Queue (简单工作队列),Work Queues (工作队列) &  空交换机   \\
			\hline
			Publish/Subscribe (发布订阅模式) &  fanout (扇形交换机)  \\
			\hline
			Routing(路由模式) &  direct (直连交换机) \\
			\hline	
            Topics(主题模式) &  topic(主题交换机) \\
			\hline						
		\end{tabular}	
	\end{center}
\end{table}

\end{document}