\documentclass[../../../interview-questions.tex]{subfiles}

\begin{document}

\subsection{Redis和Setnx命令是如何实现分布式锁的}

SETNX 是SET if Not eXists的简写。在进程释放锁,即执行 DEL lock.foo 操作前,需要先判断锁是否已超时。如果锁已超时,那么锁可能已由其他进程获得,这时直接执行 DEL lock.foo 操作会导致把其他进程已获得的锁释放掉。分布式锁的详细分析\footnote{\url{http://zhangtielei.com/posts/blog-redlock-reasoning.html}}。使用Redis分布式锁,要注意非原子操作、忘了释放锁、释放了别人的锁、大量失败请求、锁重入问题、锁竞争问题、读写锁、锁分段、锁超时问题、主从复制的问题。

\end{document}






