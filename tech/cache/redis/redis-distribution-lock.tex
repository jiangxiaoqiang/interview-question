\documentclass[../../../interview-questions.tex]{subfiles}

\begin{document}

\subsection{Redis和Setnx命令是如何实现分布式锁的}

SETNX 是SET if Not eXists的简写。在进程释放锁,即执行 DEL lock.foo 操作前,需要先判断锁是否已超时。如果锁已超时,那么锁可能已由其他进程获得,这时直接执行 DEL lock.foo 操作会导致把其他进程已获得的锁释放掉。分布式锁的详细分析\footnote{\url{http://zhangtielei.com/posts/blog-redlock-reasoning.html}}。使用Redis分布式锁,要注意非原子操作、忘了释放锁、释放了别人的锁、大量失败请求、锁重入问题、锁竞争问题、读写锁、锁分段、锁超时问题、主从复制的问题。setnx命令的命令执行器是setnxCommand,具体实现在t\_string.c中:

\begin{lstlisting}[language=C]
void setGenericCommand(redisClient *c, int flags, robj *key, robj *val, robj *expire, int unit, robj *ok_reply, robj *abort_reply) {

    long long milliseconds = 0; /* initialized to avoid any harmness warning */

    // 取出过期时间
    if (expire) {

        // 取出 expire 参数的值
        // T = O(N)
        if (getLongLongFromObjectOrReply(c, expire, &milliseconds, NULL) != REDIS_OK)
            return;

        // expire 参数的值不正确时报错
        if (milliseconds <= 0) {
            addReplyError(c,"invalid expire time in SETEX");
            return;
        }

        // 不论输入的过期时间是秒还是毫秒
        // Redis 实际都以毫秒的形式保存过期时间
        // 如果输入的过期时间为秒,那么将它转换为毫秒
        if (unit == UNIT_SECONDS) milliseconds *= 1000;
    }

    // 如果设置了 NX 或者 XX 参数,那么检查条件是否不符合这两个设置
    // 在条件不符合时报错,报错的内容由 abort reply 参数决定
    if ((flags & REDIS_SET_NX && lookupKeyWrite(c->db,key) != NULL) ||
        (flags & REDIS_SET_XX && lookupKeyWrite(c->db,key) == NULL))
    {
        addReply(c, abort_reply ? abort_reply : shared.nullbulk);
        return;
    }

    // 将键值关联到数据库
    setKey(c->db,key,val);

    // 将数据库设为脏
    server.dirty++;

    // 为键设置过期时间
    if (expire) setExpire(c->db,key,mstime()+milliseconds);

    // 发送事件通知
    notifyKeyspaceEvent(REDIS_NOTIFY_STRING,"set",key,c->db->id);

    // 发送事件通知
    if (expire) notifyKeyspaceEvent(REDIS_NOTIFY_GENERIC,
        "expire",key,c->db->id);

    // 设置成功,向客户端发送回复
    // 回复的内容由 ok reply 决定
    addReply(c, ok_reply ? ok_reply : shared.ok);
}
\end{lstlisting}




\end{document}






