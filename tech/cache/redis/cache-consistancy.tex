\documentclass[../../../interview-questions.tex]{subfiles}

\begin{document}

\subsection{缓存一致性(Cache Coherency)问题解决}

改善缓存一致性问题,可以先更新数据库,再删除缓存。这种也可能会存在缓存不一致问题,例如多个线程同时读取key时,key刚好过期。此时线程A查询数据库旧值,线程B更新数据库新值,并删除缓存。线程A将旧值放入。那么此时数据库和缓存就是不一致的数据。更新的时候通过事务来完成,且更新之前判断数据是否过期。
先更新数据库,再删除缓存。这里的一致性可以有最终一致性和强一致性。最终一致性可以用带重试逻辑的删除来处理,强一致性可以采用加分布式锁的方式,并且在查询逻辑中检测在查询时如果对应的数据已经有分布式锁,直接查询数据库,不走更新缓存的逻辑\footnote{参考来源:\url{https://www.infoq.cn/article/hh4iouiijhwb4x46vxeo}}。https://www.cnblogs.com/rjzheng/p/9041659.html
https://cloud.tencent.com/developer/article/1932934

\paragraph{改善一致性问题}

2000年7月,加州大学伯克利分校的Eric Brewer教授在ACM PODC会议上提出CAP猜想。2年后,麻省理工学院的Seth Gilbert和Nancy Lynch从理论上证明了CAP。之后,CAP理论(一致性Consistency, 可用性Availability, 分区容错性Partition Tolerance)正式成为分布式计算领域的公认定理。

\paragraph{分布式事务解决一致性问题}

http://fanyilun.me/2021/03/06/一致性问题与分布式事务/



\end{document}






