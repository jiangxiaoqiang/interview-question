\documentclass[../../../interview-questions.tex]{subfiles}

\begin{document}

\subsection{Redis Ziplist}

压缩列表(ziplist)是链表键和哈希键的底层实现之一。当链表键或哈希键只有少量列表项,且列表项中是小整数值或短字符串,则会采用压缩列表作为底层实现。在Redis中,有五种基本数据类型,除了提到的String,还有list,hash,zset,set,其中list,hash,zset都间接或者直接使用了ziplist,所以说理解ziplist也是相当重要的。从上述的描述中,知道ziplist是一个经过特殊编码的双向链表,它的设计目标就是为了提高存储效率。ziplist可以用于存储字符串或整数,其中整数是按真正的二进制表示进行编码的,而不是编码成字符串序列。它能以O(1)的时间复杂度在表的两端提供push和pop操作。

一个普通的双向链表,链表中每一项都占用独立的一块内存,各项之间用地址指针(或引用)连接起来。这种方式会带来大量的内存碎片,而且地址指针也会占用额外的内存。而ziplist却是将表中每一项存放在前后连续的地址空间内。一个ziplist整体占用一大块内存,它是一个表(list),但其实不是一个链表(linked list)。

ziplist为了节省内存,提高存储效率,对于值的存储采用了变长的编码方式,大概意思是说,对于大的整数,就多用一些字节来存储,而对于小的整数,就少用一些字节来存储。

\end{document}






