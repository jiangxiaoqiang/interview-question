\documentclass[../../../interview-questions.tex]{subfiles}

\begin{document}

\subsection{Redis的数据驱逐机制(Redis Eviction)}

Redis 主要有 2 种过期删除策略,一种是惰性删除,一种是定期删除。惰性删除指的是当我们查询 key 的时候才对 key 进行检测,如果已经达到过期时间,则删除。显然,他有一个缺点就是如果这些过期的 key 没有被访问,那么他就一直无法被删除,而且一直占用内存。定期删除指的是 redis 每隔一段时间对数据库做一次检查,删除里面的过期 key。由于不可能对所有 key 去做轮询来删除,所以 redis 会每次随机取一些 key 去做检查和删除。假设 redis 每次定期随机查询 key 的时候没有删掉,这些 key 也没有做查询的话,就会导致这些 key 一直保存在 redis 里面无法被删除,这时候就会走到 redis 的内存淘汰机制。

Redis的数据驱逐策略在使用达到内存上限时应用。Redis的数据淘汰机制可以从官方文档https://redis.io/docs/manual/eviction/查看。volatile翻译:不穩定的;易變的;易怒的,喜怒無常的, (液體或固體)易揮發的,易氣化的。

noeviction

禁止驱逐数据,永远不过期,仅对写操作返回一个错误,默认为该项

allkeys-random

从数据集(server.db[i].dict)中任意选择数据淘汰

allkeys-lfu(least frequently used (LFU) keys)

从数据集(server.db[i].dict)中挑选最近最少频次使用的数据淘汰

volatile-lru(Least Recently Used)

从已设置过期时间的数据集(server.db[i].expires)中挑选最近最少使用的数据淘汰

volatile-random

从已设置过期时间的数据集(server.db[i].expires)中任意选择数据淘汰

volatile-ttl

从已设置过期时间的数据集(server.db[i].expires)中挑选将要过期的数据淘汰


Redis有了LRU之后,为什么还需要LFU呢?因为Redis作者发现就算提高采样数量或者pool的大小,也无法再提高缓存命中率,而LFU算法能起到更好的效果。

LFU近似于LRU,它使用一个概率计数器(morris counter),用来估计访问频率;counter的计数有两个特点,1.随着访问次数的增加,counter的计数会越来越缓慢(counter最大值为255),2.随着时间的流逝,counter会逐渐衰减。淘汰时也会有一个pool,也采取与LRU类似的方式,但是排序是按照计数从大到小排列(越靠后越容易被淘汰)
为什么Redis不采用标准LRU算法:

标准LRU算法为了达到查找和删除的时间复杂度一般采用hash表和双向链表结合的数据结构。这样会增加额外的内存占用。

\end{document}






