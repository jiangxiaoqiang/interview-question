\documentclass[../../../interview-questions.tex]{subfiles}

\begin{document}

\subsection{Redis简单动态字符串}

Redis 是用 C 语言写的,但是对于Redis的字符串,却不是 C 语言中的字符串(即以空字符’0’结尾的字符数组),它是自己构建了一种名为 简单动态字符串(simple dynamic string,SDS)的抽象类型,并将 SDS 作为 Redis的默认字符串表示。 


\paragraph{为什么使用SDS}

\begin{enumerate}
    \item {\textbf{常数复杂度获取字符串长度}}由于 len 属性的存在,我们获取 SDS 字符串的长度只需要读取 len 属性,时间复杂度为 O(1)\footnote{参见:\url{https://pdai.tech/md/db/nosql-redis/db-redis-x-redis-ds.html}}。而对于 C 语言,获取字符串的长度通常是经过遍历计数来实现的,时间复杂度为 O(n)。通过 strlen key 命令可以获取 key 的字符串长度。
    \item {\textbf{杜绝缓冲区溢出}}我们知道在 C 语言中使用 strcat  函数来进行两个字符串的拼接,一旦没有分配足够长度的内存空间,就会造成缓冲区溢出。而对于 SDS 数据类型,在进行字符修改的时候,会首先根据记录的 len 属性检查内存空间是否满足需求,如果不满足,会进行相应的空间扩展,然后在进行修改操作,所以不会出现缓冲区溢出。
    \item {\textbf{减少修改字符串的内存重新分配次数}}C语言由于不记录字符串的长度,所以如果要修改字符串,必须要重新分配内存(先释放再申请),因为如果没有重新分配,字符串长度增大时会造成内存缓冲区溢出,字符串长度减小时会造成内存泄露。 而对于SDS,由于len属性和alloc属性的存在,对于修改字符串SDS实现了空间预分配和惰性空间释放两种策略: 1、空间预分配:对字符串进行空间扩展的时候,扩展的内存比实际需要的多,这样可以减少连续执行字符串增长操作所需的内存重分配次数。 2、惰性空间释放:对字符串进行缩短操作时,程序不立即使用内存重新分配来回收缩短后多余的字节,而是使用 alloc 属性将这些字节的数量记录下来,等待后续使用。(当然SDS也提供了相应的API,当我们有需要时,也可以手动释放这些未使用的空间。)
    \item {\textbf{二进制安全}}因为C字符串以空字符作为字符串结束的标识,而对于一些二进制文件(如图片等),内容可能包括空字符串,因此C字符串无法正确存取;而所有 SDS 的API 都是以处理二进制的方式来处理 buf 里面的元素,并且 SDS 不是以空字符串来判断是否结束,而是以 len 属性表示的长度来判断字符串是否结束。 
    \item {\textbf{兼容部分 C 字符串函数}}虽然 SDS 是二进制安全的,但是一样遵从每个字符串都是以空字符串结尾的惯例,这样可以重用 C 语言库<string.h> 中的一部分函数。
\end{enumerate}
        


\end{document}






