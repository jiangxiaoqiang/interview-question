\documentclass[../../../interview-questions.tex]{subfiles}

\begin{document}

\subsection{Redis数据持久化}

BGSAVE可以在不阻塞主进程的情况下完成数据的备份。可以通过redis.conf中设置多个自动保存条件,只要有一个条件被满足,服务器就会执行BGSAVE命令。BGSAVE原理是fork() + copyonwrite。

AOF持久化(Append-Only-File),与RDB\footnote{RDB(Redis Database)是Redis用来进行持久化的一种方式,是把当前内存中的数据集快照写入磁盘,也就是 Snapshot 快照(数据库中所有键值对数据)。恢复时是将快照文件直接读到内存里。}持久化不同,AOF持久化是通过保存Redis服务器锁执行的写状态来记录数据库的。
具体来说,RDB持久化相当于备份数据库状态,而AOF持久化是备份数据库接收到的命令,所有被写入AOF的命令都是以redis的协议格式来保存的。
在AOF持久化的文件中,数据库会记录下所有变更数据库状态的命令,除了指定数据库的select命令,其他的命令都是来自client的,这些命令会以追加(append)的形式保存到文件中。

\end{document}






