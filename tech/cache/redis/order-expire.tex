\documentclass[../../../interview-questions.tex]{subfiles}

\begin{document}

\subsection{Redis应用之订单过期优雅实现}

核心是利用Redis的过期事件通知特性。一般在订单生成后,一段时间用户未支付会自动取消订单或过期订单。如何优雅的实现这样的业务呢?在生成订单时,向 Redis 中增加一个 KV 键值对,K 为订单号,保证通过 K 能定位到数据库中的某个订单即可,V 可为任意值。假设,生成订单时向 Redis 中存放 K 为订单号,V 也为订单号的键值对,并设置过期时间为 30 分钟,如果该键值对在 30 分钟过期后能够发送给程序一个通知,或者执行一个方法,那么即可解决订单关闭问题。实现:通过监听 Redis 提供的过期队列来实现,监听过期队列后,如果 Redis 中某一个 KV 键值对过期了,那么将向监听者发送消息,监听者可以获取到该键值对的 K,注意,是获取不到 V 的,因为已经过期了,这就是上面所提到的,为什么要保证能通过 K 来定位到订单,而 V 为任意值即可。拿到 K 后,通过 K 定位订单,并判断其状态,如果是未支付,更新为关闭,或者取消状态即可。

\paragraph{开启 Redis key 过期提醒}


修改 redis 相关事件配置。找到 redis 配置文件 redis.conf,查看 notify-keyspace-events 配置项,如果没有,添加 notify-keyspace-events Ex,如果有值,则追加 Ex,相关参数说明如下:

K:keyspace 事件,事件以 keyspace@ 为前缀进行发布

E:keyevent 事件,事件以 keyevent@ 为前缀进行发布

g:一般性的,非特定类型的命令,比如del,expire,rename等

\$:字符串特定命令

l:列表特定命令

s:集合特定命令

h:哈希特定命令

z:有序集合特定命令

x:过期事件,当某个键过期并删除时会产生该事件

e:驱逐事件,当某个键因 maxmemore 策略而被删除时,产生该事件

A:g\$lshzxe的别名,因此”AKE”意味着所有事件

\end{document}






