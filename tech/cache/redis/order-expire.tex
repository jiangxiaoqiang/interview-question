\documentclass[../../../interview-questions.tex]{subfiles}

\begin{document}

\subsection{Redis应用之订单过期优雅实现}


用MQ的延迟队列,不要使用Redis的过期监听。redis 自动过期的实现方式是:定时任务离线扫描并删除部分过期键;在访问键时惰性检查是否过期并删除过期键。redis 从未保证会在设定的过期时间立即删除并发送过期通知。实际上,过期通知晚于设定的过期时间数分钟的情况也比较常见。此外键空间通知采用的是发送即忘 (fire and forget) 策略,并不像消息队列一样保证送达。当订阅事件的客户端会丢失所有在断线期间所有分发给它的事件。https://learnku.com/articles/69335


\paragraph{开启 Redis key 过期提醒}


修改 redis 相关事件配置。找到 redis 配置文件 redis.conf,查看 notify-keyspace-events 配置项,如果没有,添加 notify-keyspace-events Ex,如果有值,则追加 Ex,相关参数说明如下:

K:keyspace 事件,事件以 keyspace@ 为前缀进行发布

E:keyevent 事件,事件以 keyevent@ 为前缀进行发布

g:一般性的,非特定类型的命令,比如del,expire,rename等

\$:字符串特定命令

l:列表特定命令

s:集合特定命令

h:哈希特定命令

z:有序集合特定命令

x:过期事件,当某个键过期并删除时会产生该事件

e:驱逐事件,当某个键因 maxmemore 策略而被删除时,产生该事件

A:g\$lshzxe的别名,因此”AKE”意味着所有事件

\end{document}






