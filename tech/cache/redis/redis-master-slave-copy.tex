\documentclass[../../../interview-questions.tex]{subfiles}

\begin{document}

\subsection{Redis主从复制原理}

Redis主从复制就我的总结分为3个阶段,准备阶段(Prepare)、数据同步阶段(Data Sync)、命令传播阶段(Command Sync)。准备阶段包括建立套接字连接、发送PING、认证。数据同步阶段主节点fork进程生成RDB文件发送给slave节点,slave节点同步,并根据offset跟上主节点的进度。命令传播阶段就是从节点根据主节点发送的命令执行与主节点一致的数据操作。

在分布式环境中,数据副本 (Replica) 和复制 (Replication) 作为提升系统可用性和读写性能的有效手段被大量应用在各种分布式系统中,Redis 也不例外。Redis主从复制,以一主多从的模式建立的分布式系统,是Redis搭建高可用集群(哨兵模式、Cluster模式)的基础,为容错、故障转移提供强有力的支撑。

在主从复制模式下,Redis使用一对Replicaion ID, offset来唯一识别Master节点数据集的版本,要理解这个“版本“的概念需要认识Redis的以下三个概念:

Replication ID(复制ID):每个Redis的主节点都用一个随机生成的字符串来表示在某一时刻其内部存储数据的状态,“某一时刻”可以理解为其成为master角色的那一刻,由源码可知在第一个从节点加入时,Redis初始化了复制ID。

offset(复制偏移量):主从模式下,主节点会持续不断的向从节点传播引起数据集更改的命令,offset所表示的是主节点向从节点传递命令字节总数。它不是孤立存在的,需要配合复制积压缓冲区才能工作。

backlog(复制积压缓冲区):它是一个环形缓冲区,用来存储主节点向从节点传递的命令,它的大小是固定的,可存储的命令有限,超出部分将会被删除。它即可用于部分同步,也可用于命令传播阶段的命令重推。



总的来说主从复制功能的详细步骤可以分为7个步骤:

设置主节点的地址和端口
建立套接字连接
发送PING命令
权限验证
同步
命令传播

\paragraph{主从复制问题}

(1)延迟与不一致问题

(2)数据过期问题

(3)故障切换问题

\end{document}






