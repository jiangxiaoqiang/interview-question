\documentclass[../../../interview-questions.tex]{subfiles}

\begin{document}

\subsection{Redis集群}

集群的作用,可以归纳为两点:

1、数据分区:数据分区(或称数据分片)是集群最核心的功能。

集群将数据分散到多个节点,一方面突破了Redis单机内存大小的限制,存储容量大大增加;另一方面每个主节点都可以对外提供读服务和写服务,大大提高了集群的响应能力。

Redis单机内存大小受限问题,在介绍持久化和主从复制时都有提及;例如,如果单机内存太大,bgsave和bgrewriteaof的fork操作可能导致主进程阻塞,主从环境下主机切换时可能导致从节点长时间无法提供服务,全量复制阶段主节点的复制缓冲区可能溢出……。

2、高可用:集群支持主从复制和主节点的自动故障转移(与哨兵类似);当任一节点发生故障时,集群仍然可以对外提供服务。

\paragraph{集群方案设计}

(1)高可用要求:根据故障转移的原理,至少需要3个主节点才能完成故障转移,且3个主节点不应在同一台物理机上;每个主节点至少需要1个从节点,且主从节点不应在一台物理机上;因此高可用集群至少包含6个节点。

(2)数据量和访问量:估算应用需要的数据量和总访问量(考虑业务发展,留有冗余),结合每个主节点的容量和能承受的访问量(可以通过benchmark得到较准确估计),计算需要的主节点数量。

(3)节点数量限制:Redis官方给出的节点数量限制为1000,主要是考虑节点间通信带来的消耗。在实际应用中应尽量避免大集群;如果节点数量不足以满足应用对Redis数据量和访问量的要求,可以考虑:

\begin{enumerate}
    \item {业务分割,大集群分为多个小集群;}
    \item {减少不必要的数据;}
    \item {调整数据过期策略等。}
    \item {适度冗余:Redis可以在不影响集群服务的情况下增加节点,因此节点数量适当冗余即可,不用太大。}
\end{enumerate}

\end{document}






