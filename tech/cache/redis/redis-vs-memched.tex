\documentclass[../../../interview-questions.tex]{subfiles}

\begin{document}

\subsubsection{Redis和memcached:什么时候选择Redis,什么时候选择Memcached,内存模型和存储策略是什么样的}


选择Redis\footnote{\url{http://redisdoc.com/topic/index.html}}还是Memcached,需要根据具体的应用场景和他们之间的区别:

\begin{enumerate}
\item {\textbf{数据类型支持不同}}与Memcached仅支持简单的key-value结构的数据记录不同,Redis支持的数据类型要丰富得多。最为常用的数据类型主要由五种:String、Hash、List、Set和Sorted Set。Redis内部使用一个redisObject对象来表示所有的key和value。
\item {\textbf{数据持久化支持}}Redis虽然是基于内存的存储系统,但是它本身是支持内存数据的持久化的,而且提供两种主要的持久化策略:RDB快照和AOF日志。而memcached是不支持数据持久化操作的。
\item {\textbf{集群管理不同}}Memcached本身并不支持分布式,因此只能在客户端通过像一致性哈希这样的分布式算法来实现Memcached的分布式存储。
\end{enumerate}                                  

内存模型参考\footnote{\url{http://www.dongcoder.com/detail-1031666.html}},如图所示。



字符串是 Redis最基本的数据类型,Redis 中字符串对象的编码可以是 int,raw 或者 embstr 中的某一种.




\end{document}






