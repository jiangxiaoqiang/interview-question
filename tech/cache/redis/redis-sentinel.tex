\documentclass[../../../interview-questions.tex]{subfiles}

\begin{document}

\subsection{Redis哨兵}

本文首先介绍了哨兵的作用:监控、故障转移、配置提供者和通知;然后讲述了哨兵系统的部署方法,以及通过客户端访问哨兵系统的方法;再然后简要说明了哨兵实现的基本原理;最后给出了关于哨兵实践的一些建议。

在主从复制的基础上,哨兵引入了主节点的自动故障转移,进一步提高了Redis的高可用性;但是哨兵的缺陷同样很明显:哨兵无法对从节点进行自动故障转移,在读写分离场景下,从节点故障会导致读服务不可用,需要我们对从节点做额外的监控、切换操作。

此外,哨兵仍然没有解决写操作无法负载均衡、及存储能力受到单机限制的问题;这些问题的解决需要使用集群。

\end{document}






