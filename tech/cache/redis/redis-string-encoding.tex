\documentclass[../../../interview-questions.tex]{subfiles}

\begin{document}

\subsection{Redis String编码}

字符串的编码有raw、embstr、int三种。

raw 用于长字符串。
embstr 用于短字符串。
int 用于整数类型。

raw编码主要用来保存长度超过 44 的字符串。其真实数据,由sdshdr结构来表示存储,外层还是由redisObject包装。embstr 编码是专门用于保存短字符串的一种优化编码方式。当字符串的长度小于等于 44 的时候,将采用 embstr 编码。

\begin{itemize}
    \item {分配内存的时候,只需要分配一次。而 raw 编码的sds跟redisObject分离,就要分配两次内存。}
    \item {同样,释放内存也只需要释放一次。}
    \item {连续内存能更好利用内存带来的优势。}
\end{itemize}

https://www.cnblogs.com/chenchuxin/p/14204452.html

\end{document}