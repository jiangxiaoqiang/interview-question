\documentclass[../../../interview-questions.tex]{subfiles}

\begin{document}

\subsection{Redis的事务机制}

Redis 通过 MULTI、EXEC、WATCH 等命令来实现事务机制,事务执行过程将一系列多个命令按照顺序一次性执行,并且在执行期间,事务不会被中断,也不会去执行客户端的其他请求,直到所有命令执行完毕。事务的执行过程如下:

\begin{enumerate}
    \item {服务端收到客户端请求,事务以 MULTI 开始}
    \item {如果客户端正处于事务状态,则会把事务放入队列同时返回给客户端 QUEUED,反之则直接执行这个命令}
    \item {当收到客户端 EXEC 命令时,WATCH 命令监视整个事务中的 key 是否有被修改,如果有则返回空回复到客户端表示失败,否则 Redis 会遍历整个事务队列,执行队列中保存的所有命令,最后返回结果给客户端}
\end{enumerate}

WATCH 的机制本身是一个 CAS 的机制,被监视的 key 会被保存到一个链表中,如果某个 key 被修改,那么 REDIS\_DIRTY\_CAS 标志将会被打开,这时服务器会拒绝执行事务。

\end{document}