\documentclass[../../../interview-questions.tex]{subfiles}

\begin{document}

\subsection{各场景下复制的选择及优化技巧}

在介绍了Redis复制的种种细节之后,现在我们可以来总结一下,在下面常见的场景中,何时使用部分复制,以及需要注意哪些问题。

(1)第一次建立复制
此时全量复制不可避免,但仍有几点需要注意:如果主节点的数据量较大,应该尽量避开流量的高峰期,避免造成阻塞;如果有多个从节点需要建立对主节点的复制,可以考虑将几个从节点错开,避免主节点带宽占用过大。此外,如果从节点过多,也可以调整主从复制的拓扑结构,由一主多从结构变为树状结构(中间的节点既是其主节点的从节点,也是其从节点的主节点);但使用树状结构应该谨慎:虽然主节点的直接从节点减少,降低了主节点的负担,但是多层从节点的延迟增大,数据一致性变差;且结构复杂,维护相当困难。

(2)主节点重启
主节点重启可以分为两种情况来讨论,一种是故障导致宕机,另一种则是有计划的重启。

主节点宕机

主节点宕机重启后,runid会发生变化,因此不能进行部分复制,只能全量复制。

实际上在主节点宕机的情况下,应进行故障转移处理,将其中的一个从节点升级为主节点,其他从节点从新的主节点进行复制;且故障转移应尽量的自动化,后面文章将要介绍的哨兵便可以进行自动的故障转移。

安全重启:debug reload

在一些场景下,可能希望对主节点进行重启,例如主节点内存碎片率过高,或者希望调整一些只能在启动时调整的参数。如果使用普通的手段重启主节点,会使得runid发生变化,可能导致不必要的全量复制。

为了解决这个问题,Redis提供了debug reload的重启方式:重启后,主节点的runid和offset都不受影响,避免了全量复制。


\end{document}
