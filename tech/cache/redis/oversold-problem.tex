\documentclass[../../../interview-questions.tex]{subfiles}

\begin{document}

\subsection{避免商品超卖}

常见的抢购场景中,最重要的一点是要避免商品超卖。如何做到不超卖呢?核心是使用Redis的decrement/increment操作,它是原子的。当大量的下单的请求到来时,直接decre缓存中预先加载的商品总数,当然如果没有加载商品总数时,可先获取分布式锁,将商品总数放入内存中,不能直接加载商品总数,否则会有大量的线程同时加载。

当缓存商品数量小于等于0时,直接返回无库存相关文案。有几个问题,当流量确实非常大时,可以考虑将商品拆分成几份存放到Redis集群中,多台服务器分担压力。还有就是商品数量非常多时,例如商品有10万件,Redis此时做decre操作后会产生大量的订单任务,此时数据库不管是写入还是更新库存都很难立即响应所有请求的结果,如何保证界面立即响应?此时可以考虑将瞬间产生的大量任务写入内存队列,或者Redis的Stream队列中(推荐写入Stream,因为Redis本身有AOF持久化,可以在宕机的情况下尽可能少的丢失数据),直接返回成功(这里会不会有decre成功但是持久化失败的情况?不敢保证绝对不存在,但是概率很低)。即使持久化失败了,也不会存在超卖的情况。

还有考虑可以存储一个商品状态标记,表示有无库存,在decre之前先检查,这样后期就可以直接根据状态标记的结果直接返回,不会再调取decre操作。如果不这样,每次下单还是会继续调取decre操作,存在数据溢出可能。

\paragraph{更新库存}

在后台更新商品数量时,也要利用Redis原子增操作的特性,可以使用阶梯加、阶梯减。

\end{document}






