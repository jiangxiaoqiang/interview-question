\documentclass[../../../interview-questions.tex]{subfiles}

\begin{document}

\subsection{Filter和Interceptor区别}

Filter是基于函数回调(doFilter()方法)的,而Interceptor则是基于Java反射的(AOP思想)。

Filter依赖于Servlet容器(Servlet容器也叫做Servlet引擎,是Web服务器或应用程序服务器的一部分,用于在发送的请求和响应之上提供网络服务,解码基于 MIME的请求,格式化基于MIME的响应。Servlet没有main方法,不能独立运行,它必须被部署到Servlet容器中,由容器来实例化和调用 Servlet的方法(如doGet()和doPost()),Servlet容器在Servlet的生命周期内包容和管理Servlet。在JSP技术 推出后,管理和运行Servlet/JSP的容器也称为Web容器。),而Interceptor不依赖于Servlet容器。

Filter对几乎所有的请求起作用,而Interceptor只能对action请求起作用。

Interceptor可以访问Action的上下文,值栈里的对象,而Filter不能。

在action的生命周期里,Interceptor可以被多次调用,而Filter只能在容器初始化时调用一次。

Filter在过滤是只能对request和response进行操作,而interceptor可以对request、response、handler、modelAndView、exception进行操作。

https://www.cnblogs.com/junzi2099/p/8022058.html

\end{document}





