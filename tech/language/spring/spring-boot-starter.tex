
\documentclass[../../../interview-questions.tex]{subfiles}

\begin{document}

\subsection{Spring Boot Starter原理}

starter可以理解为一个依赖组,其主要功能就是完成引入依赖和初始化配置。@SpringBootApplication 注解中核心注解@EnableAutoConfiguration注解在starter起什么作用呢?@EnableAutoConfiguration源码分析:

\begin{lstlisting}[language=Java]
@Target(ElementType.TYPE)
@Retention(RetentionPolicy.RUNTIME)
@Documented
@Inherited
@AutoConfigurationPackage
@Import(AutoConfigurationImportSelector.class)
public @interface EnableAutoConfiguration {
    String ENABLED_OVERRIDE_PROPERTY = "spring.boot.enableautoconfiguration";
    Class<?>[] exclude() default {};
    String[] excludeName() default {};
}
\end{lstlisting}

从源码看出关键功能是@import注解导入自动配置功能类AutoConfigurationImportSelector类,主要方法getCandidateConfigurations()使用了\url{SpringFactoriesLoader.loadFactoryNames()}方法加载META-INF/spring.factories的文件(spring.factories声明具体自动配置)。项目启动的时候会去加载项目中所有的 spring.factories 文件,然后加载对应的配置类,因此我们就需要在 spring.factories 中只指定需要扫描的类。Spring Boot 的 starter 的核心其实就是通过 SPI 机制自动注入配置类,不过是它自己实现的一套SPI机制,我们先了解一下Java的SPI机制。SPI 全称 Service Provider Interface,是 Java 提供的一套用来被第三方实现或者扩展的 API,它可以用来启用框架扩展和替换组件。SPI的大概流程是:调用方 –> 标准服务接口 –> 本地服务发现 (配置文件) –> 具体实现。

\paragraph{如何开发一个自定义的starter}

开发一个 starter 简单来说以下几步即可:

\begin{itemize}
    \item {一个/多个自定义配置的属性配置类(可选)}
    \item {一个/多个自动配置类}
    \item {自动配置类写入到 Spring Boot 的 SPI 机制配置文件:spring.factories}
\end{itemize}

\end{document}





