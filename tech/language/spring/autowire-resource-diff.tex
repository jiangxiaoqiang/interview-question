
\documentclass[../../../interview-questions.tex]{subfiles}

\begin{document}

\subsection{@Autowired和@Resource的区别}

\begin{itemize}
    \item {@Autowired是Spring的注解,@Resource是Java的注解。}
    \item {@Autowired默认按类型匹配,如果有多个同类型的bean,可以结合@Qualifier指定名称匹配。@Resource默认按名称匹配,如果没有指定name属性,就按字段名匹配,如果没有找到同名的bean,就按类型匹配。}
    \item {@Autowired只有一个属性required,表示是否必须注入,默认为true。@Resource有七个属性,其中最重要的两个是name和type,分别表示bean的名称和类型。}
\end{itemize}

@Autowired为Spring提供的注解,只按照byType注入。@Autowired注解是按照类型(byType)装配依赖对象,默认情况下它要求依赖对象必须存在,如果允许null值,可以设置它的required属性为false。通过类型注入的时候,有时会因为多个 Bean 的类型相同而产生冲突。例如:同一类型注册多个不同名称的Bean或者抽象类型注册多个不同实现类的 Bean。如果我们想使用按照名称(byName)来装配,可以结合@Qualifier注解一起使用。注解驱动配置会向Spring容器中注册AutowiredAnnotationBeanPostProcessor
当 Spring 容器启动时,AutowiredAnnotationBeanPostProcessor 将扫描 Spring 容器中所有 Bean,当发现 Bean 中拥有 @Autowired 注释时就找到和其匹配(默认按类型匹配)的 Bean,并注入到对应的地方中去。

@Resource 是JDK1.6支持的注解,默认按照名称进行装配,名称可以通过name属性进行指定。也提供按照byType 注入。@Resource默认按照ByName自动注入,由J2EE提供,需要导入包javax.annotation.Resource。@Resource有两个重要的属性:name和type,而Spring将@Resource注解的name属性解析为bean的名字,而type属性则解析为bean的类型。所以,如果使用name属性,则使用byName的自动注入策略,而使用type属性时则使用byType自动注入策略。如果既不制定name也不制定type属性,这时将通过反射机制使用byName自动注入策略。

如果没有指定name属性,当注解写在字段上时,默认取字段名,按照名称查找。
当注解标注在属性的setter方法上,即默认取属性名作为bean名称寻找依赖对象。
当找不到与名称匹配的bean时才按照类型进行装配。但是需要注意的是,如果name属性一旦指定,就只会按照名称进行装配。

\end{document}





