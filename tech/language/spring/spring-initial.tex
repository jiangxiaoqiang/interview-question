
\documentclass[../../../interview-questions.tex]{subfiles}

\begin{document}

\subsection{Spring容器启动过程}

容器启动的过程可以分为2大步

1:获取、解析、注册配置信息,将配置的文件信息转换Map<name,beanDefinition>

2:根据上述的Map<name,beanDefinition>去实例化bean,并完成依赖注入

\paragraph{获取Bean定义}

获取、解析、注册配置信息,将配置的文件信息转换Map<name,beanDefinition>,启动Spring容器实质上就是创建一个ApplicationContext的上下文对象,这个对象描述了Spring容器的基本操作,包括资源的加载(ResourcePatternResolver)、Bean的创建(ListableBeanFactory, HierarchicalBeanFactory)、应用环境的获取(EnvironmentCapable)、应用的事件推送(ApplicationEventPublisher)等。

而该接口最重要的抽象实现AbstractApplicationContext,因为所有容器的初始化流程都是委托到这个类的refresh()方法中。可以说弄清楚了AbstractApplicationContext.refresh()方法,就弄清楚了Spring容器初始化的整个流程。那么第一步的处理配置信息在哪里呢?在AbstractRefreshableApplicationContext类的refreshBeanFactory方法中。

\begin{lstlisting}[language=Java]
// AbstractRefreshableApplicationContext.java
@Override
protected final void refreshBeanFactory() throws BeansException {
    if (hasBeanFactory()) {
        destroyBeans();
        closeBeanFactory();
    }
    try {
        DefaultListableBeanFactory beanFactory = createBeanFactory();
        beanFactory.setSerializationId(getId());
        customizeBeanFactory(beanFactory);
        loadBeanDefinitions(beanFactory);
        this.beanFactory = beanFactory;
    }
    catch (IOException ex) {
        throw new ApplicationContextException("I/O error parsing bean definition source for " + getDisplayName(), ex);
    }
}    
\end{lstlisting}

其中AbstractBeanDefinitionReader类的loadBeanDefinitions是Spring容器的关键步骤之一,它实现了载入我们配置的xml文件或者注解配置,并将其解析为BeanDefinition,以供创建Bean使用。

\paragraph{注入bean}

根据转换得到的Map<name,beanDefinition>去实例化bean,并完成注入。


\end{document}





