
\documentclass[../../../interview-questions.tex]{subfiles}

\begin{document}

\subsection{Spring 框架中用到了哪些设计模式}

工厂设计模式 : Spring使用工厂模式通过 BeanFactory、ApplicationContext 创建 bean 对象。

代理设计模式 : Spring AOP 功能的实现。

单例设计模式 : Spring 中的 Bean 默认都是单例的。

模板方法模式 : Spring 中 jdbcTemplate、hibernateTemplate 等以 Template 结尾的对数据库操作的类,它们就使用到了模板模式。

包装器设计模式 : 我们的项目需要连接多个数据库,而且不同的客户在每次访问中根据需要会去访问不同的数据库。这种模式让我们可以根据客户的需求能够动态切换不同的数据源。

观察者模式: Spring 事件驱动模型就是观察者模式很经典的一个应用。

适配器模式 :Spring AOP 的增强或通知(Advice)使用到了适配器模式、spring MVC 中也是用到了适配器模式适配Controller。

\end{document}





