
\documentclass[../../../interview-questions.tex]{subfiles}

\begin{document}

\subsection{Spring AOP原理}

AOP(Aspect Orient Programming)的拦截功能是由Java中的动态代理(Dynamic Proxy)来实现的。代理类在程序运行时创建的代理方式被成为动态代理,也就是说,这种情况下,代理类并不是在Java代码中定义的,而是在运行时根据我们在Java代码中的“指令”动态生成的。相比于静态代理, 动态代理的优势在于可以很方便的对代理类的函数进行统一的处理,而不用修改每个代理类的函数。

​ 相比于静态代理来说,动态代理更加灵活。我们不需要针对每个目标类都单独创建一个代理类,并且也不需要我们必须实现接口,我们可以直接代理实现类( CGLIB 动态代理机制)。

​ 动态代理其实是一种方便运行时候动态的处理代理方法的调用机制,通过代理可以让调用者和实现者之间解耦。

​ 从 JVM 角度来说,动态代理是在运行时动态生成类字节码,并加载到 JVM 中的。JDK动态代理主要是基于反射,使用反射解析目标对象的属性、方法等

根据解析的内容生成proxy.class,说白了就是把要生成的class按照字符串的形式拼接,最终通过ClassLoader加载。JDK 动态代理有一个最致命的问题是只能代理实现了接口的类。

​ CGLIB是一个基于ASM的字节码生成库,它允许我们在运行时对字节码进行修改和动态生成。CGLIB 通过继承方式实现代理。很多知名的开源框架都使用到了CGLIB, 例如 Spring 中的 AOP 模块中:如果目标对象实现了接口,则默认采用 JDK 动态代理,否则采用 CGLIB 动态代理。

​ CGLIB是针对类来实现代理的,他的原理是对代理的目标类生成一个子类,并覆盖其中方法实现增强,因为底层是基于创建被代理类的一个子类,所以它避免了JDK动态代理类的缺陷。AOP的源码中用到了两种动态代理来实现拦截切入功能:jdk动态代理和cglib动态代理。两种方法同时存在,各有优劣。 jdk动态代理是由java内部的反射机制来实现的,cglib动态代理底层则是借助asm来实现的。 总的来说,反射机制在生成类的过程中比较高效,执行时候通过反射调用委托类接口方法比较慢;而asm在生成类之后的相关代理类执行过程中比较高效(可以通过将asm生成的类进行缓存,这样解决asm生成类过程低效问题)。 还有一点必须注意:jdk动态代理的应用前提,必须是委托类基于统一的接口。如果没有上述前提,jdk动态代理不能应用。 由此可以看出,jdk动态代理有一定的局限性,cglib这种第三方类库实现的动态代理应用更加广泛,且在效率上更有优势。

​ 实现AOP关键特点是定义好两个角色 切点 和 切面 。 代理模式中被代理类 委托类处于切点角色,需要添加的其他比如 校验逻辑,事务,审计逻辑 属于非功能实现逻辑通过 切面类定义的方法插入进去。

在目标类的基础上增加切面逻辑,生成增强的目标类(该切面逻辑或者在目标类函数执行之前,或者目标类函数执行之后,或者在目标类函数抛出异常时候执行。不同的切入时机对应不同的Interceptor的种类,如BeforeAdviseInterceptor,AfterAdviseInterceptor以及ThrowsAdviseInterceptor等)。JDK动态代理:java.lang.reflect.InvocationHandler。该动态代理是基于接口的动态代理,所以并没有一个原始方法的调用过程,整个方法都是被拦截的。通过Cglib动态创建类进行动态代理。org.springframework.cglib.proxy包下的原生接口,同net.sf.cglib.proxy包下的接口,都是源自cglib库。Spring内部的cglib动态代理使用了这种方式。org.aopalliance的拦截体系
该包是AOP组织下的公用包,用于AOP中方法增强和调用。相当于一个jsr标准,只有接口和异常。在AspectJ、Spring等AOP框架中使用。JDK的动态代理依靠接口实现,如果有些类并没有实现接口,则不能使用JDK代理。这个时候就需要使用Cglib在字节码上做代理。动态代理的底层是通过Java的反射实现,Java反射通过JVM的native方法实现。Spring提供了两种方式来生成代理对象: JDKProxy和Cglib,具体使用哪种方式生成由AopProxyFactory根据AdvisedSupport对象的配置来决定。默认的策略是如果目标类是接口,则使用JDK动态代理技术,否则使用Cglib来生成代理。


AOP使用场景
AOP用来封装横切关注点,具体可以在下面的场景中使用:

Authentication 权限
Caching 缓存
Context passing 内容传递
Error handling 错误处理
Lazy loading 懒加载
Debugging  调试
logging, tracing, profiling and monitoring 记录跟踪 优化 校准
Performance optimization 性能优化
Persistence  持久化
Resource pooling 资源池
Synchronization 同步
Transactions 事务

\end{document}





