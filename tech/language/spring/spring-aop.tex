
\documentclass[../../../interview-questions.tex]{subfiles}

\begin{document}

\subsection{Spring AOP原理}

AOP(Aspect Orient Programming)的拦截功能是由Java中的动态代理(Dynamic Proxy)来实现的\footnote{参见:\url{https://docs.spring.io/spring-framework/docs/current/reference/html/core.html\#aop}},准确的说应该默认是用Java的动态代理来实现,当代理的是类而非接口的时候是通过CGLIB来实现的(Spring AOP can also use CGLIB proxies. This is necessary to proxy classes rather than interfaces.)。代理类在程序运行时创建的代理方式被成为动态代理,也就是说,这种情况下,代理类并不是在Java代码中定义的,而是在运行时根据我们在Java代码中的“指令”动态生成的。相比于静态代理, 动态代理的优势在于可以很方便的对代理类的函数进行统一的处理,而不用修改每个代理类的函数。

​ 相比于静态代理来说,动态代理更加灵活。我们不需要针对每个目标类都单独创建一个代理类,并且也不需要我们必须实现接口(更严谨的说法是CGLIB动态代理不需要实现接口,Java的动态代理还是要实现InvocationHandler接口的),我们可以直接代理实现类( CGLIB 动态代理机制)。

​ 动态代理其实是一种方便运行时候动态的处理代理方法的调用机制,通过代理可以让调用者和实现者之间解耦。

​ 从 JVM 角度来说,动态代理是在运行时动态生成类字节码,并加载到 JVM 中的。JDK动态代理主要是基于反射,使用反射解析目标对象的属性、方法等。

根据解析的内容生成proxy.class,说白了就是把要生成的class按照字符串的形式拼接,最终通过ClassLoader加载。JDK 动态代理有一个最致命的问题是只能代理实现了接口的类。

​ CGLIB是一个基于ASM的字节码生成库,它允许我们在运行时对字节码进行修改和动态生成。CGLIB 通过继承方式实现代理。很多知名的开源框架都使用到了CGLIB, 例如 Spring 中的 AOP 模块中:如果目标对象实现了接口,则默认采用 JDK 动态代理,否则采用 CGLIB 动态代理。新版的Java使用bytebuddy\footnote{\url{https://github.com/raphw/byte-buddy}}。

​ CGLIB是针对类来实现代理的,他的原理是对代理的目标类生成一个子类,并覆盖其中方法实现增强,因为底层是基于创建被代理类的一个子类,所以它避免了JDK动态代理类的缺陷。AOP的源码中用到了两种动态代理来实现拦截切入功能:jdk动态代理和cglib动态代理。两种方法同时存在,各有优劣。 jdk动态代理是由java内部的反射机制来实现的,cglib动态代理底层则是借助asm来实现的。 总的来说,反射机制在生成类的过程中比较高效,执行时候通过反射调用委托类接口方法比较慢;而asm在生成类之后的相关代理类执行过程中比较高效(可以通过将asm生成的类进行缓存,这样解决asm生成类过程低效问题)。 还有一点必须注意:jdk动态代理的应用前提,必须是委托类基于统一的接口。如果没有上述前提,jdk动态代理不能应用。 由此可以看出,jdk动态代理有一定的局限性,cglib这种第三方类库实现的动态代理应用更加广泛,且在效率上更有优势。

​ 实现AOP关键特点是定义好两个角色 切点 和 切面 。 代理模式中被代理类 委托类处于切点角色,需要添加的其他比如 校验逻辑,事务,审计逻辑 属于非功能实现逻辑通过 切面类定义的方法插入进去。

在目标类的基础上增加切面逻辑,生成增强的目标类(该切面逻辑或者在目标类函数执行之前,或者目标类函数执行之后,或者在目标类函数抛出异常时候执行。不同的切入时机对应不同的Interceptor的种类,如BeforeAdviseInterceptor,AfterAdviseInterceptor以及ThrowsAdviseInterceptor等)。JDK动态代理:java.lang.reflect.InvocationHandler。该动态代理是基于接口的动态代理,所以并没有一个原始方法的调用过程,整个方法都是被拦截的。通过Cglib动态创建类进行动态代理。org.springframework.cglib.proxy包下的原生接口,同net.sf.cglib.proxy包下的接口,都是源自cglib库。Spring内部的cglib动态代理使用了这种方式。org.aopalliance的拦截体系
该包是AOP组织下的公用包,用于AOP中方法增强和调用。相当于一个jsr标准,只有接口和异常。在AspectJ、Spring等AOP框架中使用。JDK的动态代理依靠接口实现,如果有些类并没有实现接口,则不能使用JDK代理。这个时候就需要使用Cglib在字节码上做代理。动态代理的底层是通过Java的反射实现,Java反射通过JVM的native方法实现。Spring提供了两种方式来生成代理对象: JDKProxy和Cglib,具体使用哪种方式生成由AopProxyFactory根据AdvisedSupport对象的配置来决定。默认的策略是如果目标类是接口,则使用JDK动态代理技术,否则使用Cglib来生成代理。AOP用来封装横切关注点,具体可以在下面的场景中使用:

\begin{enumerate}
    \item {Authentication 权限}
    \item {Caching 缓存}
    \item {Context passing 内容传递}
    \item {Error handling 错误处理}
    \item {Lazy loading 懒加载}
    \item {Debugging  调试}
    \item {logging, tracing, profiling and monitoring 记录跟踪 优化 校准}
    \item {Performance optimization 性能优化}
    \item {Persistence  持久化}
    \item {Resource pooling 资源池}
    \item {Synchronization 同步}
    \item {Transactions 事务}
\end{enumerate}

\paragraph{切点Pointcut、切面Aspect、通知Advice、连接点Joinpoint}

http://shouce.jb51.net/spring/aop.html
切面也需要完成工作。在 AOP 术语中,切面的工作被称为通知。通知定义了切面是什么以及何时使用。除了描述切面要完成的工作,通知还解决何时执行这个工作。Spring 切面可应用的 5 种通知类型:

\end{document}





