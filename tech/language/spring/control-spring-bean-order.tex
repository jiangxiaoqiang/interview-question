\documentclass[../../../interview-questions.tex]{subfiles}

\begin{document}

\subsection{控制Bean的加载顺序}

1.通过构造方法依赖的方式,来控制有依赖关系的bean之间初始化顺序,但是需要注意循环依赖的问题

2.@DependsOn注解,来控制bean之间的实例顺序,需要注意的是bean的初始化方法调用顺序无法保证

3.BeanPostProcessor方式,来手动控制bean的加载顺序

4.以上方式只能控制特定的2个Bean的加载顺序,如果需要自定义许多Bean的顺序,可通过实现BeanDefinitionRegistryPostProcessor接口,BeanDefinitionRegistryPostProcessor接口是在Bean定义信息注册后触发,下一步创建Bean实例之前,在所有bean定义信息将要被加载,bean实例还未创建的;优先于BeanFactoryPostProcessor执行;利用BeanDefinitionRegistryPostProcessor给容器中再额外添加一些组件。在postProcessBeanDefinitionRegistry方法中通过BeanDefinitionRegistry获取到所有bean的注册信息,将bean保存到LinkedHashMap中,并从BeanDefinitionRegistry中删除,然后将保存的bean定义排序后,重新再注册到BeanDefinitionRegistry中,即可实现bean加载顺序的控制。

\begin{lstlisting}[language=Java]
public interface BeanDefinitionRegistryPostProcessor extends BeanFactoryPostProcessor {

    /**
        * Modify the application context's internal bean definition registry after its
        * standard initialization. All regular bean definitions will have been loaded,
        * but no beans will have been instantiated yet. This allows for adding further
        * bean definitions before the next post-processing phase kicks in.
        * @param registry the bean definition registry used by the application context
        * @throws org.springframework.beans.BeansException in case of errors
        */
    void postProcessBeanDefinitionRegistry(BeanDefinitionRegistry registry) throws BeansException;

}
\end{lstlisting}



\end{document}