
\documentclass[../../../interview-questions.tex]{subfiles}

\begin{document}

\subsection{Spring事务原理}

Spring声明式事务的实现原理是基于AOP(面向切面编程)的。Spring使用TransactionInterceptor作为切面类,拦截需要进行事务管理的方法,根据方法的执行情况调用PlatformTransactionManager接口来控制事务的开始、提交和回滚。Spring还使用ProxyFactoryBean或者AnnotationTransactionAspect来生成目标对象的代理对象,使得目标对象可以被TransactionInterceptor拦截。在应用系统调用声明@Transactional 的目标方法时,Spring Framework 默认使用 AOP 代理,在代码运行时生成一个代理对象,根据@Transactional 的属性配置信息,这个代理对象决定该声明@Transactional 的目标方法是否由拦截器 TransactionInterceptor 来使用拦截,在 TransactionInterceptor 拦截时,会在在目标方法开始执行之前创建并加入事务,并执行目标方法的逻辑, 最后根据执行情况是否出现异常,利用抽象事务管理器(图 2 有相关介绍)AbstractPlatformTransactionManager 操作数据源 DataSource 提交或回滚事务。https://zhuanlan.zhihu.com/p/98980533。@Transactional实现原理三要素切面、切点、通知

\begin{enumerate}
    \item {InfrastructureAdvisorAutoProxyCreator后置处理器拦截所有Bean}
    \item {遍历所有类型为Advisor的切面}
    \item {返回满足切点条件的切面列表}
    \item {选择代理方法}
    \item {生成代理}
    \item {调用通知的invoke()方法,开启事务,调用其它通知的invoke()方法,如果没有执行目标方法,执行异常,回滚事务,执行成功,提交事务}
    \item {执行目标方法}
\end{enumerate}

了解@Transactional注解实现原理,不仅可以让我们对切面、切点、通知有一个清晰的认识,还可以让我们通过其思想实现类似功能,如@Cache注解实现应用缓存,@Async注解实现业务异步执行。Spring事务基本执行原理如下:

\begin{enumerate}
    \item {Bean的生命周期中,初始化后阶段,通过BeanPostProcessor机制,经过InfrastructureAdvisorAutoProxyCreator的后置处理方法检查当前初始化的Bean是否存在@Transactional注解,随之生成一个代理对象。}
    \item {在外部调用这个代理对象的事务方法的时候,命中MethodInterceptor拦截,检查当前方法是否有匹配的Advisor,有则执行对应的invoke方法,在invoke方法里定义了事务的实现原理。}
    \item {利用配置好的PlatformTransactionManager事务管理器新建一个数据库连接。}
    \item {修改这个Connection的提交模式autoCommit为false,否则每一条SQL都要自动提交。}
    \item {执行MethodInvocation.proceed( )方法,这里面就会调用我们自己的方法逻辑,例如SQL的执行。}
    \item {执行结束,如果没有抛出异常则提交,否则回滚。}
\end{enumerate}

https://itcn.blog/p/4114578327.html

\end{document}





