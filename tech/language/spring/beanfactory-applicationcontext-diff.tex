\documentclass[../../../interview-questions.tex]{subfiles}

\begin{document}

\subsection{BeanFactory和ApplicationContext的区别}

BeanFactory 是 Spring 的“心脏”。它就是 Spring IoC 容器的真面目。Spring 使用 BeanFactory 来实例化、配置和管理 Bean。

BeanFactory:是IOC容器的核心接口, 它定义了IOC的基本功能,我们看到它主要定义了getBean方法。getBean方法是IOC容器获取bean对象和引发依赖注入的起点。方法的功能是返回特定的名称的Bean。

BeanFactory 是初始化 Bean 和调用它们生命周期方法的“吃苦耐劳者”。注意,BeanFactory 只能管理单例(Singleton)Bean 的生命周期。它不能管理原型(prototype,非单例)Bean 的生命周期。这是因为原型 Bean 实例被创建之后便被传给了客户端,容器失去了对它们的引用。

BeanFactory有着庞大的继承、实现体系,有众多的子接口、实现类。


如果说BeanFactory是Spring的心脏,那么ApplicationContext就是完整的躯体了,ApplicationContext由BeanFactory派生而来,提供了更多面向实际应用的功能。在BeanFactory中,很多功能需要以编程的方式实现,而在ApplicationContext中则可以通过配置实现。

BeanFactorty接口提供了配置框架及基本功能,但是无法支持Spring的AOP功能和Web应用。而ApplicationContext接口作为BeanFactory的派生,因而提供BeanFactory所有的功能。而且ApplicationContext还在功能上做了扩展,相较于BeanFactorty,ApplicationContext还提供了以下的功能: 

(1)MessageSource, 提供国际化的消息访问

(2)资源访问,如URL和文件 

(3)事件传播特性,即支持AOP特性

(4)载入多个(有继承关系)上下文 ,使得每一个上下文都专注于一个特定的层次,比如应用的web层 

ApplicationContext:是IoC容器另一个重要接口, 它继承了BeanFactory的基本功能, 同时也继承了容器的高级功能,如:MessageSource(国际化资源接口)、ResourceLoader(资源加载接口)、ApplicationEventPublisher(应用事件发布接口)等。

\paragraph{二者的区别}

1.BeanFactroy采用的是延迟加载形式来注入Bean的,即只有在使用到某个Bean时(调用getBean()),才对该Bean进行加载实例化,这样,我们就不能发现一些存在的Spring的配置问题。而ApplicationContext则相反,它是在容器启动时,一次性创建了所有的Bean。这样,在容器启动时,我们就可以发现Spring中存在的配置错误。 相对于基本的BeanFactory,ApplicationContext 唯一的不足是占用内存空间。当应用程序配置Bean较多时,程序启动较慢。

BeanFacotry延迟加载,如果Bean的某一个属性没有注入,BeanFacotry加载后,直至第一次使用调用getBean方法才会抛出异常;而ApplicationContext则在初始化自身是检验,这样有利于检查所依赖属性是否注入;所以通常情况下我们选择使用 ApplicationContext。
应用上下文则会在上下文启动后预载入所有的单实例Bean。通过预载入单实例bean ,确保当你需要的时候,你就不用等待,因为它们已经创建好了。

2.BeanFactory和ApplicationContext都支持BeanPostProcessor、BeanFactoryPostProcessor的使用,但两者之间的区别是:BeanFactory需要手动注册,而ApplicationContext则是自动注册。(Applicationcontext比 beanFactory 加入了一些更好使用的功能。而且 beanFactory 的许多功能需要通过编程实现而 Applicationcontext 可以通过配置实现。比如后处理 bean , Applicationcontext 直接配置在配置文件即可而 beanFactory 这要在代码中显示的写出来才可以被容器识别。 )

3.beanFactory主要是面对与 spring 框架的基础设施,面对 spring 自己。而 Applicationcontex 主要面对与 spring 使用的开发者。基本都会使用 Applicationcontex 并非 beanFactory 。

\end{document}





