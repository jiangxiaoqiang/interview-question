
\documentclass[../../../interview-questions.tex]{subfiles}

\begin{document}

\subsection{Spring事务失效场景}

如下场景会导致事务失效:

\begin{enumerate}
    \item {数据库引擎不支持事务}
    \item {没有被Spring管理。}例如如下的代码:
    
\begin{lstlisting}[language=Java]
// @Service
public class OrderServiceImpl implements OrderService {
    @Transactional
    public void updateOrder(Order order) {
        // update order
    }
}
\end{lstlisting}

    如果此时把 @Service 注解注释掉,这个类就不会被加载成一个 Bean,那这个类就不会被 Spring 管理了,事务自然就失效了

    \item {方法不是public的}不是public的方法没有权限访问,自然无法代理,如果能代理,那么就违背了设计初衷。When using proxies, you should apply the @Transactional annotation only to methods with public visibility. If you do annotate protected, private or package-visible methods with the @Transactional annotation, no error is raised, but the annotated method does not exhibit the configured transactional settings. Consider the use of AspectJ (see below) if you need to annotate non-public methods.
    \item {自身调用(Self Invocation)问题}Spring的事务实现是根据JDK的动态代理,如果我们为Spring的某个bean配置了切面,那么Spring在创建这个bean的时候,实际上创建的是这个bean的一个代理对象,我们后续对bean中方法的调用,实际上调用的是代理类重写的代理方法。在类中调用事务注解的方法,事务不会生效。因为事务注解是基于动态代理实现的,通过代理调用的方法才会产生切面通知,而类中的自身调用是通过this方式调用,并不会产生切面通知\footnote{\url{https://blog.csdn.net/weixin_40482816/article/details/116235521}}。
    \item {数据源没有配置事务管理器}
    \item {异常被捕获}如果在执行事务的方法中捕获了异常,没有抛出,那么会导致回滚失效,可能会出现部分成功的情况,也就是事务失效了,没有要么全部成功要么全部失败事务效果。
    \item {异常类型错误。}例如如下代码片段:
    
\begin{lstlisting}[language=Java]
// @Service
public class OrderServiceImpl implements OrderService {
    @Transactional
    public void updateOrder(Order order) {
        try {
            // update order
        } catch {
            throw new Exception("更新错误");
        }
    }  
}
\end{lstlisting}

    这样事务也是不生效的,因为默认回滚的是:RuntimeException,如果你想触发其他异常的回滚,需要在注解上配置一下,如:

\begin{lstlisting}[language=Java]
@Transactional(rollbackFor = Exception.class)
\end{lstlisting}

    \item {方法用final修饰} Spring事务底层使用了AOP,也就是通过JDK动态代理或者CGlib,帮我们生成了代理类,在代理类中实现的事务功能。但如果某个方法用final修饰了,那么在它的代理类中,就无法重写该方法,而添加事务功能。注意:如果某个方法是static的,同样无法通过动态代理,变成事务方法。
    \item {多线程调用}
\end{enumerate}

https://www.cnblogs.com/javastack/p/12160464.html

\end{document}





