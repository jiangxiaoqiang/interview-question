
\documentclass[../../../interview-questions.tex]{subfiles}

\begin{document}

\subsection{Spring事务失效场景}

如下场景会导致事务失效:

\begin{enumerate}
    \item {数据库引擎不支持事务}
    \item {没有被Spring管理}
    \item {方法不是public的}不是public的方法没有权限访问,自然无法代理,如果能代理,那么就违背了设计初衷。
    \item {自身调用(Self Invocation)问题}Spring的事务实现是根据JDK的动态代理,如果我们为Spring的某个bean配置了切面,那么Spring在创建这个bean的时候,实际上创建的是这个bean的一个代理对象,我们后续对bean中方法的调用,实际上调用的是代理类重写的代理方法。在类中调用事务注解的方法,事务不会生效。因为事务注解是基于动态代理实现的,通过代理调用的方法才会产生切面通知,而类中的自身调用是通过this方式调用,并不会产生切面通知\footnote{\url{https://blog.csdn.net/weixin_40482816/article/details/116235521}}。
    \item {数据源没有配置事务管理器}
    \item {异常被捕获}如果在执行事务的方法中捕获了异常,没有抛出,那么会导致回滚失效,可能会出现部分成功的情况,也就是事务失效了,没有要么全部成功要么全部失败事务效果。
    \item {异常类型错误}
\end{enumerate}

https://segmentfault.com/a/1190000040059901

\end{document}





