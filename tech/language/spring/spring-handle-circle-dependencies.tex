\documentclass[../../../interview-questions.tex]{subfiles}

\begin{document}

\subsection{Spring Bean如何解决循环依赖问题}

在 Spring 单例 Bean 的创建 中介绍介绍了使用三级缓存。

singletonObjects: 一级缓存,存储单例对象,Bean 已经实例化,初始化完成。

earlySingletonObjects: 二级缓存,存储 singletonObject,这个 Bean 实例化了,还没有初始化。

singletonFactories: 三级缓存,存储 singletonFactory。


一级缓存singletonObjects是完整的bean,它可以被外界任意使用,并且不会有歧义。

二级缓存earlySingletonObjects是不完整的bean,没有完成初始化,它与singletonObjects的分离主要是职责的分离以及边界划分,可以试想一个Map缓存里既有完整可使用的bean,也有不完整的,只能持有引用的bean,在复杂度很高的架构中,很容易出现歧义,并带来一些不可预知的错误。

三级缓存singletonFactories,其职责就是包装一个bean,有回调逻辑,所以它的作用非常清晰,并且只能处于第三层。

在实际使用中,要获取一个bean,先从一级缓存一直查找到三级缓存,缓存bean的时候是从三级到一级的顺序保存,并且缓存bean的过程中,三个缓存都是互斥的,只会保持bean在一个缓存中,而且,最终都会在一级缓存中。Spring 是如何通过上面介绍的三级缓存来解决循环依赖的呢?这里只用 A,B 形成的循环依赖来举例:

\begin{enumerate}
    \item {实例化 A,此时 A 还未完成属性填充和初始化方法(@PostConstruct)的执行,A 只是一个半成品。}
    \item {为 A 创建一个 Bean 工厂,并放入到三级缓存singletonFactories 中。}
    \item {发现 A 需要注入 B 对象,但是一级、二级、三级缓存均为发现对象 B。}
    \item {实例化 B,此时 B 还未完成属性填充和初始化方法(@PostConstruct)的执行,B 只是一个半成品。}
    \item {为 B 创建一个 Bean 工厂,并放入到三级缓存singletonFactories 中。}
    \item {发现 B 需要注入 A 对象,此时在一级、二级未发现对象 A,但是在三级缓存中发现了对象 A,从三级缓存中得到对象 A,并将对象 A 放入二级缓存中,同时删除三级缓存中的对象 A。(注意,此时的 A 还是一个半成品,并没有完成属性填充和执行初始化方法)}
    \item {将对象 A 注入到对象 B 中。}
    \item {对象 B 完成属性填充,执行初始化方法,并放入到一级缓存中,同时删除二级缓存中的对象 B。(此时对象 B 已经是一个成品)}
    \item {对象 A 得到对象 B,将对象 B 注入到对象 A 中。(对象 A 得到的是一个完整的对象 B)}
    \item {对象 A 完成属性填充,执行初始化方法,并放入到一级缓存中,同时删除二级缓存中的对象 A。}
\end{enumerate}

\paragraph{为什么需要三级缓存}

从软件设计角度考虑,三个缓存代表三种不同的职责,根据单一职责原理,从设计角度就需分离三种职责的缓存,所以形成三级缓存的状态。

\end{document}





