
\documentclass[../../../interview-questions.tex]{subfiles}

\begin{document}

\subsection{Spring事务传播行为(Spring Transation Propagation)}

一个方法执行,在它之间,假如已经创建了事务,那它要创建新事务还是加入原有事务;假如之前没有事务,那它又该如何处理;假如外层事务异常要回滚,内层事务要不要回滚;假如内层事务异常要回滚,外层事务要不要回滚。这就要求必须定义好事务的传播特性,使得程序任一步都知道如何处理事务问题。Propagation是传播的意思。保证同一个事务中
PROPAGATION\_REQUIRED 支持当前事务,如果不存在 就新建一个(默认)
PROPAGATION\_SUPPORTS 支持当前事务,如果不存在,就不使用事务
PROPAGATION\_MANDATORY 支持当前事务,如果不存在,抛出异常
保证没有在同一个事务中
PROPAGATION\_REQUIRES\_NEW 如果有事务存在,挂起当前事务,创建一个新的事务
PROPAGATION\_NOT\_SUPPORTED 以非事务方式运行,如果有事务存在,挂起当前事务
PROPAGATION\_NEVER 以非事务方式运行,如果有事务存在,抛出异常
PROPAGATION\_NESTED 如果当前事务存在,则嵌套事务执行

\begin{enumerate}
\item {\color{blue}{PROPAGATION\_REQUIRES}}	表示当前方法必须在一个事务中运行。如果一个现有事务正在进行中,该方法将在那个事务中运行,否则就要开始一个新事务。
\item{\color{blue}{PROPAGATION\_SUPPORTS}}	表示当前方法不需要事务性上下文,但是如果有一个事务已经在运行的话,它也可以在这个事务里运行。
\item{\color{blue}{PROPAGATION\_MANDATORY}}	表示该方法必须运行在一个事务中。如果当前没有事务正在发生,将抛出一个异常

\item{PROPAGATION\_REQUIRES\_NEW	表示当前方法必须在它自己的事务里运行。一个新的事务将被启动,而且如果有一个现有事务在运行的话,则将在这个方法运行期间被挂起。}
\item{PROPAGATION\_NESTED	表示如果当前正有一个事务在进行中,则该方法应当运行在一个嵌套式事务中。被嵌套的事务可以独立于封装事务进行提交或回滚。如果封装事务不存在,行为就像PROPAGATION\_REQUIRES一样。他会和父事务一起commit,当它回滚时,父事务有条件的选择是否跟随回滚,或者继续执行}
\item{PROPAGATION\_NEVER	表示当前的方法不应该在一个事务中运行。如果一个事务正在进行,则会抛出一个异常。}
\item{PROPAGATION\_NOT\_SUPPORTED	表示该方法不应该在一个事务中运行。如果一个现有事务正在进行中,它将在该方法的运行期间被挂起。}
\end{enumerate}

使用PROPAGATION\_REQUIRES\_NEW,事务和事务之间是隔离开的,内层事务失败不会影响外层事务。

\end{document}





