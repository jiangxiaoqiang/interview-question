\documentclass[../../../interview-questions.tex]{subfiles}

\begin{document}

\subsection{拦截器如何触发}

一直以为拦截器是基于AOP实现的,然而事实并不是。SpringMVC框架的核心类为DispatchServlet(org.springframework.web.servlet),http请求的核心执行方法doService(),我们画下SpringMVC的工作流程图:

\begin{figure}[htbp]
	\centering
	\includegraphics[scale=0.35]{interceptor.png}
	\caption{拦截器执行流程}
	\label{fig:interceptor}
\end{figure}

从流程图可以看到,拦截器的执行是在穿插在SpringMVC的工作流程中的,并没有用到动态代理机制,DispatchServlet的doDispatch方法直接调用的拦截器方法applyPreHandle和applyPostHandle。拦截器在实现层面,并没有用到AOP,并没有切面,通知这一类的代码,所以它的实现并不是基于AOP的。但是拦截器从思想层面上,是面向切面编程的,是在controller这个层面上进行的代码织入。

\end{document}










