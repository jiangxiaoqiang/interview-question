
\documentclass[../../../interview-questions.tex]{subfiles}

\begin{document}

\subsection{微服务负载均衡}

截止2022年,微服务的负载均衡可以选择Spring Cloud Ribbon和Spring Cloud Loadbalance,Ribbon作为Netflix开源的一个组件,Ribbon早已进入维护状态。现在推荐使用的是Loadbalancer,LoadBalancer是Spring Cloud官方提供的负载均衡组件,可用于替代Ribbon。其使用方式与Ribbon基本兼容,可以从Ribbon进行平滑过渡。Spring Cloud Ribbon 是基于​​Netffix Ribbon​​​是实现的一套客户端负载均衡工具,Ribbon客户端提供了一系列的完善的配置,如超时,重试等等,通过​​LoadBalance​​​获取到服务提供的所有机器的实例。Ribbon会自动地基于某种规则​​(轮询,随机)​​ 去调用这些服务,Ribbon也可以自己实现自己的负载均衡算法。

\paragraph{Spring Cloud Ribbon负载均衡策略}

\begin{itemize}
    \item {轮询策略}一个简单的轮询策略实现如下:
    
\begin{lstlisting}[language=Java]
public class RoundRobin<T>
{
    //服务器列表
    private readonly IList<T> _items;
    
    //锁
    private readonly object _syncLock = new object();
    
    //当前访问的服务器索引,开始是-1,因为没有人访问
    private int _currentIndex = -1;

    public RoundRobin(IEnumerable<T> sequence)
    {
        _items = sequence.ToList();

        if(_items.Count <= 0 )
        {
            throw new ArgumentException("Sequence contains no elements.", nameof(sequence));
        }                           
    }

    public T GetNextItem()
    {
        lock (this._syncLock)
        {
            _currentIndex++;
            //超过数量,索引归0
            if (_currentIndex >= _items.Count)
                _currentIndex = 0;
            return _items[_currentIndex];
        }
    }
}    
\end{lstlisting}


    \item {权重策略}
    \item {随机策略}
    \item {最小连接数策略}
    \item {重试策略}RetryRule,按照轮询策略来获取服务,如果获取的服务实例为 null 或已经失效,则在指定的时间之内不断地进行重试来获取服务,如果超过指定时间依然没获取到服务实例则返回 null。
    \item {可用性敏感策略}AvailabilityFilteringRule,先过滤掉非健康的服务实例,然后再选择连接数较小的服务实例。
    \item {区域敏感策略}ZoneAvoidanceRule,根据服务所在区域(zone)的性能和服务的可用性来选择服务实例,在没有区域的环境下,该策略和轮询策略类似。
\end{itemize}

\paragraph{一般负载均衡策略}

负载均衡 Load Balance 是最基本的分流策略, 将负载尽量均匀地分布到下游节点上, 但是如何分派, 是有讲究的。负载均衡大致有以下几种基本策略:

\begin{itemize}
    \item{RANDOM随机以及加权随机}
    \item{ROUND\_ROBIN 均匀分配
    周而复始地循环依次地选择一个个节点来分配负载, 保持每个节点具有相同的负载
    它还有一个加强版, 可以为节点设置不同的权重,为加权轮询法}
    \item {LEAST\_CONNECTIONS 最少连接
    根据客户端建立的连接数选择节点, 连接数量最少的节点将被选中.}
    \item {LEAST\_RESPONE\_TIME 最短响应时间
    根据服务器端对从客户端请求的响应时间长短, 响应最快的节点将被选中.}
    \item {SOURCE\_IP 源地址哈希
    根据客户端的IP地址计算哈希值,然后使用哈希值进行路由, 这可确保即使面对断开的连接,来自同一客户端的请求也始终会转到同一服务器。}
\end{itemize}

https://zhuanlan.zhihu.com/p/171645408

\end{document}








