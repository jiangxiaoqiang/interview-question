\documentclass[../../../interview-questions.tex]{subfiles}

\begin{document}

\subsection{Sring Bean创建过程}

AbstractAutowireCapableBeanFactory继承自AbstractBeanFactory,实现抽象类AbstractBeanFactory创建Bean有3类,Autowire注解的AbstractAutowireCapableBeanFactory、DefaultListableBeanFactory和基于XML方式的XmlBeanFactory。获取参数 name 对应的真正的 beanName
检查缓存或者实例工厂中是否有对应的单例,若存在则进行实例化并返回对象,否则继续往下执行
执行 prototype 类型依赖检查,防止循环依赖
如果当前 beanFactory 中不存在需要的 bean,则尝试从 parentBeanFactory 中获取
将之前解析过程返得到的 GenericBeanDefinition 对象合并为 RootBeanDefinition 对象,便于后续处理
如果存在依赖的 bean,则递归加载依赖的 bean
依据当前 bean 的作用域对 bean 进行实例化
如果对返回 bean 类型有要求,则进行检查,按需做类型转换
返回 bean 实例


Sring Bean创建过程简单分为7个步骤。

\begin{enumerate}
\item{获取beanName}
\item{实例化bean}
\item{单例(Singleton)依赖处理}
\item{属性填充}
\item{原型(Prototype)依赖检查}
\item{注册DisposableBean}
\end{enumerate}


\begin{enumerate}
\item{如果是单例则需要首先清除缓存} 
\item{实例化bean,将BeanDefinition转换为BeanWrapper}
转换是一个复杂的过程,但是我们可以尝试概括大致的功能:

如果存在工厂方法则使用工厂方法进行初始化
一个类有多个构造函数,每个构造函数都有不同的参数,所以需要根据参数锁定构造函数并进行初始化
如果既不存在工厂方法也不存在带有参数的构造函数,则使用默认的构造函数进行 bean 的实例化.首先就是创建一个bean的实例且封装到BeanWrapper中,在这里bean已经实例化了。具体的实现方法是在SimpleInstantiationStrategy类的方法instantiate中。

\item{MergedBeanDefinitionPostProcessor的应用}
bean合并后的处理,Autowired注解正是通过此方法实现诸如类型的预解析。

\item{依赖处理}
在Spring中会有循环依赖的情况,例如,当A中含有B的属性,而B中又含有A的属性时就会构成一个循环依赖,此时如果A和B都是单例,那么在Spring中的处理方式就是当创建B的时候,涉及自动注入A的步骤时,并不是直接去再次创建A,而是通过放入缓存中的ObjectFactory来创建实例,这样就解决了循环依赖的问题。

\item{属性填充。将所有属性填充至bean的实例中}findAutowiringMetadata方法能拿到使用了特定注解的属性(Field)、方法(Method)及依赖的关系保存到checkedElements集合<Set>里,然后再执行自己的inject方法。设置值的关键代码,实质就是通过JDK的反射特性:


\begin{lstlisting}[language=Java]
Field field = (Field) this.member;
ReflectionUtils.makeAccessible(field);
field.set(target, getResourceToInject(target, requestingBeanName));
\end{lstlisting}



\item{循环依赖检查}
Spring 中解决循环依赖只对单例有效,而对于 prototype 的 bean,Spring 没有好的解决办法,唯一要做的就是抛出异常。在这个步骤里面会检测已经加载的 bean 是否已经出现了依赖循环,并判断是否需要抛出异常。

\item{注册DisposableBean}
如果配置了destroy-method,这里需要注册以便于在销毁时候调用。

\item{完成创建并返回}

\end{enumerate}



\end{document}





