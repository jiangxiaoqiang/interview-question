
\documentclass[../../../interview-questions.tex]{subfiles}

\begin{document}

\subsection{Zuul过滤器类型}

外部http请求到达api网关服务的时候,首先它会进入第一个阶段pre,在这里它会被pre类型的过滤器进行处理。该类型过滤器的主要目的是在进行请求路由之前做一些前置加工,比如请求的校验等。在完成了pre类型的过滤器处理之后,请求进入第二个阶段routing,也就是之前说的路由请求转发阶段,请求将会被routing类型的处理器处理。这里的具体处理内容就是将外部请求转发到具体服务实例上去的过程,当服务实例请求结果都返回之后,routing阶段完成,请求进入第三个阶段post。此时请求将会被post类型的过滤器处理,这些过滤器在处理的时候不仅可以获取到请求信息,还能获取到服务实例的返回信息,所以在post类型的过滤器中,我们可以对处理结果进行一些加工或转换等内容。另外,还有一个特殊的阶段error,该阶段只有在上述三个阶段中发生异常的时候才会触发,但是它的最后流向还是post类型的过滤器,因为它需要通过post过滤器将最终结果返回给请求客户端(对于error过滤器的处理,在spring cloud zuul的过滤链中实际上有一些不同)。

Zuul主要有四种类型的过滤器,我们可以为特定的url模式添加任意数量的过滤器。
“pre” 预过滤器 - 在路由分发一个请求之前调用。
“post” 后过滤器 - 在路由分发一个请求后调用。
“route” 路由过滤器 - 用于路由请求分发。
“error” 错误过滤器 - 在处理请求时发生错误时调用

\end{document}





