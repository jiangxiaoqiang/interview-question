
\documentclass[../../../interview-questions.tex]{subfiles}

\begin{document}

\subsection{Spring IoC}

首先我们要知道IoC(Inverse of Control:控制反转)是一种设计思想,就是 将原本在程序中手动创建对象的控制权,交由Spring框架来管理。这并非Spring特有,在其他语言里面也有体现。IOC容器是Spring用来实现IOC的载体, IoC容器实际上就是个Map(key,value),Map 中存放的是各种对象。spring 中有多少种 IOC 容器?BeanFactory - BeanFactory 就像一个包含 bean 集合的工厂类。它会在客户端要求时实例化 bean。
ApplicationContext - ApplicationContext 接口扩展了 BeanFactory 接口。它在 BeanFactory 基础上提供了一些额外的功能。

或许是IoC不够开门见山,Martin Fowler提出了DI(Dependency Injection)来替代IoC(Inverse of Control),即让调用类对某一接口实现类的依赖关系由第三方(容器或协作类)注入,以移除调用类对某一接口实现类的依赖。

所以我们要区别IoC与DI(Dependency Injection),简单来说IOC的主要实现方式有两种:依赖查找和依赖注入。

我们DI就是依赖注入,也就是IOC的一种可取的实现方式!对两个概念总结以下:

IOC (Inversion of control ) 控制反转/反转控制。是站在对象的角度,对象实例化以及管理的权限(反转)交给了容器。

DI (Dependancy Injection)依赖注入。是站在容器的角度,容器会把对象依赖的其他对象注入(送进去)。例如:对象A 实例化过程中因为声明了一个B类型的属性,那么就需要容器把B对象注入到A中。可以通过多少种方式完成依赖注入?
通常,依赖注入可以通过三种方式完成,即:构造函数注入、setter 注入、接口注入。

在 Spring Framework 中,仅使用构造函数和 setter 注入。

通过使用IOC容器可以对我们的对象注入依赖(DI),实现控制反转!

优点:实现组件之间的解耦,提高程序的灵活性和可维护性。
缺点:生成一个对象的步骤变复杂了,生成因为是使用反射编程,在效率上有些损耗。但相对于IoC提高的维护性和灵活性来说,这点损耗是微不足道的,除非某对象的生成对效率要求特别高。

通过上面的介绍,我们大概理解了IOC的概念,也知道它的作用。那么也会有疑惑,为什么需要依赖反转呢,有什么好处,解决了什么问题?

简单来说,IOC 容器就像是一个工厂一样,当我们需要创建一个对象的时候,只需要配置好配置文件/注解即可,完全不用考虑对象是如何被创建出来的。 在实际项目中一个 Service 类可能有几百甚至上千个类作为它的底层,假如我们需要实例化这个 Service,你可能要每次都要搞清这个 Service 所有底层类的构造函数,这可能会把人逼疯。如果利用 IOC 的话,你只需要配置好,然后在需要的地方引用就行了,这大大增加了项目的可维护性且降低了开发难度。

举个例子:现有一个针对User的操作,利用 Service 和 Dao 两层结构进行开发!

在没有使用IOC思想的情况下,Service 层想要使用 Dao层的具体实现的话,需要通过new关键字在UserServiceImpl 中手动 new出 IUserDao 的具体实现类 UserDaoImpl(不能直接new接口类)。

这种方式可以实现,但是如果开发过程中接到新需求,针对IUserDao 接口开发出另一个具体实现类。因为Server层依赖了IUserDao的具体实现,所以我们需要修改UserServiceImpl中new的对象。如果只有一个类引用了IUserDao的具体实现,可能觉得还好,修改起来也不是很费力气,但是如果有许许多多的地方都引用了IUserDao的具体实现的话,一旦需要更换IUserDao的实现方式,那修改起来将会非常的头疼。

\end{document}





