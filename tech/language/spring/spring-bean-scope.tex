\documentclass[../../../interview-questions.tex]{subfiles}

\begin{document}

\subsection{Spring Bean作用域}

当在 Spring 中定义一个 bean 时,你必须声明该 bean 的作用域的选项。例如,为了强制 Spring 在每次需要时都产生一个新的 bean 实例,你应该声明 bean 的作用域的属性为 prototype。同理,如果你想让 Spring 在每次需要时都返回同一个bean实例,你应该声明 bean 的作用域的属性为 singleton。Spring 框架支持以下五个作用域,分别为 singleton、prototype、request、session 和 global session。

\begin{enumerate}
    \item {singleton : 唯一 bean 实例,Spring 中的 bean 默认都是单例的。}
    \item {prototype : 每次请求都会创建一个新的 bean 实例。}
    \item {request : 每一次HTTP请求都会产生一个新的bean,该bean仅在当前HTTP request内有效。}
    \item {session : :在一个HTTP Session中,一个Bean定义对应一个实例。该作用域仅在基于web的Spring ApplicationContext情形下有效。}
    \item {global-session: 全局session作用域,仅仅在基于portlet的web应用中才有意义,Spring5已经没有了。Portlet是能够生成语义代码(例如:HTML)片段的小型Java Web插件。它们基于portlet容器,可以像servlet一样处理HTTP请求。但是,与 servlet 不同,每个 portlet 都有不同的会话}
\end{enumerate}

\end{document}





