\documentclass[../../../interview-questions.tex]{subfiles}

\begin{document}

\subsection{Apollo刷新配置的原理}

原理就是把这些配置都存储起来,当配置发生变化的时候进行修改就可以\footnote{参见:\url{https://www.apolloconfig.com/\#/zh/README}}。Apollo 中定义了一个 SpringValueProcessor 类,用来处理 Spring 中值的修改。通过实现 BeanPostProcessor 来处理每个 bean 中的值,然后将这个配置信息封装成一个 SpringValue 存储到 springValueRegistry 中。Apollo刷新配置的流程是:Portal修改配置,Config配置服务发现更改,通知客户端更新。发现配置中更改是通过线程定时扫描表的方式,通知客户端更新是通过长轮询的方式,还有一个备用机制是定时更新,防止长轮询失效的情况下配置无法更新。定时轮询的代码在com.ctrip.framework.apollo.internals命名空间下的RemoteConfigRepository类schedulePeriodicRefresh方法。再com.ctrip.framework.apollo.internals命名空间的类emoteConfigRepository下还有长轮询方法scheduleLongPollingRefresh。





\end{document}










