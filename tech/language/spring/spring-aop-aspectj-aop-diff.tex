
\documentclass[../../../interview-questions.tex]{subfiles}

\begin{document}

\subsection{Spring AOP 和 AspectJ AOP 有什么区别}

Spring AOP 属于运行时增强,而 AspectJ 是编译时增强。 Spring AOP 基于代理(Proxying),而 AspectJ 基于字节码操作(Bytecode Manipulation)。
Spring AOP 已经集成了 AspectJ  ,AspectJ  应该算的上是 Java 生态系统中最完整的 AOP 框架了。AspectJ  相比于 Spring AOP 功能更加强大,但是 Spring AOP 相对来说更简单,
如果我们的切面比较少,那么两者性能差异不大。但是,当切面太多的话,最好选择 AspectJ ,它比Spring AOP 快很多。


ASM :一个轻量级的字节码操作框架,直接涉及到jvm底层操作和指令,使用难度较大。

CGLIB:属于动态织入(字节码加载之后)技术,基于ASM实现,性能高。同时,CGLIB突破了Java动态代理基于接口的限制,采用子类继承的方式。

JAVAssist:属于动态织入技术,操作简单,接口强大,性能较ASM差。

ASPECTJ:静态织入(字节码加载之前)框架,常用于AOP编程框架。

\end{document}





