\documentclass[../../../interview-questions.tex]{subfiles}

\begin{document}

\subsection{Spring Bean实例化流程}

https://zhuanlan.zhihu.com/p/368720644。Spring IOC容器就好像一个生产产品的流水线上的机器,Spring创建出来的Bean就好像是流水线的终点生产出来的一个个精美绝伦的产品。既然是机器,总要先启动,Spring也不例外。因此Bean的一生从总体上来说可以分为两个阶段:

\begin{itemize}
    \item {容器启动阶段}
    \item {Bean实例化阶段}
\end{itemize}

Spring管理的bean的实例化不是在容器启动的时候做的,容器启动只是把配置的bean从配置文件xml中读到内存里,然后保存成BeanDefinition的形式。实例化开始是当调用getBean(..)的时候才会触发实例化,或者使用ApplicationContext这个容器,因为ApplicationContext在启动的时候会采用非延迟的策略来调用finishBeanFactoryInitialization(beanFactory)来实例化所有注册到容器中的bean,而BeanFactory类型的容器不是getBean的时候一定触发实例化,比如单例类型的bean只在第一次调用getBean触发实例化,后面是从三级缓存中获取的。实例化Spring Bean可以在类AbstractApplicationContext的refresh()方法中查看。实例化 Bean 的大致流程:

遍历所有扫描出来的 BeanDefinition,进行验证:是否 Lazy、是否 prototype、是否 abstract 等等,这一步发生在refresh()方法:

\begin{lstlisting}[language=Java]
// Tell the subclass to refresh the internal bean factory.
ConfigurableListableBeanFactory beanFactory = obtainFreshBeanFactory();  
\end{lstlisting}

对 dependsOn 进行一些特殊处理,递归的实例化依赖的 bean

实例化 Bean,实例化Bean在refresh()方法的代码:

\begin{lstlisting}[language=Java]
// Instantiate all remaining (non-lazy-init) singletons.
finishBeanFactoryInitialization(beanFactory);    
\end{lstlisting}


执行属性注入

执行 Bean 初始化

初始化完成,加入单例池

注册 DisposableBean(销毁 Bean 时回调方法)

\end{document}





