
\documentclass[../../../interview-questions.tex]{subfiles}

\begin{document}

\subsection{Spring 事务中的隔离级别有哪几种}

TransactionDefinition 接口中定义了五个表示隔离级别的常量:

TransactionDefinition.ISOLATION\_DEFAULT:  使用后端数据库默认的隔离级别,Mysql 默认采用的REPEATABLE\_READ隔离级别 Oracle 默认采用的 READ\_COMMITTED隔离级别.
TransactionDefinition.ISOLATION\_READ\_UNCOMMITTED: 最低的隔离级别,允许读取尚未提交的数据变更,可能会导致脏读、幻读或不可重复读
TransactionDefinition.ISOLATION\_READ\_COMMITTED:   允许读取并发事务已经提交的数据,可以阻止脏读,但是幻读或不可重复读仍有可能发生
TransactionDefinition.ISOLATION\_REPEATABLE\_READ:  对同一字段的多次读取结果都是一致的,除非数据是被本身事务自己所修改,可以阻止脏读和不可重复读,但幻读仍有可能发生。
TransactionDefinition.ISOLATION\_SERIALIZABLE:   最高的隔离级别,完全服从ACID的隔离级别。所有的事务依次逐个执行,这样事务之间就完全不可能产生干扰,也就是说,该级别可以防止脏读、不可重复读以及幻读。但是这将严重影响程序的性能。通常情况下也不会用到该级别。

\end{document}





