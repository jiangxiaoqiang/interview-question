\documentclass[../../../interview-questions.tex]{subfiles}

\begin{document}

\subsection{++i和i++区别}

关于这个前加加和后加加,搞不懂为什么要设计前加和后加,也没有找到那种场合必须得用i++才能Hold住局面。一句话,前加加是加了之后在参与运算,后加加是参与运算再加。为啥要这样设计?前加加的C++示例代码:

\begin{lstlisting}[language=C++]
int add(int num) {
    int y = ++num;
    return y;
}
\end{lstlisting}

汇编代码长这样:

\begin{lstlisting}[language={[x86masm]Assembler}]
add(int):
    push    rbp
    mov     rbp, rsp
    ; 
    mov     DWORD PTR [rbp-20], edi
    add     DWORD PTR [rbp-20], 1
    mov     eax, DWORD PTR [rbp-20]
    mov     DWORD PTR [rbp-4], eax
    mov     eax, DWORD PTR [rbp-4]
    pop     rbp
    ret
\end{lstlisting}

比如mov eax, dword ptr [12345678]  把内存地址12345678中的双字型(32位)数据赋给eax。rbp栈基地址寄存器,保存当前帧的栈底地址,32位位ebp。rsp 栈指针寄存器,保存当前栈顶。寄存器ESI、EDI、SI和DI称为变址寄存器(Index Register),它们主要用于存放存储单元在段内的偏移量,用它们可实现多种存储器操作数的寻址方式,为以不同的地址形式访问存储单元提供方便。
变址寄存器不可分割成8位寄存器。作为通用寄存器,也可存储算术逻辑运算的操作数和运算结果。x86的通用寄存器有eax、ebx、ecx、edx、edi、esi。这些寄存器在大多数指令中是可以任意选用的,比如movl指令可以把一个立即数传送到eax中,也可传送到ebx中。但也有一些指令规定只能用其中某个寄存器做某种用途,例如除法指令idivl要求被除数在eax寄存器中,edx寄存器必须是0,而除数可以在任意寄存器中,计算结果的商数保存在eax寄存器中(覆盖原来的被除数),余数保存在edx寄存器中。也就是说,通用寄存器对于某些特殊指令来说也不是通用的。后加加的C++示例代码:

\begin{lstlisting}[language=C++]
int add(int num) {
    int y = num++;
    return y;
}
\end{lstlisting}

汇编代码长这样:

\begin{lstlisting}[language={[x86masm]Assembler}]
add(int):
    push    rbp
    mov     rbp, rsp
    mov     DWORD PTR [rbp-20], edi
    mov     eax, DWORD PTR [rbp-20]
    lea     edx, [rax+1]
    mov     DWORD PTR [rbp-20], edx
    mov     DWORD PTR [rbp-4], eax
    mov     eax, DWORD PTR [rbp-4]
    pop     rbp
    ret
\end{lstlisting}

LEA指令功能:取源操作数地址的偏移量,并把它传送到目的操作数所在的单元。LEA 指令要求原操作数必须是存储单元,而且目的操作数必须是一个除段寄存器之外的16位或32位寄存器。当目的操作数是16位通用寄存器时,那么只装入有效地址的低16位。使用时要注意它与MOV指令的区别,MOV指令传送的一般是源操作数中的内容而不是地址。

\end{document}