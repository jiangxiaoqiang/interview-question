\documentclass[../../../interview-questions.tex]{subfiles}

\begin{document}

\subsection{进程间通信}

\begin{enumerate}
    \item {管道( pipe):管道是一种半双工的通信方式,数据只能单向流动,而且只能在具有亲缘关系的进程间使用。进程的亲缘关系通常是指父子进程关系。}
    \item{信号量( semophore ) : 信号量是一个计数器,可以用来控制多个进程对共享资源的访问。它常作为一种锁机制,防止某进程正在访问共享资源时,其他进程也访问该资源。因此,主要作为进程间以及同一进程内不同线程之间的同步手段。}
    \item{消息队列( message queue ) : 消息队列是由消息的链表,存放在内核中并由消息队列标识符标识。消息队列克服了信号传递信息少、管道只能承载无格式字节流以及缓冲区大小受限等缺点。}
    \item{信号 ( signal ) : 信号是一种比较复杂的通信方式,用于通知接收进程某个事件已经发生。}
    \item{共享内存( shared memory ) :共享内存就是映射一段能被其他进程所访问的内存,这段共享内存由一个进程创建,但多个进程都可以访问。共享内存是最快的 IPC 方式,它是针对其他进程间通信方式运行效率低而专门设计的。它往往与其他通信机制,如信号量,配合使用,来实现进程间的同步和通信。}
    \item{套接字( socket ) : 套解口也是一种进程间通信机制,与其他通信机制不同的是,它可用于不同机器间的进程通信。}
    \end{enumerate}

\end{document}


