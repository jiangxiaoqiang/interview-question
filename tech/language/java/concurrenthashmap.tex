\documentclass[../../../interview-questions.tex]{subfiles}

\begin{document}

\subsection{\color{red}{ConcurrentHashMap}}

\paragraph{ConcurrentHashMap如何实现线程安全的}

ConcurrentHashMap是基于Segment分段锁来实现的,这句话也不能说不对,加个前提条件就是正确的了,ConcurrentHashMap从JDK1.5开始随java.util.concurrent包一起引入JDK中,在JDK8以前,ConcurrentHashMap都是基于Segment分段锁来实现的,在 jdk 1.7 中,ConcurrentHashMap 是由 Segment 数据结构和 HashEntry 数组结构构成。采取分段锁来保证安全性。Segment 是 ReentrantLock 重入锁,在 ConcurrentHashMap 中扮演锁的角色;HashEntry 则用于存储键值对数据。一个 ConcurrentHashMap 里包含一个 Segment 数组,一个 Segment 里包含一个 HashEntry 数组,Segment 的结构和 HashMap 类似,是一个数组和链表结构。在JDK8以后,就换成synchronized和CAS这套实现机制了。其实ConcurrentHashMap保证线程安全主要有三个地方。

\begin{enumerate}
    \item {使用volatile保证当Node中的值变化时对于其他线程是可见的}
    \item {使用table数组的头结点作为synchronized的锁来保证写操作的安全}
    \item {当头结点为null时,使用CAS操作来保证数据能正确的写入。}
\end{enumerate}

ConcurrentHashMap中synchronized只锁定当前链表或红黑二叉树的首节点,只要节点hash不冲突,就不会产生并发。

\paragraph{关于ConcurrentHashMap的key和value不能为null的深层次原因}

原作者认为,在ConcurrentMaps (ConcurrentHashMaps, ConcurrentSkipListMaps)上,不允许null值的出现的主要原因是他可能会在并发的情况下带来难以容忍的二义性。如果在HashMap等非并发容器中,你可以通过contains方法来判断,这个key是究竟不存在,还是本来就是null。但是在并发容器中,如果允许空值的存在的话,你就没法判断真正的情况。用作者的话说就是:在Maps或者Sets集合中允许null值的存在,就是公开邀请错误进入你的程序。而这些错误,只有在发生错误的情况下才能被发现。


\end{document}