\documentclass[../../../interview-questions.tex]{subfiles}

\begin{document}

\subsection{\color{red}{ConcurrentHashMap}}

\paragraph{ConcurrentHashMap如何实现线程安全的}

ConcurrentHashMap是基于Segment分段锁来实现的,这句话也不能说不对,加个前提条件就是正确的了,ConcurrentHashMap从JDK1.5开始随java.util.concurrent包一起引入JDK中,在JDK8以前,ConcurrentHashMap都是基于Segment分段锁来实现的,在JDK8以后,就换成synchronized和CAS这套实现机制了。其实ConcurrentHashMap保证线程安全主要有三个地方。

\begin{enumerate}
    \item {使用volatile保证当Node中的值变化时对于其他线程是可见的}
    \item {使用table数组的头结点作为synchronized的锁来保证写操作的安全}
    \item {当头结点为null时,使用CAS操作来保证数据能正确的写入。}
\end{enumerate}

ConcurrentHashMap中synchronized只锁定当前链表或红黑二叉树的首节点,只要节点hash不冲突,就不会产生并发。

\end{document}