\documentclass[../../../interview-questions.tex]{subfiles}

\begin{document}

\subsection{Java类初始化相关}

\paragraph{Java类初始化顺序}

Java类的初始化顺序记住口诀:先父后子,先静后动,成员变量代码块出现顺序决定初始化顺序,构造函数最后。

\begin{enumerate}
\item{规律一、初始化构造时,先父后子;只有在父类所有都构造完后子类才被初始化}
\item{规律二、类加载先是静态、后非静态、最后是构造函数}静态构造块、静态类属性按出现在类定义里面的先后顺序初始化,同理非静态的也是一样的,只是静态的只在加载字节码时执行一次,不管你new多少次,非静态会在new多少次就执行多少次
\item{规律三、java中的类只有在被用到的时候才会被加载}
\item{规律四、java类只有在类字节码被加载后才可以被构造成对象实例}
\end{enumerate}

先成员字段后构造函数。

"基类静态代码块" -> "基类静态成员字段" -> "派生类静态代码块"
    -> "派生类静态成员字段" -> "基类普通代码块" -> "基类普通成员字段"
    -> "基类构造函数" -> "派生类普通代码块"
    -> "派生类普通成员字段" -> "派生类构造函数";

\paragraph{Java类初始化时机}
    
在虚拟机规范《The Java Virtual Machine Specification, Java SE 8 Edition》中是有严格规定的,虚拟机规范指明 有且只有如下几种情况必须立即对类进行初始化(而这一过程自然发生在加载、验证、准备之后),具体的规定在《The Java Virtual Machine Specification, Java SE 8 Edition》的5.5章节Initialization:

\begin{enumerate}

\item{遇到new、getstatic、putstatic或invokestatic这四条字节码指令时,如果类没有进行过初始化,则需要先对其进行初始化。生成这四条指令的最常见的Java代码场景是:}
使用new关键字实例化对象的时候;
读取或设置一个类的静态字段(被final修饰,已在编译器把结果放入常量池的静态字段除外)的时候;
调用一个类的静态方法的时候。
\item{使用java.lang.reflect包的方法对类进行反射调用的时候,如果类没有进行过初始化,则需要先触发其初始化。}

\item{当初始化一个类的时候,如果发现其父类还没有进行过初始化,则需要先触发其父类的初始化。}

\item{当虚拟机启动时,用户需要指定一个要执行的主类(包含main()方法的那个类),虚拟机会先初始化这个主类。}

\item{如果一个java.lang.invoke.MethodHandle实例最后的解析结果REF\_getstatic,REF\_putstatic,REF\_invokeStatic的方法句柄,并且这个方法句柄所对应的类没有进行初始化,则需要先出触发其初始化。}

\item{If C is an interface that declares a non-abstract, non-static method, the initialization of a class that implements C directly or indirectly.} 

注意,对于这几种会触发类进行初始化的场景,虚拟机规范中使用了一个很强烈的限定语:“有且只有”,这五种场景中的行为称为对一个类进行主动引用。除此之外,所有引用类的方式,都不会触发初始化,称为被动引用。

\end{enumerate}

\end{document}