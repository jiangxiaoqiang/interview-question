\documentclass[../../../interview-questions.tex]{subfiles}

\begin{document}

\subsection{Java类初始化顺序}

记住定理:

\begin{enumerate}
\item{规律一、初始化构造时,先父后子;只有在父类所有都构造完后子类才被初始化}
\item{规律二、类加载先是静态、后非静态、最后是构造函数}静态构造块、静态类属性按出现在类定义里面的先后顺序初始化,同理非静态的也是一样的,只是静态的只在加载字节码时执行一次,不管你new多少次,非静态会在new多少次就执行多少次
\item{规律三、java中的类只有在被用到的时候才会被加载}
\item{规律四、java类只有在类字节码被加载后才可以被构造成对象实例}
\end{enumerate}


"基类静态代码块" -> "基类静态成员字段" -> "派生类静态代码块"
    -> "派生类静态成员字段" -> "基类普通代码块" -> "基类普通成员字段"
    -> "基类构造函数" -> "派生类普通代码块"
    -> "派生类普通成员字段" -> "派生类构造函数";
    
\end{document}