\documentclass[../../../interview-questions.tex]{subfiles}

\begin{document}

\subsection{线程池类型}

Java通过Executors提供了四种线程池,单线程化线程池(newSingleThreadExecutor)、可控最大并发数线程池(newFixedThreadPool)、可回收缓存线程池(newCachedThreadPool)、支持定时与周期性任务的线程池(newScheduledThreadPool)、newWorkStealingPool。

\paragraph{newCachedThreadPool}

创建一个可缓存线程池,如果线程池长度超过处理需要,可灵活回收空闲线程,若无可回收,则新建线程。这种类型的线程池特点是:

工作线程的创建数量几乎没有限制(其实也有限制的,数目为Interger. MAX\_VALUE), 这样可灵活的往线程池中添加线程。

如果长时间没有往线程池中提交任务,即如果工作线程空闲了指定的时间(默认为1分钟),则该工作线程将自动终止。终止后,如果你又提交了新的任务,则线程池重新创建一个工作线程。

在使用CachedThreadPool时,一定要注意控制任务的数量,否则,由于大量线程同时运行,很有会造成系统OOM。

\paragraph{newFixedThreadPool}

创建一个指定工作线程数量的线程池。每当提交一个任务就创建一个工作线程,如果工作线程数量达到线程池初始的最大数,则将提交的任务存入到池队列中。

FixedThreadPool是一个典型且优秀的线程池,它具有线程池提高程序效率和节省创建线程时所耗的开销的优点。但是,在线程池空闲时,即线程池中没有可运行任务时,它不会释放工作线程,还会占用一定的系统资源。

\paragraph{newSingleThreadExecutor}

创建一个单线程化的Executor,即只创建唯一的工作者线程来执行任务,它只会用唯一的工作线程来执行任务,保证所有任务按照指定顺序(FIFO, LIFO, 优先级)执行。如果这个线程异常结束,会有另一个取代它,保证顺序执行。单工作线程最大的特点是可保证顺序地执行各个任务,并且在任意给定的时间不会有多个线程是活动的。

\paragraph{newScheduleThreadPool}

创建一个定长的线程池,而且支持定时的以及周期性的任务执行,支持定时及周期性任务执行。

\paragraph{newSingleThreadScheduledExecutor}

创建一个单线程执行程序,它可安排在给定延迟后运行命令或者定期地执行。线程池中最多执行1个线程,之后提交的线程活动将会排在队列中以此执行并且可定时或者延迟执行线程活动。

\paragraph{newWorkStealingPool}工作窃取核心思想是,自己的活干完了去看看别人有没有没干完的活,如果有就拿过来帮他干。大多数实现机制是:为每个工作者程分配一个双端队列(本地队列)用于存放需要执行的任务,当自己的队列没有数据的时候从其它工作者队列中获得一个任务继续执行。newWorkStealingPool 是Java8添加的线程池。和别的4种不同,它用的是ForkJoinPool。使用ForkJoinPool的好处是,把1个任务拆分成多个“小任务”,把这些“小任务”分发到多个线程上执行。这些“小任务”都执行完成后,再将结果合并。之前的线程池中,多个线程共有一个阻塞队列,而newWorkStealingPool 中每一个线程都有一个自己的队列。当线程发现自己的队列没有任务了,就会到别的线程的队列里获取任务执行。可以简单理解为”窃取“。一般是自己的本地队列采取LIFO(后进先出),窃取时采用FIFO(先进先出),一个从头开始执行,一个从尾部开始执行,由于偷取的动作十分快速,会大量降低这种冲突,也是一种优化方式。

https://www.cnblogs.com/gaopengpy/p/12148971.html

\end{document}