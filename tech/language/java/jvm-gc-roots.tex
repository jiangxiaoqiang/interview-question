\documentclass[../../../interview-questions.tex]{subfiles}

\begin{document}

\subsection{哪些对象可以作为Java GC roots对象}

JVM判断对象存活的算法有引用计数法(Reference Counting)和可达性算法(Reachability Analysis),所有的可达性算法都会有起点,那么这个起点就是GC Root。在Java中,可作为GC Roots对象的列表:

1.虚拟机栈(栈帧中的本地变量表)中引用的对象\footnote{\url{https://blog.csdn.net/u010798968/article/details/72835255}};

2.方法区中的类静态属性引用的对象;

3.方法区中常量引用的对象;

4.本地方法栈中JNI(即一般说的Native方法)中引用的对象

GC管理的主要区域是Java堆,一般情况下只针对堆进行垃圾回收。方法区、栈和本地方法区不被GC所管理,因而选择这些区域内的对象作为GC roots,被GC roots引用的对象不被GC回收。

\end{document}