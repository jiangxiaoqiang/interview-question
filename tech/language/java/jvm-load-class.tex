\documentclass[../../../interview-questions.tex]{subfiles}

\begin{document}

\subsection{\color{red}{Java虚拟机如何加载类}}

首先可以暂时不关注细节,避免沉在细节的海洋,从整体上理解加载3大步骤,从 class 文件到内存中的类,按先后顺序需要经过加载(Loading)、链接(Linking)以及初始化(Initialization)三大步骤。其中验证(Verification)、准备(Preparation)、解析(Resolution)3个部分统称为链接(Linking)。加载完毕后,后续的的阶段就是使用(Using)和卸载(Unloading)类。哪些类型需要加载?Java 语言的类型可以分为两大类:基本类型(primitive types)和引用类型(reference types)。Java 的基本类型,它们是由 Java 虚拟机预先定义好的。所以不存在加载基本类型一说。另一大类引用类型,Java 将其细分为四种:类(Class)、接口(Interface)、数组类和泛型参数。由于泛型参数会在编译过程中被擦除,因此 Java 虚拟机加载类实际上只有前三种,即类(Class)、接口(Interface)、数组。在类、接口和数组类中,数组类是由 Java 虚拟机直接生成的,其他两种则有对应的字节流。

\paragraph{加载(Loading)}

加载,是指查找字节流,并且据此创建类的过程。前面提到,对于数组类来说,它并没有对应的字节流,而是由 Java 虚拟机直接生成的。对于其他的类来说,Java 虚拟机则需要借助类加载器来完成查找字节流的过程。

以盖房子为例,村里的 Tony 要盖个房子,那么按照流程他得先找个建筑师,跟他说想要设计一个房型,比如说“一房、一厅、四卫”。你或许已经听出来了,这里的房型相当于类,而建筑师,就相当于类加载器。

村里有许多建筑师,他们等级森严,但有着共同的祖师爷,叫启动类加载器(bootstrap class loader)。启动类加载器是由 C++ 实现的,没有对应的 Java 对象,因此在 Java 中只能用 null 来指代。换句话说,祖师爷不喜欢像 Tony 这样的小角色来打扰他,所以谁也没有祖师爷的联系方式。

除了启动类加载器之外,其他的类加载器都是 java.lang.ClassLoader 的子类,因此有对应的 Java 对象。这些类加载器需要先由另一个类加载器,比如说启动类加载器,加载至 Java 虚拟机中,方能执行类加载。

村里的建筑师有一个潜规则,就是接到单子自己不能着手干,得先给师傅过过目。师傅不接手的情况下,才能自己来。在 Java 虚拟机中,这个潜规则有个特别的名字,叫双亲委派模型(Parents Delegation Model)。每当一个类加载器接收到加载请求时,它会先将请求转发给父类加载器。在父类加载器没有找到所请求的类的情况下,该类加载器才会尝试去加载。

在 Java 9 之前,启动类加载器负责加载最为基础、最为重要的类,比如存放在 JRE 的 lib 目录下 jar 包中的类(以及由虚拟机参数 -Xbootclasspath 指定的类)。除了启动类加载器之外,另外两个重要的类加载器是扩展类加载器(extension class loader)和应用类加载器(application class loader),均由 Java 核心类库提供。

扩展类加载器的父类加载器是启动类加载器。它负责加载相对次要、但又通用的类,比如存放在 JRE 的 lib/ext 目录下 jar 包中的类(以及由系统变量 java.ext.dirs 指定的类)。

应用类加载器的父类加载器则是扩展类加载器。它负责加载应用程序路径下的类。(这里的应用程序路径,便是指虚拟机参数 -cp/-classpath、系统变量 java.class.path 或环境变量 CLASSPATH 所指定的路径。)默认情况下,应用程序中包含的类便是由应用类加载器加载的。

Java 9 引入了模块系统,并且略微更改了上述的类加载器1。扩展类加载器被改名为平台类加载器(platform class loader)。Java SE 中除了少数几个关键模块,比如说 java.base 是由启动类加载器加载之外,其他的模块均由平台类加载器所加载。

除了由 Java 核心类库提供的类加载器外,我们还可以加入自定义的类加载器,来实现特殊的加载方式。举例来说,我们可以对 class 文件进行加密,加载时再利用自定义的类加载器对其解密。

除了加载功能之外,类加载器还提供了命名空间的作用。这个很好理解,打个比方,咱们这个村不讲究版权,如果你剽窃了另一个建筑师的设计作品,那么只要你标上自己的名字,这两个房型就是不同的。

在 Java 虚拟机中,类的唯一性是由类加载器实例以及类的全名一同确定的。即便是同一串字节流,经由不同的类加载器加载,也会得到两个不同的类。在大型应用中,我们往往借助这一特性,来运行同一个类的不同版本。

\paragraph{链接(Linking)}

链接,是指将创建成的类合并至 Java 虚拟机中,使之能够执行的过程。它可分为验证(Verification)、准备(Preparation)、解析(Resolution)三个阶段。

验证阶段的目的,在于确保被加载类能够满足 Java 虚拟机的约束条件。这就好比 Tony 需要将设计好的房型提交给市政部门审核。只有当审核通过,才能继续下面的建造工作。

通常而言,Java 编译器生成的类文件必然满足 Java 虚拟机的约束条件。因此,这部分我留到讲解字节码注入时再详细介绍。

准备阶段的目的,则是为被加载类的静态字段分配内存,并将其初始化为默认值。(int 0,boolean false)。Java 代码中对静态字段的具体初始化,则会在稍后的初始化阶段中进行。过了这个阶段,咱们算是盖好了毛坯房。虽然结构已经完整,但是在没有装修之前是不能住人的。

除了分配内存外,部分 Java 虚拟机还会在此阶段构造其他跟类层次相关的数据结构,比如说用来实现虚方法的动态绑定的方法表。

在 class 文件被加载至 Java 虚拟机之前,这个类无法知道其他类及其方法、字段所对应的具体地址,甚至不知道自己方法、字段的地址。因此,每当需要引用这些成员时,Java 编译器会生成一个符号引用。在运行阶段,这个符号引用一般都能够无歧义地定位到具体目标上。

举例来说,对于一个方法调用,编译器会生成一个包含目标方法所在类的名字、目标方法的名字、接收参数类型以及返回值类型的符号引用,来指代所要调用的方法。

解析阶段的目的,正是将这些符号引用解析成为实际引用。如果符号引用指向一个未被加载的类,或者未被加载类的字段或方法,那么解析将触发这个类的加载(但未必触发这个类的链接以及初始化。)

如果将这段话放在盖房子的语境下,那么符号引用就好比“Tony 的房子”这种说法,不管它存在不存在,我们都可以用这种说法来指代 Tony 的房子。实际引用则好比实际的通讯地址,如果我们想要与 Tony 通信,则需要启动盖房子的过程。

Java 虚拟机规范并没有要求在链接过程中完成解析。它仅规定了:如果某些字节码使用了符号引用,那么在执行这些字节码之前,需要完成对这些符号引用的解析。

\paragraph{初始化(Initialization)}

在 Java 代码中,如果要初始化一个静态字段,我们可以在声明时直接赋值,也可以在静态代码块中对其赋值。

如果直接赋值的静态字段被 final 所修饰,并且它的类型是基本类型或字符串时,那么该字段便会被 Java 编译器标记成常量值(ConstantValue),其初始化直接由 Java 虚拟机完成。除此之外的直接赋值操作,以及所有静态代码块中的代码,则会被 Java 编译器置于同一方法中,并把它命名为 < clinit >。

类加载的最后一步是初始化,便是为标记为常量值的字段赋值,以及执行 < clinit > 方法的过程。Java 虚拟机会通过加锁来确保类的 < clinit > 方法仅被执行一次。

只有当初始化完成之后,类才正式成为可执行的状态。这放在我们盖房子的例子中就是,只有当房子装修过后,Tony 才能真正地住进去。

那么,类的初始化何时会被触发呢?JVM 规范枚举了下述多种触发情况:

当虚拟机启动时,初始化用户指定的主类;

当遇到用以新建目标类实例的 new 指令时,初始化 new 指令的目标类;

当遇到调用静态方法的指令时,初始化该静态方法所在的类;

当遇到访问静态字段的指令时,初始化该静态字段所在的类;

子类的初始化会触发父类的初始化;

如果一个接口定义了 default 方法,那么直接实现或者间接实现该接口的类的初始化,会触发该接口的初始化;

使用反射 API 对某个类进行反射调用时,初始化这个类;

当初次调用 MethodHandle 实例时,初始化该 MethodHandle 指向的方法所在的类。

\begin{lstlisting}[language=Java]
public class Singleton {
    private Singleton() {}
    private static class LazyHolder {
        static final Singleton INSTANCE = new Singleton();
    }
    public static Singleton getInstance() {
        return LazyHolder.INSTANCE;
    }
}
\end{lstlisting}

我在文章中贴了一段代码,这段代码是在著名的单例延迟初始化例子中2,只有当调用 Singleton.getInstance 时,程序才会访问 LazyHolder.INSTANCE,才会触发对 LazyHolder 的初始化(对应第 4 种情况),继而新建一个 Singleton 的实例。

由于类初始化是线程安全的,并且仅被执行一次,因此程序可以确保多线程环境下有且仅有一个 Singleton 实例。

\end{document}