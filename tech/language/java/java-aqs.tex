\documentclass[../../../interview-questions.tex]{subfiles}

\begin{document}

\subsection{Reentrant Lock以及AQS实现原理}

队列同步器Abstract Queued Synchronizer(后面简称AQS)是实现锁和有关同步器的一个基础框架。
在JDK5中,Doug Lea在并发包中加入了大量的同步工具,例如重入锁(ReentrantLock)、读写锁(ReentrantReadWriteLock)、信号量(Semaphore)、CountDownLatch等,都是基于AQS的。
其内部通过一个被标识为volatile的名为state的变量来控制多个线程之间的同步状态。多个线程之间可以通过AQS来独占式或共享式的抢占资源。AQS核心思想是,如果被请求的共享资源空闲,那么就将当前请求资源的线程设置为有效的工作线程,将共享资源设置为锁定状态;如果共享资源被占用,就需要一定的阻塞等待唤醒机制来保证锁分配。这个机制主要用的是CLH队列的变体实现的,将暂时获取不到锁的线程加入到队列中。

CLH:Craig、Landin and Hagersten队列,是单向链表,AQS中的队列是CLH变体的虚拟双向队列(FIFO),AQS是通过将每条请求共享资源的线程封装成一个节点来实现锁的分配。AQS使用一个volatile的int类型的成员变量来表示同步状态,通过内置的FIFO队列来完成资源获取的排队工作,通过CAS完成对State值的修改。

ReentrantLock是可重入锁,什么是可重入锁呢?可重入锁就是当前持有该锁的线程能够多次获取该锁,无需等待。可重入锁是如何实现的呢?当某一线程获取锁后,将state值+1,并记录下当前持有锁的线程,再有线程来获取锁时,判断这个线程与持有锁的线程是否是同一个线程,如果是,将state值再+1,如果不是,阻塞线程。
当线程释放锁时,将state值-1,当state值减为0时,表示当前线程彻底释放了锁,然后将记录当前持有锁的线程的那个字段设置为null,并唤醒其他线程,使其重新竞争锁。这要从ReentrantLock的一个内部类Sync的父类说起,Sync的父类是AbstractQueuedSynchronizer(后面简称AQS)。

AQS(Abstract Queued Synchronizer)是JDK1.5提供的一个基于FIFO等待队列实现的一个用于实现同步器的基础框架,这个基础框架的重要性可以这么说,JCU(Java Conccurrent Utils)包里面几乎所有的有关锁、多线程并发以及线程同步器等重要组件的实现都是基于AQS这个框架。AQS的核心思想是基于volatile int state这样的一个属性同时配合Unsafe工具对其原子性的操作来实现对当前锁的状态进行修改。当state的值为0的时候,标识改Lock不被任何线程所占有。AQS(AbstractQueuedSynchronizer)是java.util.concurrent的基础。也是Doug Lea大神为广大java开发作出的卓越贡献。J.U.C中的工具类如Semaphore、CountDownLatch、ReentrantLock、ReentrantReadWriteLock等都极大程度依赖了AQS.ReentrantLock是AQS的独占锁,Semaphore是AQS共享锁。

当有多个线程去竞争同一个锁的时候,假设锁被某个线程占用,那么如果有成千上万个线程在等待锁,有一种做法是同时唤醒这成千上万个线程去去竞争锁,这个时候就发生了羊群效应,海量的竞争必然造成资源的剧增和浪费,因此终究只能有一个线程竞争成功,其他线程还是要老老实实的回去等待。AQS的FIFO的等待队列给解决在锁竞争方面的羊群效应问题提供了一个思路:保持一个FIFO队列,队列每个节点只关心其前一个节点的状态,线程唤醒也只唤醒队头等待线程。Java中ThreadPoolExecutor到Worker继承了AbstractQueuedSynchronizer,并实现了Runnable。实际上取任务的过程也是一个AQS队列。

\end{document}