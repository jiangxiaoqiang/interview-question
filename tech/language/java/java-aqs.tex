\documentclass[../../../interview-questions.tex]{subfiles}

\begin{document}

\subsection{ReentrantLock以及AQS实现原理}

ReentrantLock是可重入锁,什么是可重入锁呢?可重入锁就是当前持有该锁的线程能够多次获取该锁,无需等待。可重入锁是如何实现的呢?这要从ReentrantLock的一个内部类Sync的父类说起,Sync的父类是AbstractQueuedSynchronizer(后面简称AQS)。

AQS(Abstract Queued Synchronizer)是JDK1.5提供的一个基于FIFO等待队列实现的一个用于实现同步器的基础框架,这个基础框架的重要性可以这么说,JCU(Java Conccurrent Utils)包里面几乎所有的有关锁、多线程并发以及线程同步器等重要组件的实现都是基于AQS这个框架。AQS的核心思想是基于volatile int state这样的一个属性同时配合Unsafe工具对其原子性的操作来实现对当前锁的状态进行修改。当state的值为0的时候,标识改Lock不被任何线程所占有。AQS(AbstractQueuedSynchronizer)是java.util.concurrent的基础。也是Doug Lea大神为广大java开发作出的卓越贡献。J.U.C中的工具类如Semaphore、CountDownLatch、ReentrantLock、ReentrantReadWriteLock等都极大程度依赖了AQS.ReentrantLock是AQS的独占锁,Semaphore是AQS共享锁。

当有多个线程去竞争同一个锁的时候,假设锁被某个线程占用,那么如果有成千上万个线程在等待锁,有一种做法是同时唤醒这成千上万个线程去去竞争锁,这个时候就发生了羊群效应,海量的竞争必然造成资源的剧增和浪费,因此终究只能有一个线程竞争成功,其他线程还是要老老实实的回去等待。AQS的FIFO的等待队列给解决在锁竞争方面的羊群效应问题提供了一个思路:保持一个FIFO队列,队列每个节点只关心其前一个节点的状态,线程唤醒也只唤醒队头等待线程。

\end{document}