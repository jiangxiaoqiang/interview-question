\documentclass[../../../interview-questions.tex]{subfiles}

\begin{document}

\subsection{MyBatis的一级缓存二级缓存}

一级缓存是SqlSession级别的缓存。在操作数据库时需要构造 sqlSession对象,在对象中有一个(内存区域)数据结构(HashMap)用于存储缓存数据。不同的sqlSession之间的缓存数据区域(HashMap)是互相不影响的。一级缓存,又叫本地缓存,是PerpetualCache类型的永久缓存,保存在执行器中(BaseExecutor),而执行器又在SqlSession(DefaultSqlSession)中,所以一级缓存的生命周期与SqlSession是相同的。

一级缓存的作用域是同一个SqlSession,在同一个sqlSession中两次执行相同的sql语句,第一次执行完毕会将数据库中查询的数据写到缓存(内存),第二次会从缓存中获取数据将不再从数据库查询,从而提高查询效率。当一个sqlSession结束后该sqlSession中的一级缓存也就不存在了。Mybatis默认开启一级缓存。

二级缓存是mapper级别的缓存,又叫自定义缓存,实现了Cache接口的类都可以作为二级缓存,所以可配置如encache等的第三方缓存。多个SqlSession去操作同一个Mapper的sql语句,多个SqlSession去操作数据库得到数据会存在二级缓存区域,多个SqlSession可以共用二级缓存,二级缓存是跨SqlSession的。又叫自定义缓存,实现了Cache接口的类都可以作为二级缓存,所以可配置如encache等的第三方缓存。

二级缓存是多个SqlSession共享的,其作用域是mapper的同一个namespace,不同的sqlSession两次执行相同namespace下的sql语句且向sql中传递参数也相同即最终执行相同的sql语句,第一次执行完毕会将数据库中查询的数据写到缓存(内存),第二次会从缓存中获取数据将不再从数据库查询,从而提高查询效率。Mybatis默认没有开启二级缓存需要在setting全局参数中配置开启二级缓存。所有的缓存对象的操作与维护都是由Executor器执行来完成的,一级缓存由BaseExecutor(包含SimpleExecutor、ReuseExecutor、BatchExecutor三个子类)负责维护,二级缓存由CachingExecutor负责维护。因此需要注意的是:配置了二级缓存不代表mybatis就会使用二级缓存,还需要确保在创建SqlSession的过程中,mybatis创建是CachingExecutor类型的执行器\footnote{\url{https://my.oschina.net/lixin91/blog/620068}}。CachingExecutor,缓存执行器。可以把它理解为一个装饰器,主要负责二级缓存的操作和维护。


\end{document}

