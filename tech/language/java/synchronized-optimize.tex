\documentclass[../../../interview-questions.tex]{subfiles}

\begin{document}

\subsection{Synchronized优化}

早期,Synchronized属于重量级锁(Heavyweight Lock),效率低下,因为监视器锁(monitor)是依赖于底层的操作系统的Mutex Lock来实现的,而操作系统实现线程之间的切换时需要从用户态(User Model)转换到核心态(Kernel Model)\footnote{从用户态到内核态切换过程中,Linux主要做的事:

1:读取tr寄存器,访问TSS段

2:从TSS段中的sp0获取进程内核栈的栈顶指针

3:  由控制单元在内核栈中保存当前eflags,cs,ss,eip,esp寄存器的值。

4:由SAVE\_ALL保存其寄存器的值到内核栈

5:把内核代码选择符写入CS寄存器,内核栈指针写入ESP寄存器,把内核入口点的线性地址写入EIP寄存器},这个状态之间的转换需要相对比较长的时间\footnote{\url{https://www.cnblogs.com/justcxtoworld/p/3155741.html}},时间成本相对较高,这也是为什么早期的synchronized效率低的原因。庆幸的是在Java 6之后Java官方对从JVM层面对synchronized较大优化,所以现在的synchronized锁效率也优化得很不错了,Java 6之后,为了减少获得锁和释放锁所带来的性能消耗(阻塞或唤醒一个Java线程需要操作系统切换CPU状态来完成,这种状态转换需要耗费处理器时间。如果同步代码块中的内容过于简单,状态转换消耗的时间有可能比用户代码执行的时间还要长),引入了偏向锁、轻量级锁和自旋锁等概念,接下来我们将简单了解一下Java官方在JVM层面对Synchronized锁的优化。

\paragraph{偏向锁(Biased Locking)}偏向锁是指一段同步代码一直被一个线程所访问,那么该线程会自动获取锁,降低获取锁的代价。当一个线程访问同步代码块并获取锁时,会在Mark Word里存储锁偏向的线程ID。在线程进入和退出同步块时不再通过CAS操作来加锁和解锁,而是检测Mark Word里是否存储着指向当前线程的偏向锁。引入偏向锁是为了在无多线程竞争的情况下尽量减少不必要的轻量级锁执行路径,因为轻量级锁的获取及释放依赖多次CAS原子指令,而偏向锁只需要在置换ThreadID的时候依赖一次CAS原子指令即可。引入偏向锁的主要原因是,经过研究发现,在大多数情况下,锁不仅不存在多线程竞争,而且总是由同一线程多次获得,因此为了减少同一线程获取锁的代价而引入偏向锁。但是对于锁竞争比较激烈的场合,偏向锁就失效了,因为这样场合极有可能每次申请锁的线程都是不相同的,因此这种场合下不应该使用偏向锁,否则会得不偿失,需要注意的是,偏向锁失败后,并不会立即膨胀为重量级锁,而是先升级为轻量级锁\footnote{\url{http://bigdatadecode.club/JavaSynchronizedTheory.html}}。

\paragraph{轻量级锁(Lightweight Locks)}
引入轻量级锁的主要目的是,在没有多线程竞争的前提下,减少传统的重量级锁使用操作系统互斥量产生的性能消耗(多指时间消耗)。

\paragraph{自旋锁(Spin Lock)}所谓自旋锁,就是让该线程等待一段时间,不会被立即挂起,看持有锁的线程是否会很快释放锁。怎么等待呢?执行一段无意义的循环即可(自旋)。
自旋等待不能替代阻塞,虽然它可以避免线程切换带来的开销,但是它占用了处理器的时间。如果持有锁的线程很快就释放了锁,那么自旋的效率就非常好,反之,自旋的线程就会白白消耗掉处理的资源,它不会做任何有意义的工作,这样反而会带来性能上的浪费。所以说,自旋等待的时间(自旋的次数)必须要有一个限度,如果自旋超过了定义的时间仍然没有获取到锁,则应该被挂起。


\end{document}