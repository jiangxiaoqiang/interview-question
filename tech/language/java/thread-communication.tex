\documentclass[../../../interview-questions.tex]{subfiles}

\begin{document}

\subsection{线程间通信}

\textbf{锁机制:包括互斥锁、条件变量、读写锁}互斥锁提供了以排他方式防止数据结构被并发修改的方法。 读写锁允许多个线程同时读共享数据,而对写操作是互斥的。 条件变量可以以原子的方式阻塞进程,直到某个特定条件为真为止。对条件的测试是在互斥锁的保护下进行的。条件变量始终与互斥锁一起使用。

\textbf{wait/notify 等待}等待通知机制是基于wait和notify方法来实现的,在一个线程内调用该线程锁对象的wait方法,线程将进入等待队列进行等待直到被通知或者被唤醒。

\textbf{Volatile 内存共享}volatile有两大特性,一是可见性,二是有序性,禁止指令重排序,其中可见性就是可以让线程之间进行通信。
volatile语义保证线程可见性有两个原则保证
所有volatile修饰的变量一旦被某个线程更改,必须立即刷新到主内存
所有volatile修饰的变量在使用之前必须重新读取主内存的值


CountDownLatch 并发工具

CyclicBarrier 并发工具

信号量机制(Semaphore)

信号机制(Signal)

\end{document}


