\documentclass[../../../interview-questions.tex]{subfiles}

\begin{document}

\subsection{线程间通信}

\textbf{锁机制:包括互斥锁、条件变量、读写锁}互斥锁提供了以排他方式防止数据结构被并发修改的方法。 读写锁允许多个线程同时读共享数据,而对写操作是互斥的。 条件变量可以以原子的方式阻塞进程,直到某个特定条件为真为止。对条件的测试是在互斥锁的保护下进行的。条件变量始终与互斥锁一起使用。

\textbf{wait/notify 等待}等待通知机制是基于wait和notify方法来实现的,在一个线程内调用该线程锁对象的wait方法,线程将进入等待队列进行等待直到被通知或者被唤醒。

\textbf{Volatile 内存共享}volatile有两大特性,一是可见性,二是有序性,禁止指令重排序,其中可见性就是可以让线程之间进行通信。
volatile语义保证线程可见性有两个原则保证
所有volatile修饰的变量一旦被某个线程更改,必须立即刷新到主内存
所有volatile修饰的变量在使用之前必须重新读取主内存的值

\textbf{CountDownLatch 并发工具。}CountDownLatch 能够使一个线程在等待另外一些线程完成各自工作之后,再继续执行。 它相当于是一个计数器,这个计数器的初始值就是线程的数量,每当一个任务完成后,计数器的值就会减一,当计数器的值为0 时,表示所有的线程都已经任务了,然后在CountDownLatch 上等待的线程就可以恢复执行接下来的任务。

\textbf{CyclicBarrier 并发工具。}从字面上的意思可以知道,这个类的中文意思是“循环栅栏”。大概的意思就是一个可循环利用的屏障。它的作用就是会让所有线程都等待完成后才会继续下一步行动。

举个例子,就像生活中我们会约朋友们到某个餐厅一起吃饭,有些朋友可能会早到,有些朋友可能会晚到,但是这个餐厅规定必须等到所有人到齐之后才会让我们进去。这里的朋友们就是各个线程,餐厅就是 CyclicBarrier。

\textbf{信号量机制(Semaphore)。}一个计数信号量。在概念上,信号量维持一组许可证。如果有必要,每个acquire()都会阻塞,直到许可证可用,然后才能使用它。每个release()添加许可证,潜在地释放阻塞获取方。但是,没有使用实际的许可证对象; Semaphore只保留可用数量的计数,并相应地执行。
信号量通常用于限制线程数,而不是访问某些(物理或逻辑)资源。


信号机制(Signal)

\end{document}


