\documentclass[../../../interview-questions.tex]{subfiles}

\begin{document}

\subsection{线程间通信}

https://zhuanlan.zhihu.com/p/138689342线程间通信3种思路,共享内存、消息传递和管道流。消息传递可用管道输入输出流。管道输入/输出流和普通的文件输入/输出流或者网络输入/输出流不同之处在于,它主要 用于线程之间的数据传输,而传输的媒介为内存。管道输入/输出流主要包括了如下4种具体实现:PipedOutputStream、PipedInputStream、 PipedReader和PipedWriter,前两种面向字节,而后两种面向字符。

\textbf{内存共享}volatile有两大特性,一是可见性,二是有序性,禁止指令重排序,其中可见性就是可以让线程之间进行通信。
volatile语义保证线程可见性有两个原则保证
所有volatile修饰的变量一旦被某个线程更改,必须立即刷新到主内存
所有volatile修饰的变量在使用之前必须重新读取主内存的值

\textbf{消息传递}线程可以通过发送消息来进行通信。消息可以是基于共享内存的,也可以使用专用的消息传递机制,例如消息队列、管道、信号量等。

\textbf{锁机制:包括互斥锁、条件变量、读写锁}互斥锁提供了以排他方式防止数据结构被并发修改的方法。 读写锁允许多个线程同时读共享数据,而对写操作是互斥的。 条件变量可以以原子的方式阻塞进程,直到某个特定条件为真为止。对条件的测试是在互斥锁的保护下进行的。条件变量始终与互斥锁一起使用。

\textbf{等待-通知机制wait/notify 等待}等待通知机制是基于wait和notify方法来实现的,在一个线程内调用该线程锁对象的wait方法,线程将进入等待队列进行等待直到被通知或者被唤醒。

\textbf{信号量机制(Semaphore)。}一个计数信号量。在概念上,信号量维持一组许可证。如果有必要,每个acquire()都会阻塞,直到许可证可用,然后才能使用它。每个release()添加许可证,潜在地释放阻塞获取方。但是,没有使用实际的许可证对象; Semaphore只保留可用数量的计数,并相应地执行。
信号量通常用于限制线程数,而不是访问某些(物理或逻辑)资源。

\textbf{管道(Pipe)}线程可以使用管道来进行通信,管道可以实现两个线程之间的通信。一个线程可以将数据写入管道,另一个线程可以从管道中读取数据。

\end{document}


