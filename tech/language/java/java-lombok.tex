\documentclass[../../../interview-questions.tex]{subfiles}

\begin{document}

\subsection{Lombok实现原理}

自从Java 6起,javac就支持“JSR 269 Pluggable Annotation Processing API”规范,只要程序实现了该API,就能在javac运行的时候得到调用。Lombok就是一个实现了"JSR 269 API"的程序。在使用javac的过程中,它产生作用的具体流程如下:

\begin{enumerate}
\item {javac对源代码进行分析,生成一棵抽象语法树(AST)}
\item{javac编译过程中调用实现了JSR 269的Lombok程序}
\item{此时Lombok就对第一步骤得到的AST进行处理,找到Lombok注解所在类对应的语法树(AST),然后修改该语法树(AST),增加Lombok注解定义的相应树节点}
\item{javac使用修改后的抽象语法树(AST)生成字节码文件}
\end{enumerate}

\subsection{线程同步以及线程调度相关的方法}

\begin{enumerate}
\item {wait():使一个线程处于等待(阻塞)状态,并且释放所持有的对象的锁;}
\item{sleep():使一个正在运行的线程处于睡眠状态,是一个静态方法,调用此方法要处理InterruptedException异常;}
\item{notify():唤醒一个处于等待状态的线程,当然在调用此方法的时候,并不能确切的唤醒某一个等待状态的线程,而是由JVM确定唤醒哪个线程,而且与优先级无关;}
\item{notityAll():唤醒所有处于等待状态的线程,该方法并不是将对象的锁给所有线程,而是让它们竞争,只有获得锁的线程才能进入就绪状态;}
\end{enumerate}

\end{document}