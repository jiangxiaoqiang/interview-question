\documentclass[../../../interview-questions.tex]{subfiles}

\begin{document}

\subsection{三色标记算法}

三色标记算法是一种用于垃圾回收的并发算法,它将对象分为白色、灰色和黑色三种颜色,表示对象的访问状态。

\begin{itemize}
    \item {白色:对象还没有被访问过,可能是垃圾对象。}
    \item {灰色:对象已经被访问过,但是它引用的对象还没有被访问过,需要继续遍历。}
    \item {黑色:对象已经被访问过,并且它引用的对象也都被访问过,不是垃圾对象。}
\end{itemize}

三色标记算法的基本步骤如下:

\begin{enumerate}
    \item {将所有对象标记为白色。}
    \item {从根节点开始遍历所有可达对象,将其标记为灰色,并放入一个队列中。}
    \item {从队列中取出一个灰色对象,将其标记为黑色,并将它引用的所有白色对象标记为灰色并放入队列中。重复此步骤直到队列为空。}
    \item {清除所有仍然是白色的对象。}
\end{enumerate}

三色标记算法可以在程序运行时并发地执行垃圾回收,减少了停顿时间。但是也存在一些问题,比如多标和漏标。这些问题需要通过一些额外的措施来解决或者降低影响。

https://www.jianshu.com/p/e2e8f7a35445

\end{document}