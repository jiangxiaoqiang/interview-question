\documentclass[../../../interview-questions.tex]{subfiles}

\begin{document}

\subsection{Moniter的实现原理}

无论是同步方法还是同步代码块,无论是ACC\_SYNCHRONIZED还是monitorenter、monitorexit都是基于Monitor实现的,什么是Monitor?

\begin{quotation}
管程 (英语:Monitors,也称为监视器) 是一种程序结构,结构内的多个子程序(对象或模块)形成的多个工作线程互斥访问共享资源。这些共享资源一般是硬件设备或一群变量。管程实现了在一个时间点,最多只有一个线程在执行管程的某个子程序。与那些通过修改数据结构实现互斥访问的并发程序设计相比,管程实现很大程度上简化了程序设计。 管程提供了一种机制,线程可以临时放弃互斥访问,等待某些条件得到满足后,重新获得执行权恢复它的互斥访问。
\end{quotation}

在Java虚拟机(HotSpot)中,Monitor是基于C++实现的,由ObjectMonitor\footnote{\url{https://github.com/openjdk-mirror/jdk7u-hotspot/blob/50bdefc3afe944ca74c3093e7448d6b889cd20d1/src/share/vm/runtime/objectMonitor.cpp}}实现的,其主要数据结构如下:

\begin{lstlisting}[language=C]
ObjectMonitor::ObjectMonitor() {  
  _header       = NULL;  
  _count       = 0;  
  _waiters      = 0,  
  _recursions   = 0;       //线程的重入次数
  _object       = NULL;  
  _owner        = NULL;    //标识拥有该monitor的线程
  //等待线程组成的双向循环链表,\_WaitSet是第一个节点
  _WaitSet      = NULL;    
  _WaitSetLock  = 0 ;  
  _Responsible  = NULL ;  
  _succ         = NULL ;  
  _cxq          = NULL ;    //多线程竞争锁进入时的单向链表
  FreeNext      = NULL ;
  //\_owner从该双向循环链表中唤醒线程结点,\_EntryList是第一个节点  
  _EntryList    = NULL ;    
  _SpinFreq     = 0 ;  
  _SpinClock    = 0 ;  
  OwnerIsThread = 0 ;  
}
\end{lstlisting}



\end{document}