\documentclass[../../../interview-questions.tex]{subfiles}

\begin{document}

\subsection{Java的静态代理(Static Proxy)和动态代理(Dynamic Proxy)有什么差别?}

Java三种代理模式:静态代理、动态代理和cglib代理。代理模式是一种设计模式,提供了对目标对象额外的访问方式,即通过代理对象访问目标对象,这样可以在不修改原目标对象的前提下,提供额外的功能操作,扩展目标对象的功能。简言之,代理模式就是设置一个中间代理来控制访问原目标对象,以达到增强原对象的功能和简化访问方式。静态代理在存在较多相似逻辑时,会出现重复编码的情况,实现类总是需要实现一堆接口(冗余)。当接口变化时,相应的所有的实现类需要改变(不易维护)。动态代理在程序运行时运用反射机制动态创建而成。相对来说,动态代理增加了程序的灵活性。CGLib(Code Generation Library)采用了非常底层的字节码技术,其原理是通过字节码技术为一个类创建子类,并在子类中采用方法拦截的技术拦截所有父类方法的调用,顺势织入横切逻辑。JDK动态代理与CGLib动态代理均是实现Spring AOP的基础。

\end{document}