\documentclass[../../../interview-questions.tex]{subfiles}

\begin{document}

\subsection{\color{red}{Java的静态代理(Static Proxy)和动态代理(Dynamic Proxy)有什么差别?}}

动态代理是Spring AOP(Aspect Oriented Programming)的基石,必须理解掌握。动态代理分为JDK动态代理和CGLib(Byte Code Generation Library)动态代理。所谓的动态代理就是指在程序运行期间,在内存中动态的生成代理类的字节码并实例化代理对象。代理,即是将自己需要做的一些事情交给某个代理机构来做,代理机构应该和我有一样的行为,但是可以做更多的事情,比如火车票代购点,其实就是一个代理,它是为火车站做代理的,它和火车站一样,可以卖票,可以办理退票等手续,而且还可以在代理点做更多的操作,比如每次代理都收取5块钱的手续费。
在java中,代理分为静态代理和动态代理。它们的区别在于静态代理在运行之前就已经存在代理类的字节码文件了(.class文件),而动态代理是在运行时通过反射技术来实现的。为了使得代理对象拥有真实对象的行为,我们应该使真实对象和代理对象实现相同的接口,然后代理对象使用真实对象来操作,其中添加一些增强的细节。Java三种代理模式:静态代理、动态代理和CGLIB代理。代理模式是一种设计模式,提供了对目标对象额外的访问方式,即通过代理对象访问目标对象,这样可以在不修改原目标对象的前提下,提供额外的功能操作,扩展目标对象的功能。简言之,代理模式就是设置一个中间代理来控制访问原目标对象,以达到增强原对象的功能和简化访问方式。静态代理在存在较多相似逻辑时,会出现重复编码的情况,实现类总是需要实现一堆接口(冗余)。当接口变化时,相应的所有的实现类需要改变(不易维护)。动态代理在程序运行时运用反射机制动态创建而成。相对来说,动态代理增加了程序的灵活性。相比于静态代理,java动态代理不需要你实现和委托类一样的接口,所有的代理类只需要实现InvocationHandler接口就可以。CGLib是一个高性能的代码生成工具,Spring使用Cglib来做AOP,CGLib可以拦截方法的执行,然后对方法进行增强,而且Cglib实现代理不依赖接口,所以不需要向java的静态代理和动态代理那样实现接口。动态代理需要实现InvocationHandler接口,为了不依赖接口而实现代理,就需要使用Cglib了。CGLib(Code Generation Library)采用了非常底层的字节码技术,其原理是通过字节码技术为一个类创建子类,并在子类中采用方法拦截的技术拦截所有父类方法的调用,顺势织入横切逻辑。JDK动态代理(JDK dynamic proxies )与CGLib动态代理均是实现Spring AOP的基础\footnote{参见:\url{https://docs.spring.io/spring-framework/docs/current/reference/html/core.html\#aop}}。Spring AOP defaults to using standard JDK dynamic proxies for AOP proxies.Spring AOP在代理接口时使用JDK动态代理,其他类型时使用CGLIB proxies。

\end{document}