\documentclass[../../../interview-questions.tex]{subfiles}

\begin{document}

\subsection{Java注解原理}

注解本质上是继承了 Annotation 接口的接口,而当你通过反射,也就是我们这里的 getAnnotation 方法去获取一个注解类实例的时候,其实 JDK 是通过动态代理机制生成一个实现我们注解(接口)的代理类。而解析一个类或者方法的注解往往有两种形式,一种是编译期直接的扫描,一种是运行期反射。反射的事情我们待会说,而编译器的扫描指的是编译器在对 java 代码编译字节码的过程中会检测到某个类或者方法被一些注解修饰,这时它就会对于这些注解进行某些处理。
典型的就是注解 @Override,一旦编译器检测到某个方法被修饰了 @Override 注解,编译器就会检查当前方法的方法签名是否真正重写了父类的某个方法,也就是比较父类中是否具有一个同样的方法签名。
这一种情况只适用于那些编译器已经熟知的注解类,比如 JDK 内置的几个注解,而你自定义的注解,编译器是不知道你这个注解的作用的,当然也不知道该如何处理,往往只是会根据该注解的作用范围来选择是否编译进字节码文件,仅此而已。

\end{document}