\documentclass[../../../interview-questions.tex]{subfiles}

\begin{document}

\subsection{Java注解原理}

注解本质上是继承了 Annotation 接口的接口,而当你通过反射,也就是我们这里的 getAnnotation 方法去获取一个注解类实例的时候,其实 JDK 是通过动态代理(Dynamic Proxy)机制生成一个实现我们注解(接口)的代理类。而解析一个类或者方法的注解往往有两种形式,一种是编译期直接的扫描,一种是运行期反射。反射的事情我们待会说,而编译器的扫描指的是编译器在对 Java 代码编译字节码的过程中会检测到某个类或者方法被一些注解修饰,这时它就会对于这些注解进行某些处理。
典型的就是注解 @Override,一旦编译器检测到某个方法被修饰了 @Override 注解,编译器就会检查当前方法的方法签名是否真正重写了父类的某个方法,也就是比较父类中是否具有一个同样的方法签名。
这一种情况只适用于那些编译器已经熟知的注解类,比如 JDK 内置的几个注解,而你自定义的注解,编译器是不知道你这个注解的作用的,当然也不知道该如何处理,往往只是会根据该注解的作用范围来选择是否编译进字节码文件,仅此而已。实现动态代理的方式很多,比如JDK自身提供的动态代理,就是主要利用了上面提到的反射机制。还有其他的实现方式,比如利用传说中更高性能的字节码操作机制,类似ASM、cglib(基于ASM)、Javassist等。CGLIB动态生成的代理类会继承我们的业务类,并在代理类中对代理方法进行强化处理(前置处理、后置处理等)。CGLIB是高效的代码生成包,底层依靠ASM(开源的java字节码编辑类库)操作字节码实现的,性能比JDK强。CGLIB原理:动态生成一个要代理类的子类,子类重写要代理的类的所有不是final的方法。在子类中采用方法拦截的技术拦截所有父类方法的调用,顺势织入横切逻辑。它比使用java反射的JDK动态代理要快。

\end{document}