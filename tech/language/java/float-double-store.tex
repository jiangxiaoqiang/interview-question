\documentclass[../../../interview-questions.tex]{subfiles}

\begin{document}

\subsection{float和double精度损失问题}

数据库中涉及到金额、价格问题的,建议字段设置成decimal(m,n),否则会造成精度丢失,对应Java中实体类的属性BigDecimal修饰;使用string参数的构造方法,new  BigDecimal(string),若使用double参数的构造方法,会产生你不想要的结果;用compareTo方法比较两值是否相等,而不是equals,equals会比较scale(保留位数),例如2和2.0用equals比较不相等;浮点类型存储时,会导致精度丢失:

\begin{lstlisting}[language=Java]
float fval = 0.45;  // 单步调试发现其真实值为:0.449999988
double dval = 0.45; // 单步调试发现其真实值为:0.45000000000000001
\end{lstlisting}

那么float是如何存储的呢?对于浮点类型的数据采用单精度类型(float)和双精度类型(double)来存储,float数据占用32bit,double数据占用64bit,我们在声明一个变量float f= 2.25f的时候,是如何分配内存的呢?如果胡乱分配,那世界岂不是乱套了么,其实不论是float还是double在存储方式上都是遵从IEEE的规范的,float遵从的是IEEE R32.24 ,而double 遵从的是R64.53\footnote{\url{https://www.cnblogs.com/jillzhang/archive/2007/06/24/793901.html}}。浮点数的精度是由尾数的位数来决定的:对于 float 型浮点数,尾数部分 23 位,换算成十进制就是 $2^{23}=8388608$,所以十进制精度只有 6 至 7 位;对于 double 型浮点数,尾数部分 52 位,换算成十进制就是 $2^{52} = 4503599627370496$,所以十进制精度只有 15 至 16 位。所以浮点数交给计算机存储时可能会出现精度丢失的情况。记住 float 的精度上限是 6至7 位,double 的精度上限是 15至16 位。通过把浮点型数据放大10的整数倍,把它赋给一个整型变量,把得到的结果再除以10的整数倍,就会使精度损失降到最低。单精度或双精度在存储中,都分为三个部分:

符号位 (Sign):0代表正数,1代表为负数;

指数位 (Exponent):用于存储科学计数法中的指数数据;

尾数部分 (Mantissa):采用移位存储尾数部分;

BigDecimal类位于java.math包下,用于对超过16位有效位的数进行精确的运算。一般来说,double类型的变量可以处理16位有效数,但实际应用中,如果超过16位,就需要BigDecimal类来操作。BigDecimal的原理很简单,就是将小数扩大N倍,转成整数后再进行计算,同时结合指数,得出精度损失降到最低的结果。

\end{document}