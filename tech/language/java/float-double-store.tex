\documentclass[../../../interview-questions.tex]{subfiles}

\begin{document}

\subsection{float和double精度损失问题}

数据库中涉及到金额、价格问题的,建议字段设置成decimal(m,n),否则会造成精度丢失,对应Java中实体类的属性BigDecimal修饰;使用string参数的构造方法,new  BigDecimal(string),若使用double参数的构造方法,会产生你不想要的结果;用compareTo方法比较两值是否相等,而不是equals,equals会比较scale(保留位数),例如2和2.0用equals比较不相等;浮点类型存储时,会导致精度丢失:

\begin{lstlisting}[language=Java]
float fval = 0.45;  // 单步调试发现其真实值为:0.449999988
double dval = 0.45; // 单步调试发现其真实值为:0.45000000000000001
\end{lstlisting}

通过把浮点型数据放大10的整数倍,把它赋给一个整型变量,把得到的结果再除以10的整数倍,就会使精度损失降到最低。单精度或双精度在存储中,都分为三个部分:

符号位 (Sign):0代表正数,1代表为负数;

指数位 (Exponent):用于存储科学计数法中的指数数据;

尾数部分 (Mantissa):采用移位存储尾数部分;

BigDecimal的原理很简单,就是将小数扩大N倍,转成整数后再进行计算,同时结合指数,得出精度损失降到最低的结果。

\end{document}