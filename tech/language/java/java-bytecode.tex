\documentclass[../../../interview-questions.tex]{subfiles}

\begin{document}

\subsection{Java字节码原理}

Java之所以可以“一次编译,到处运行”,一是因为JVM针对各种操作系统、平台都进行了定制,二是因为无论在什么平台,都可以编译生成固定格式的字节码(.class文件)供JVM使用。因此,也可以看出字节码对于Java生态的重要性。之所以被称之为字节码,是因为字节码文件由十六进制值组成,而JVM以两个十六进制值为一组,即以字节为单位进行读取。字节码增强技术就是一类对现有字节码进行修改或者动态生成全新字节码文件的技术。字节码是单字节指令组成,因此有256种可能的操作码。实际指令使用大约200个操作码,其中一些操作码保留用于调试器操作。根据指令的性质,我们可以将这些分组分为几个类别:

\begin{itemize}
    \item {堆栈操作指令,包括与局部变量的交互。}
    \item {控制流程}
    \item {对象操作,包括 方法调用}
    \item {算术和类型转换}
\end{itemize}

\end{document}