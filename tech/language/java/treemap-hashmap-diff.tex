\documentclass[../../../interview-questions.tex]{subfiles}

\begin{document}

\subsection{说说TreeMap和HashMap的区别}

性能区别 HashMap的底层是Array,所以HashMap在添加,查找,删除等方法上面速度会非常快。 而TreeMap的底层是一个Tree结构,所以速度会比较慢。 另外HashMap因为要保存一个Array,所以会造成空间的浪费,而TreeMap只保存要保持的节点,所以占用的空间比较小。
HashMap是基于散列表实现的,时间复杂度平均能达到O(1)。 TreeMap基于红黑树(一种自平衡二叉查找树)实现的, 红黑树操作比如插入、删除和查找某个值的最坏情况时间都要求与树的高度成比例,这个在高度上的理论上限允许红黑树在最坏情况下都是高效的。 所以红黑树是复杂而高效的,其检索效率 O(logn) ,最大深度为 2log(n+1)。时间复杂度平均能达到O(log n)。TreeMap:适用于按自然顺序或自定义顺序遍历键(key)。HashMap、TreeMap都继承AbstractMap抽象类;TreeMap实现SortedMap接口,所以TreeMap是有序的!HashMap是无序的.

\end{document}

