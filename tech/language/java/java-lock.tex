\documentclass[../../../interview-questions.tex]{subfiles}

\begin{document}

\subsection{Lock实现原理}

首先和Synchronized的不同之处,Lock完全用Java写成。其实现都依赖AbstractQueuedSynchronizer类,简称AQS。简单说来,AbstractQueuedSynchronizer会把所有的请求线程构成一个CLH队列,当一个线程执行完毕(lock.unlock())时会激活自己的后继节点,但正在执行的线程并不在队列中,而那些等待执行的线程全部处于阻塞状态,线程的显式阻塞是通过调用LockSupport.park()完成,而LockSupport.park()则调用sun.misc.Unsafe.park()本地方法,再进一步,Hotspot在Linux中中通过调用pthread\_mutex\_lock函数把线程交给系统内核进行阻塞。总体来讲线程获取锁要经历以下过程(非公平):    

\begin{enumerate}
    \item {调用lock方法,会先进行cas操作看下可否设置同步状态1成功,如果成功执行临界区代码}
    \item {如果不成功获取同步状态,如果状态是0那么cas设置为1.}
    \item {如果同步状态既不是0也不是自身线程持有会把当前线程构造成一个节点。}
    \item {把当前线程节点CAS的方式放入队列中,行为上线程阻塞,内部自旋获取状态。}
    \item {线程释放锁,唤醒队列第一个节点,参与竞争。重复上述。}
\end{enumerate}

\end{document}