\documentclass[../../../interview-questions.tex]{subfiles}

\begin{document}

\subsection{代理的好处}

代理(Proxy)是一种设计模式,提供了对目标对象另外的访问方式;即通过代理对象访问目标对象.这样做的好处是:可以在目标对象实现的基础上,增强额外的功能操作,即扩展目标对象的功能。这里使用到编程中的一个思想:不要随意去修改别人已经写好的代码或者方法,如果需改修改,可以通过代理的方式来扩展该方法。举个例子来说明代理的作用:假设我们想邀请一位明星,那么并不是直接联系明星,而是联系明星的经纪人,来达到同样的目的.明星就是一个目标对象,他只要负责活动中的节目,而其他琐碎的事情就交给他的代理人(经纪人)来解决.这就是代理思想在现实中的一个例子。隐藏委托类的实现,调用者只需要和代理类进行交互即可。解耦,在不改变委托类代码情况下做一些额外处理,比如添加初始判断及其他公共操作。

https://www.zhihu.com/question/20794107

\paragraph{静态代理类优缺点}

优点:业务类只需要关注业务逻辑本身,保证了业务类的重用性。这是代理的共有优点。
缺点:
1)代理对象的一个接口只服务于一种类型的对象,如果要代理的方法很多,势必要为每一种方法都进行代理,静态代理在程序规模稍大时就无法胜任了。
2)如果接口增加一个方法,除了所有实现类需要实现这个方法外,所有代理类也需要实现此方法。增加了代码维护的复杂度。
另外,如果要按照上述的方法使用代理模式,那么真实角色(委托类)必须是事先已经存在的,并将其作为代理对象的内部属性。但是实际使用时,一个真实角色必须对应一个代理角色,如果大量使用会导致类的急剧膨胀;此外,如果事先并不知道真实角色(委托类),该如何使用代理呢?这个问题可以通过Java的动态代理类来解决。

\paragraph{动态代理类优缺点}

优点
1.动态代理类的字节码在程序运行时由Java反射机制动态生成,无需程序员手工编写它的源代码。
2.动态代理类不仅简化了编程工作,而且提高了软件系统的可扩展性,因为Java 反射机制可以生成任意类型的动态代理类。

缺点
JDK的动态代理机制只能代理实现了接口的类,而不能实现接口的类就不能实现JDK的动态代理,cglib是针对类来实现代理的,他的原理是对指定的目标类生成一个子类,并覆盖其中方法实现增强,但因为采用的是继承,所以不能对final修饰的类进行代理。

\end{document}