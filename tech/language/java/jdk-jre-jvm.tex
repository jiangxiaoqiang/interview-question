\documentclass[../../../interview-questions.tex]{subfiles}

\begin{document}

\subsection{JDK、JRE和JVM三者之间关系}

JDK(Java Development Kit)是针对Java开发员的产品,是整个Java的核心,包括了Java运行环境JRE、Java工具和Java基础类库。Java Runtime Environment(JRE)是运行JAVA程序所必须的环境的集合,包含JVM标准实现及Java核心类库。JVM是Java Virtual Machine(Java虚拟机)的缩写,是整个java实现跨平台的最核心的部分,能够运行以Java语言写作的软件程序。
JDK是java开发工具包,在其安装目录下面有六个文件夹、一些描述文件、一个src压缩文件。bin、include、lib、 jre这四个文件夹起作用,demo、sample是一些例子。可以看出来JDK包含JRE,而JRE包含JVM。

\begin{enumerate}
	\item{bin:最主要的是编译器(javac.exe)}
	\item{include:java和JVM交互用的头文件}
	\item{lib:类库}
	\item{jre:java运行环境(注意:这里的bin、lib文件夹和jre里的bin、lib是不同的)}
\end{enumerate}

总的来说JDK是用于Java程序的开发,而jre则是只能运行class而没有编译的功能。JDK是提供给Java开发人员使用的,其中包含了java的开发工具,也包括了JRE。所以安装了JDK,就不用在单独安装JRE了。其中的开发工具包括编译工具(javac.exe)打包工具(jar.exe)等。JRE是指java运行环境。光有JVM还不能成class的执行,因为在解释class的时候JVM需要调用解释所需要的类库lib。在JDK的安装目录里你可以找到jre目录,里面有两个文件夹bin和lib,在这里可以认为bin里的就是jvm,lib中则是jvm工作所需要的类库,而jvm和lib和起来就称为jre。所以,在你写完java程序编译成.class之后,你可以把这个.class文件和jre一起打包发给朋友,这样你的朋友就可以运行你写程序了。包括Java虚拟机(JVM Java Virtual Machine)和Java程序所需的核心类库等,如果想要运行一个开发好的Java程序,计算机中只需要安装JRE即可。

\end{document}