\documentclass[../../../interview-questions.tex]{subfiles}

\begin{document}

\subsection{Java J.U.C}

J.U.C并发包,即java.util.concurrent包,是JDK的核心工具包,是JDK1.5之后,由 Doug Lea实现并引入。整个java.util.concurrent包,按照功能可以大致划分如下:

\begin{enumerate}
\item {juc-locks 锁框架。}显式锁(互斥锁和速写锁)相关;
\item {juc-atomic 原子类框架。}原子变量类相关,是构建非阻塞算法的基础;早期的JDK版本中,如果要并发的对Integer、Long、Double之类的Java原始类型或引用类型进行操作,一般都需要通过锁来控制并发,以防数据不一致。从JDK1.5开始,引入了java.util.concurrent.atomic工具包,该包提供了许多Java原始/引用类型的映射类,如AtomicInteger、AtomicLong、AtomicBoolean,这些类可以通过一种“无锁算法”,线程安全的操作Integer、Long、Boolean等原始类型。
\item {juc-sync/tools 同步器框架。}同步工具相关,如信号量、闭锁、栅栏等功能;
\item {juc-collections 集合框架。}并发容器相关;这里的juc-collections集合框架,是指java.util.concurrent包下的一些同步集合类,按类型划分可以分为:Set、List、Map、Queue四大类
\item{juc-executors 执行器框架。}线程池相关;executors框架是整个J.U.C包中类/接口关系最复杂的框架,executors其实可以划分为3大块,每一块的核心都是基于Executor这个接口:线程池、Future模式、Fork/Join框架。
\end{enumerate}

\end{document}