\documentclass[../../../interview-questions.tex]{subfiles}

\begin{document}

\subsection{如何保证线程顺序执行}

\paragraph{使用join关键字实现}join关键字用于让当前线程等待join线程执行完毕后再执行,否则会处于等待阻塞状态。

\begin{lstlisting}[language=Java]
class Task implements Runnable{
    private int taskId;
    
    public Task(int taskId){
        this.taskId = taskId; 
    }    
    
    @Override
    public void run(){
        System.out.println("线程"+taskId+"运行!");
    }
}
public void method1() throws InterruptedException{
    Thread t1 = new Thread(new Task(1));
    Thread t2 = new Thread(new Task(2));
    
    t1.start();
    t1.join();//阻塞主线程,直到线程1执行完
    t2.start();
}
\end{lstlisting}

\paragraph{使用队列}把线程依次加入到队列里,按顺序执行即可。newSingleThreadExecutor是一个只有一个消费线程的线程池,这个消费线程会按队列FIFO的顺序去任务队列里取任务,只要保证三个线程按顺序放入就可以了。

\begin{lstlisting}[language=Java]
class Task implements Runnable{
    
    private int taskId;
    
    public Task(int taskId){
        this.taskId = taskId;
    }
    
    public void run(){
        System.out.println("线程"+taskId+"执行!");
    }
}
public void method3(){
    ExecutorService threadPool = Executors.newSingleThreadExecutor();
    threadPool.execute(new Task(1));
    threadPool.execute(new Task(2));
    threadPool.execute(new Task(3));
}
\end{lstlisting}

\paragraph{使用CountDownLatch关键字实现}执行它的latch.await()方法,如果计数器不为0,那么当前线程就会被阻塞;每完成一个任务,就执行latch.countDown(),计数器减一,当计数器为0时,阻塞的线程恢复执行状态\footnote{\url{http://xiaonanbobo.com/2017/12/05/如何保证线程的顺序执行?/}}。



\end{document}