\documentclass[../../../interview-questions.tex]{subfiles}

\begin{document}

\subsection{如何保证线程顺序执行(线程交替打印问题)}

\paragraph{使用join关键字实现}join关键字用于让当前线程等待join线程执行完毕后再执行,否则会处于等待阻塞状态。

\begin{lstlisting}[language=Java]
class Task implements Runnable{
    private int taskId;
    
    public Task(int taskId){
        this.taskId = taskId; 
    }    
    
    @Override
    public void run(){
        System.out.println("线程"+taskId+"运行!");
    }
}
public void method1() throws InterruptedException{
    Thread t1 = new Thread(new Task(1));
    Thread t2 = new Thread(new Task(2));
    
    t1.start();
    t1.join();//阻塞主线程,直到线程1执行完
    t2.start();
}
\end{lstlisting}

\paragraph{使用队列}把线程依次加入到队列里,按顺序执行即可。newSingleThreadExecutor是一个只有一个消费线程的线程池,这个消费线程会按队列FIFO的顺序去任务队列里取任务,只要保证三个线程按顺序放入就可以了。

\begin{lstlisting}[language=Java]
class Task implements Runnable{
    
    private int taskId;
    
    public Task(int taskId){
        this.taskId = taskId;
    }
    
    public void run(){
        System.out.println("线程"+taskId+"执行!");
    }
}
public void method3(){
    ExecutorService threadPool = Executors.newSingleThreadExecutor();
    threadPool.execute(new Task(1));
    threadPool.execute(new Task(2));
    threadPool.execute(new Task(3));
}
\end{lstlisting}

\paragraph{使用CountDownLatch关键字实现}执行它的latch.await()方法,如果计数器不为0,那么当前线程就会被阻塞;每完成一个任务,就执行latch.countDown(),计数器减一,当计数器为0时,阻塞的线程恢复执行状态\footnote{\url{http://xiaonanbobo.com/2017/12/05/如何保证线程的顺序执行}}。应用场景之一是需要下载一个大文件,开启多个线程分别下载文件的一部分,然后有一个线程最后拼接所有的文件。我们可以考虑使用 CountDownLatch 来控制并发,使拼接的线程放在最后执行。

\paragraph{利用newSingleThreadExecutor控制线程的执行顺序}

Java的 Executors 线程池平时工作中用得很多了,Java通过Executors提供了四种线程池,单线程化线程池(newSingleThreadExecutor)、可控最大并发数线程池(newFixedThreadPool)、可回收缓存线程池(newCachedThreadPool)、支持定时与周期性任务的线程池(newScheduledThreadPool)、newWorkStealingPool。

顾名思义,newSingleThreadExecutor 线程池只有一个线程。它存在的意义就在于控制线程执行的顺序,保证任务的执行顺序和提交顺序一致。其实保证顺序执行的原理也很简单,因为总是只有一个线程处理任务队列上的任务,先提交的任务必将被先处理。

\paragraph{object的 wait(),notify()方法,配合synchronized使用}

\paragraph{CyclicBarrier}

栅栏类似于闭锁,它能阻塞一组线程直到某个事件的发生。栅栏与闭锁的关键区别在于,所有的线程必须同时到达栅栏位置,才能继续执行。闭锁用于等待事件,而栅栏用于等待其他线程。 CyclicBarrier可以使一定数量的线程反复地在栅栏位置处汇集。当线程到达栅栏位置时将调用await方法,这个方法将阻塞直到所有线程都到达栅栏位置。如果所有线程都到达栅栏位置,那么栅栏将打开,此时所有的线程都将被释放,而栅栏将被重置以便下次使用。

\paragraph{ReentrantLock}

配合自定义执行条件 原理同wait() notify()。https://ost.51cto.com/posts/3536

\paragraph{Semaphore}

Semaphore 通常我们叫它信号量, 可以用来控制同时访问特定资源的线程数量,通过协调各个线程,以保证合理的使用资源。一个线程要访问共享资源,先使用acquire()方法获得信号量,如果信号量的计数器值大于等于1,意味着有共享资源可以访问,则使其计数器值减去1,再访问共享资源。如果计数器值为0,线程进入休眠。
当某个线程使用完共享资源后,使用release()释放信号量,并将信号量内部的计数器加1,之前进入休眠的线程将被唤醒并再次试图获得信号量。

\end{document}