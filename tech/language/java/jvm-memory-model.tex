\documentclass[../../../interview-questions.tex]{subfiles}

\begin{document}

\subsection{\color{red}{Java内存模型(Java Memory Model)}}

Java的多线程之间是通过共享内存进行通信的,而由于采用共享内存进行通信,在通信过程中会存在一系列如原子性、可见性和有序性的问题。JMM就是为了解决这些问题而出现的,这个模型建立了一些规范,来屏蔽各种硬件和操作系统的内存访问差异,以实现让Java程序在各种平台下都能达到一致的内存访问效果,以保证在多核CPU多线程编程环境下,对共享变量读写的原子性、可见性和有序性。为了既保证CPU的高效执行,又保证共享内存读写的正确性(原子性、可见性和有序性),人们定义了内存模型。内存模型是一个规范,这个规范能保证共享内存读写的正确性。上面提到内存模型的出现是为了解决共享变量读写的原子性、可见性和有序性问题,再简单点说 JMM就是一个为了解决多核CPU多线程编程环境下对共享变量访问存在原子性、可见性和有序性问题的规范。Java线程之间的通信由Java内存模型(本文简称为JMM)控制,JMM决定一个线程对共享变量的写入何时对另一个线程可见。从抽象的角度来看,JMM定义了线程和主内存之间的抽象关系:线程之间的共享变量存储在主内存(Main Memory)中,每个线程都有一个私有的本地内存(Local Memory),本地内存中存储了该线程以读/写共享变量的副本。本地内存是JMM的一个抽象概念,并不真实存在。它涵盖了缓存、写缓冲区、寄存器以及其他的硬件和编译器优化。https://www.cnblogs.com/54chensongxia/p/12022648.html


\end{document}