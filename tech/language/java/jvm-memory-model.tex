\documentclass[../../../interview-questions.tex]{subfiles}

\begin{document}

\subsection{\color{red}{Java内存模型(Java Memory Model)}}

Java线程之间的通信由Java内存模型(本文简称为JMM)控制,JMM决定一个线程对共享变量的写入何时对另一个线程可见。从抽象的角度来看,JMM定义了线程和主内存之间的抽象关系:线程之间的共享变量存储在主内存(Main Memory)中,每个线程都有一个私有的本地内存(Local Memory),本地内存中存储了该线程以读/写共享变量的副本。本地内存是JMM的一个抽象概念,并不真实存在。它涵盖了缓存、写缓冲区、寄存器以及其他的硬件和编译器优化。https://www.cnblogs.com/54chensongxia/p/12022648.html

\end{document}