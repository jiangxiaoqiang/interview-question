\documentclass[../../../interview-questions.tex]{subfiles}

\begin{document}

\subsection{Synchronized和Lock的区别}

区别如下:

\begin{enumerate}
    \item {synchronized是关键字,而Lock是接口}
    \item {synchronized无法判断是否获取锁的状态,Lock可以判断是否获取到锁}
    \item {synchronized自动释放锁(a线程执行完同步代码会自动释放锁,b线程执行过程中发生异常会释放锁)}
    \item {lock需在finally中手动释放锁(unlock()方法释放锁),否则容易造成线程死锁。}
    \item {synchronized关键字的两个线程1和线程2,若当前线程1获得锁,线程2等待,如果线程1阻塞,线程2会一直等待下去。}
    \item {而lock锁不一定会等待下去,如果尝试获得不到锁,线程可以不用一直等待就结束了。}
\end{enumerate}

\paragraph{什么情况下使用Lock}

ReentrantLock 在内存上的语义与 synchronized 相同,但是它提供了额外的功能,可以作为一种高级工具。当需要一些可定时,可轮询,可中断的锁获取操作,或者希望使用公平锁,或者使用非块结构的编码时才应该考虑 ReetrantLock。在业务并发简单清晰的情况下推荐 synchronized,在业务逻辑并发复杂,或对使用锁的扩展性要求较高时,推荐使用 ReentrantLock 这类锁。

\begin{enumerate}
    \item {某个线程在等待一个锁的控制权的这段时间需要中断。}
    \item {分开处理一些 wait-notify,ReentrantLock 里面的 Condition 应用,能够控制 notify 哪个线程。}
    \item {公平锁功能,每个到来的线程都将排队等候。}
\end{enumerate}

当需要更细粒度的控制锁的时候使用Lock\footnote{\url{https://xie.infoq.cn/article/4e370ded27e4419d2a94a44b3}}。


\end{document}