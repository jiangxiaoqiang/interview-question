\documentclass[../../../interview-questions.tex]{subfiles}

\begin{document}

\subsection{在JVM中如何判断一个对象的生死状态?}

\paragraph{引用计数器算法}
通过一系列称之为 “GC Roots” 的根对象作为起点,从这些对象开始向下搜索遍历,所有搜索过的路径称之为引用链(Reference Chain),当一个对象到 GC Roots 没有任何一个引用链相连(对象不可达)时,证明该对象已不再被引用(失效),此时该对象将被 GC 回收

优点:实现简单、性能高。

缺点:增减处理频繁消耗CPU计算、计数器占用很多位浪费空间、最重要的缺点是无法解决循环引用的问题。

\paragraph{可达性分析算法}

3、对象生死与引用的关系

JDK1.2之后对引用进行了扩充,将引用分为:

强引用(Strong Reference)

软引用(Soft Reference)

弱引用(Weak Reference)

虚引用(Phantom Reference)

1.虚拟机栈(栈帧中的本地变量表)中引用的对象\footnote{\url{https://blog.csdn.net/u010798968/article/details/72835255}};

2.方法区中的类静态属性引用的对象;

3.方法区中常量引用的对象;

4.本地方法栈中JNI(即一般说的Native方法)中引用的对象

GC管理的主要区域是Java堆,一般情况下只针对堆进行垃圾回收。方法区、栈和本地方法区不被GC所管理,因而选择这些区域内的对象作为GC roots,被GC roots引用的对象不被GC回收。


4、死亡标记与拯救

\end{document}

