\documentclass[../../../interview-questions.tex]{subfiles}

\begin{document}

\subsection{Java创建线程}

https://stackoverflow.com/questions/15471432/why-implements-runnable-is-preferred-over-extends-thread

\begin{enumerate}
    \item{继承Thread类实现多线程} 继承Thread类,并复写run方法,创建该类对象,调用start方法开启线程。此方式没有返回值。
    \item{覆写Runnable()接口实现多线程,而后同样覆写run()} 实现Runnable接口,复写run方法,创建Thread类对象,将Runnable子类对象传递给Thread类对象。调用start方法开启线程。此方法2较之方法1好,将线程对象和线程任务对象分离开。降低了耦合性,利于维护。此方式没有返回值。Runnable和Thread相比优点有:
    
    \begin{itemize}
        \item {由于Java不允许多继承,因此实现了Runnable接口可以再继承其他类,但是Thread明显不可以}
        \item {Runnable可以实现多个相同的程序代码的线程去共享同一个资源,而Thread并不是不可以,而是相比于Runnable来说,不太适合}
    \end{itemize}

    \item{实现Callable接口,结合 FutureTask使用} 创建FutureTask对象,创建Callable子类对象,复写call(相当于run)方法,将其传递给FutureTask对象(相当于一个Runnable)。 创建Thread类对象,将FutureTask对象传递给Thread对象。调用start方法开启线程。这种方式可以获得线程执行完之后的返回值。该方法使用Runnable功能更加强大的一个子类.这个子类是具有返回值类型的任务方法。
    \item {通过线程池启动多线程}
\end{enumerate}

示例代码:

\begin{lstlisting}[language=Java]
import java.util.concurrent.Callable;
import java.util.concurrent.FutureTask;
import java.util.concurrent.TimeUnit;

public class NewThreadDemo {

    public static void main(String[] args) throws Exception {
            
        //第一种方式
        Thread t1 = new Thread(){
            @Override
            public void run() {
                System.out.println("第1种方式:new Thread 1");
            }
        };
        t1.start();
        
        TimeUnit.SECONDS.sleep(1);
        
        //第二种方式
        Thread t2 = new Thread(new Runnable() {
            @Override
            public void run() {
                System.out.println("第2种方式:new Thread 2");
            }
        });
        t2.start();

        TimeUnit.SECONDS.sleep(1);
        
        //第三种方式
        FutureTask<String> ft = new FutureTask<>(new Callable<String>() {
            @Override
            public String call() throws Exception {
                String result = "第3种方式:new Thread 3";
                return result;
            }
        });
        Thread t3 = new Thread(ft);
        t3.start();
        
        // 线程执行完,才会执行get(),所以FutureTask也可以用于闭锁
        String result = ft.get();
        System.out.println(result);
        
        TimeUnit.SECONDS.sleep(1);
        
        //第四种方式
        ExecutorService pool = Executors.newFixedThreadPool(5);

        Future<String> future = pool.submit(new Callable<String>(){
            @Override
            public String call() throws Exception {
                String result = "第4种方式:new Thread 4";
                return result;
            }
        });

        pool.shutdown();
        System.out.println(future.get());
    }
}       
\end{lstlisting}




\end{document}