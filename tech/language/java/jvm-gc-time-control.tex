\documentclass[../../../interview-questions.tex]{subfiles}

\begin{document}

\subsection{JVM线程池任务调度}

任务调度是线程池的主要入口,当用户提交了一个任务,接下来这个任务将如何执行都是由这个阶段决定的。了解这部分就相当于了解了线程池的核心运行机制。

首先,所有任务的调度都是由execute方法完成的,这部分完成的工作是:检查现在线程池的运行状态、运行线程数、运行策略,决定接下来执行的流程,是直接申请线程执行,或是缓冲到队列中执行,亦或是直接拒绝该任务。其执行过程如下:

首先检测线程池运行状态,如果不是RUNNING,则直接拒绝,线程池要保证在RUNNING的状态下执行任务。
如果workerCount < corePoolSize,则创建并启动一个线程来执行新提交的任务。
如果workerCount >= corePoolSize,且线程池内的阻塞队列未满,则将任务添加到该阻塞队列中。
如果workerCount >= corePoolSize \&\& workerCount < maximumPoolSize,且线程池内的阻塞队列已满,则创建并启动一个线程来执行新提交的任务。
如果workerCount >= maximumPoolSize,并且线程池内的阻塞队列已满, 则根据拒绝策略来处理该任务, 默认的处理方式是直接抛异常。


\end{document}

