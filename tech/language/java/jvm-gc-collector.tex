\documentclass[../../../interview-questions.tex]{subfiles}

\begin{document}

\subsection{JVM的垃圾收集器哪几种}

JVM的垃圾回收算法一共有7种。新生代3种(Serial/ParNew(Parallel New)/Parallel Scavenge),老年代3种(CMS/Serial Old(MSC\footnote{Mark Sweep Collection的缩写,不过未找到可信的出处。})/Parallel Old),外加年轻代和老年代都会用到的G1(Garbage First)收集器。

\paragraph{年轻代收集器(Yong Generation Collector)}
在用户的桌面应用场景中,分配给虚拟机管理的内存一般不会很大,收集几十兆甚至一两百兆的新生代,停顿时间完全可以控制在几十毫秒最多一百毫秒以内,只要不频繁发生,这点停顿时间可以接受。所以,Serial收集器对于运行在Client模式下的虚拟机来说是一个很好的选择,可以通过-XX:+UseSerialGC来强制指定。

Parallel Scavenge\footnote{Scavenge有清除的意思。}收集器除了会显而易见地提供可以精确控制吞吐量的参数,还提供了一个参数-XX:+UseAdaptiveSizePolicy,这是一个开关参数,打开参数后,就不需要手工指定新生代的大小(-Xmn)、Eden和Survivor区的比例(-XX:SurvivorRatio)、晋升老年代对象年龄(-XX:PretenureSizeThreshold)等细节参数了,虚拟机会根据当前系统的运行情况收集性能监控信息,动态调整这些参数以提供最合适的停顿时间或者最大的吞吐量,这种方式称为GC自适应的调节策略(GC Ergonomics)。自适应调节策略也是Parallel Scavenge收集器与ParNew收集器的一个重要区别。

\paragraph{老年代收集器}
Serial Old 是 Serial收集器的老年代版本,它同样是一个单线程收集器,使用“标记-整理”(Mark-Compact)算法。此收集器的主要意义也是在于给Client模式下的虚拟机使用。如果在Server模式下,它还有两大用途:

在JDK1.5 以及之前版本(Parallel Old诞生以前)中与Parallel Scavenge收集器搭配使用。
作为CMS收集器的后备预案,在并发收集发生Concurrent Mode Failure时使用。
它的工作流程与Serial收集器相同

Parallel Old收集器是Parallel Scavenge收集器的老年代版本,使用多线程和“标记-整理”算法。前面已经提到过,这个收集器是在JDK 1.6中才开始提供的,在此之前,如果新生代选择了Parallel Scavenge收集器,老年代除了Serial Old以外别无选择,所以在Parallel Old诞生以后,“吞吐量优先”收集器终于有了比较名副其实的应用组合,在注重吞吐量以及CPU资源敏感的场合,都可以优先考虑Parallel Scavenge加Parallel Old收集器。Parallel Old收集器的工作流程与Parallel Scavenge相同


CMS(Concurrent Mark Sweep)收集器是一种以获取最短回收停顿时间为目标的收集器,它非常符合那些集中在互联网站或者B/S系统的服务端上的Java应用,这些应用都非常重视服务的响应速度。从名字上(“Mark Sweep”)就可以看出它是基于“标记-清除”算法实现的。

CMS收集器工作的整个流程分为以下4个步骤:

初始标记(CMS initial mark):仅仅只是标记一下GC Roots能直接关联到的对象,速度很快,需要“Stop The World”。
并发标记(CMS concurrent mark):进行GC Roots Tracing的过程,在整个过程中耗时最长。
重新标记(CMS remark):为了修正并发标记期间因用户程序继续运作而导致标记产生变动的那一部分对象的标记记录,这个阶段的停顿时间一般会比初始标记阶段稍长一些,但远比并发标记的时间短。此阶段也需要“Stop The World”。
并发清除(CMS concurrent sweep)
由于整个过程中耗时最长的并发标记和并发清除过程收集器线程都可以与用户线程一起工作,所以,从总体上来说,CMS收集器的内存回收过程是与用户线程一起并发执行的。


G1(Garbage-First)收集器是当今收集器技术发展最前沿的成果之一,它是一款面向服务端应用的垃圾收集器,HotSpot开发团队赋予它的使命是(在比较长期的)未来可以替换掉JDK 1.5中发布的CMS收集器。与其他GC收集器相比,G1具备如下特点:

并行与并发 G1 能充分利用多CPU、多核环境下的硬件优势,使用多个CPU来缩短“Stop The World”停顿时间,部分其他收集器原本需要停顿Java线程执行的GC动作,G1收集器仍然可以通过并发的方式让Java程序继续执行。
分代收集 与其他收集器一样,分代概念在G1中依然得以保留。虽然G1可以不需要其他收集器配合就能独立管理整个GC堆,但它能够采用不同方式去处理新创建的对象和已存活一段时间、熬过多次GC的旧对象来获取更好的收集效果。
空间整合 G1从整体来看是基于“标记-整理”算法实现的收集器,从局部(两个Region之间)上来看是基于“复制”算法实现的。这意味着G1运行期间不会产生内存空间碎片,收集后能提供规整的可用内存。此特性有利于程序长时间运行,分配大对象时不会因为无法找到连续内存空间而提前触发下一次GC。
可预测的停顿 这是G1相对CMS的一大优势,降低停顿时间是G1和CMS共同的关注点,但G1除了降低停顿外,还能建立可预测的停顿时间模型,能让使用者明确指定在一个长度为M毫秒的时间片段内,消耗在GC上的时间不得超过N毫秒,这几乎已经是实时Java(RTSJ)的垃圾收集器的特征了。
\end{document}

