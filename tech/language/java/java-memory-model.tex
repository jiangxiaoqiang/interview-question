\documentclass[../../../interview-questions.tex]{subfiles}

\begin{document}

\subsection{Java内存模型(Java Memory Model)}

根据JVM虚拟机规范,运行时数据区通常包括这几个部分:程序计数器(Program Counter Register)、Java栈(VM Stack)、本地方法栈(Native Method Stack)、方法区(Method Area)、堆(Heap)。Java栈中存放的是一个个的栈帧,每个栈帧对应一个被调用的方法,在栈帧中包括局部变量表(Local Variables)、操作数栈(Operand Stack)、指向当前方法所属的类的运行时常量池(运行时常量池的概念在方法区部分会谈到)的引用(Reference to runtime constant pool)、方法返回地址(Return Address)和一些额外的附加信息。当线程执行一个方法时,就会随之创建一个对应的栈帧,并将建立的栈帧压栈。当方法执行完毕之后,便会将栈帧出栈。\textbf{程序计数器(Program Counter Register)}是一个记录着当前线程所执行的字节码的行号指示器。JVM的多线程是通过CPU时间片轮转(即线程轮流切换并分配处理器执行时间)算法来实现的。也就是说,某个线程在执行过程中可能会因为时间片耗尽而被挂起,而另一个线程获取到时间片开始执行。当被挂起的线程重新获取到时间片的时候,它要想从被挂起的地方继续执行,就必须知道它上次执行到哪个位置,在JVM中,通过程序计数器来记录某个线程的字节码执行位置。因此,程序计数器是具备线程隔离的特性,也就是说,每个线程工作时都有属于自己的独立计数器。\textbf{方法区(Method Area)}在JVM中也是一个非常重要的区域,它与堆一样,是被线程共享的区域。在方法区中,存储了每个类的信息(包括类的名称、方法信息、字段信息)、静态变量、常量以及编译器编译后的代码等。\textbf{虚拟机栈(VM Stack)}:每个线程有一个私有的栈,随着线程的创建而创建。栈里面存着的是一种叫“栈帧”的东西,每个方法会创建一个栈帧,栈帧中存放了局部变量表(基本数据类型和对象引用)、操作数栈、方法出口等信息。栈的大小可以固定也可以动态扩展。当栈调用深度大于JVM所允许的范围,会抛出StackOverflowError的错误.

\tikzstyle{heap} = [rectangle, rounded corners, minimum width=3cm, minimum height=1cm,text centered, draw=black, fill=red!30]
\tikzstyle{heapshare} = [rectangle, rounded corners, minimum width=3cm, minimum height=1cm,text centered, draw=black, fill=red!30]
\tikzstyle{heapindependent} = [rectangle, rounded corners, minimum width=3cm, minimum height=1cm,text centered, draw=black, fill=red!30]
\tikzstyle{methodarea} = [rectangle, rounded corners, minimum width=4cm, minimum height=2cm,text centered, draw=black, fill=yellow!30]
\tikzstyle{javaheap} = [rectangle, rounded corners, minimum width=4cm, minimum height=2cm,text centered, draw=black, fill=yellow!30]
\begin{tikzpicture}
\node at (4, 0) [heap, draw=red!50, fill=red!20, text=red!70] {虚拟机栈(VM Stack)};
\node at (8, 0) [heap, draw=red!50, fill=red!20, text=red!70,text width=3cm] {本地方法栈 (Native Method Stack)};
\node at (0, 0) [heap, draw=red!50, fill=red!20, text=red!70,text width=3.5cm] {程序计数器(Program Counter Register)};
\node at (0, -2) [javaheap, draw=red!50, fill=yellow!30, text=red!70] {堆(Heap)};
\node at (5, -2) [methodarea, draw=red!50, fill=yellow!30, text=red!70] {方法区(Method Area)};
\end{tikzpicture}

注意,虚拟机规范并不是一成不变的,Oracle在发布新的JAVA版本时,可能会对JVM做一定的优化和改进,例如在JDK8的版本中,方法区被移除,取而代之的是metaspace(元数据空间)。JVM中的堆(Heap),一般分为三大部分:新生代、老年代、永久代:新生代又分为 Eden区、ServivorFrom、ServivorTo三个区。Eden区:Java新对象的出生地(如果新创建的对象占用内存很大,则直接分配到老年代)。当Eden区内存不够的时候就会触发MinorGC,对新生代区进行一次垃圾回收。ServivorTo:保留了一次MinorGC过程中的幸存者。ServivorFrom:上一次GC的幸存者,作为这一次GC的被扫描者。




\end{document}