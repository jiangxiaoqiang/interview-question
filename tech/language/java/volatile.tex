\documentclass[../../../interview-questions.tex]{subfiles}

\begin{document}

\subsection{\color{red}{volatile的实现原理}}

在 x86 处理器下通过工具获取 JIT(Java-In-Time Compiler)编译器生成的汇编指令来看看对 volatile 进行写操作 CPU 会做什么事情。

\begin{lstlisting}
0x01a3de1d: movb $0x0,0x1104800(%esi);
0x01a3de24: lock addl $0x0,(%esp);
\end{lstlisting}

有 volatile 变量修饰的共享变量进行写操作的时候会多第二行汇编代码,通过查 IA-32 架构软件开发者手册可知,lock 前缀的指令在多核处理器下会引发了两件事情。将当前处理器缓存行的数据会写回到系统内存。这个写回内存的操作会引起在其他 CPU 里缓存了该内存地址的数据无效。处理器为了提高处理速度,不直接和内存进行通讯,而是先将系统内存的数据读到内部缓存(L1,L2 或其他)后再进行操作,但操作完之后不知道何时会写到内存,如果对声明了 volatile 变量进行写操作,JVM 就会向处理器发送一条 lock 前缀的指令,将这个变量所在缓存行的数据写回到系统内存。但是就算写回到内存,如果其他处理器缓存的值还是旧的,再执行计算操作就会有问题,所以在多处理器下,为了保证各个处理器的缓存是一致的,就会实现缓存一致性协议\footnote{参见:\url{https://www.wikiwand.com/en/MESI_protocol}}(MESI Protocol)。其实缓存一致性协议还有很多,例如:MSI、MESI、MOSI、Synapse、Firefly 以及 Dragon Protocol 等等\footnote{参见:\url{https://createchance.github.io/post/java-并发之基石篇/}}。每个处理器通过嗅探在总线上传播的数据来检查自己缓存的值是不是过期了,当处理器发现自己缓存行对应的内存地址被修改,就会将当前处理器的缓存行设置成无效状态,当处理器要对这个数据进行修改操作的时候,会强制重新从系统内存里把数据读到处理器缓存里\footnote{\url{https://www.infoq.cn/article/ftf-java-volatile}}。

\paragraph{volatile关键字有什么作用?}

Java 语言中的 volatile 变量可以被看作是一种 “程度较轻的 synchronized”;与 synchronized 块相比,volatile 变量所需的编码较少,并且运行时开销也较少,但是它所能实现的功能也仅是 synchronized 的一部分。volatile 变量具有 synchronized 的可见性特性,但是不具备原子特性。这就是说线程能够自动发现 volatile 变量的最新值。volatile 变量可用于提供线程安全,但是只能应用于非常有限的一组用例:多个变量之间或者某个变量的当前值与修改后值之间没有约束。

\paragraph{happens-before与volatile语义}

从JDK5开始,Java使用新的JSR-133内存模型。JSR-133使用happens-before的概念来阐述操作之间的内存可见性。在JMM中,如果一个操作执行的结果需要对另一个操作可见,以这两个操作之间必须要有happens-before关系。这里提到的两个操作,既可以是在一个线程之内,也可以是在不同的线程之间。与程序员密切相关的happens-before规则如下。

1)程序顺序规则:一个线程中的每个操作,happens-before于该线程的任意后续操作。

2)监视器锁规则:对一个锁的解锁,happens-before于随后对这个锁的加锁。

3)volatile变量规则:对一个volatile域的写,happens-before于任意后续对这个volatile域的读。

4)传递性:如果A happens-before B,B happens-before C,那么A happens-before C.

\paragraph{Java 中如何保证底层操作的有序性和可见性}

Java 中如何保证底层操作的有序性和可见性?可以通过内存屏障。内存屏障是被插入两个 CPU 指令之间的一种指令,用来禁止处理器指令发生重排序(像屏障一样),从而保障有序性的。另外,为了达到屏障的效果,它也会使处理器写入、读取值之前,将主内存的值写入高速缓存,清空无效队列,从而保障可见性。对于上面的一组 CPU 指令(Store 表示写入指令,Load 表示读取指令,StoreLoad 代表写读内存屏障),StoreLoad 屏障之前的 Store 指令无法与 StoreLoad 屏障之后的 Load 指令进行交换位置,即重排序。但是 StoreLoad 屏障之前和之后的指令是可以互换位置的,即 Store1 可以和 Store2 互换,Load2 可以和 Load3 互换。常见有 4 种屏障:

\begin{itemize}
    \item {LoadLoad 屏障:对于这样的语句 Load1;LoadLoad;Load2,在 Load2 及后续读取操作要读取的数据被访问前,保证 Load1 要读取的数据被读取完毕。}
    \item {StoreStore 屏障:对于这样的语句 Store1;StoreStore;Store2,在 Store2 及后续写入操作执行前,保证 Store1 的写入操作对其他处理器可见。}
    \item {LoadStore 屏障:对于这样的语句 Load1;LoadStore;Store2,在 Store2 及后续写入操作被执行前,保证 Load1 要读取的数据被读取完毕。}
    \item {StoreLoad 屏障:对于这样的语句 Store1;StoreLoad;Load2,在 Load2 及后续所有读取操作执行前,保证 Store1 的写入对所有处理器可见。它的开销是四种屏障中最大的(冲刷写缓冲器,清空无效化队列)。在大多数处理器的实现中,这个屏障是个万能屏障,兼具其他三种内存屏障的功能。}
\end{itemize}


\end{document}