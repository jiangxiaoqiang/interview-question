\documentclass[../../../interview-questions.tex]{subfiles}

\begin{document}

\subsection{Volatile的实现原理}

在 x86 处理器下通过工具获取 JIT(Java-In-Time Compiler)编译器生成的汇编指令来看看对 Volatile 进行写操作 CPU 会做什么事情。

\begin{lstlisting}
0x01a3de1d: movb $0x0,0x1104800(%esi);
0x01a3de24: lock addl $0x0,(%esp);
\end{lstlisting}

有 volatile 变量修饰的共享变量进行写操作的时候会多第二行汇编代码,通过查 IA-32 架构软件开发者手册可知,lock 前缀的指令在多核处理器下会引发了两件事情。将当前处理器缓存行的数据会写回到系统内存。这个写回内存的操作会引起在其他 CPU 里缓存了该内存地址的数据无效。处理器为了提高处理速度,不直接和内存进行通讯,而是先将系统内存的数据读到内部缓存(L1,L2 或其他)后再进行操作,但操作完之后不知道何时会写到内存,如果对声明了 Volatile 变量进行写操作,JVM 就会向处理器发送一条 Lock 前缀的指令,将这个变量所在缓存行的数据写回到系统内存。但是就算写回到内存,如果其他处理器缓存的值还是旧的,再执行计算操作就会有问题,所以在多处理器下,为了保证各个处理器的缓存是一致的,就会实现缓存一致性协议\footnote{参见:\url{https://www.wikiwand.com/en/MESI_protocol}}(MESI Protocol)。其实缓存一致性协议还有很多,例如:MSI、MESI、MOSI、Synapse、Firefly 以及 Dragon Protocol 等等\footnote{参见:\url{https://createchance.github.io/post/java-并发之基石篇/}}。每个处理器通过嗅探在总线上传播的数据来检查自己缓存的值是不是过期了,当处理器发现自己缓存行对应的内存地址被修改,就会将当前处理器的缓存行设置成无效状态,当处理器要对这个数据进行修改操作的时候,会强制重新从系统内存里把数据读到处理器缓存里\footnote{\url{https://www.infoq.cn/article/ftf-java-volatile}}。

\paragraph{volatile关键字有什么作用?}

Java 语言中的 volatile 变量可以被看作是一种 “程度较轻的 synchronized”;与 synchronized 块相比,volatile 变量所需的编码较少,并且运行时开销也较少,但是它所能实现的功能也仅是 synchronized 的一部分。Volatile 变量具有 synchronized 的可见性特性,但是不具备原子特性。这就是说线程能够自动发现 volatile 变量的最新值。Volatile 变量可用于提供线程安全,但是只能应用于非常有限的一组用例:多个变量之间或者某个变量的当前值与修改后值之间没有约束。

\end{document}