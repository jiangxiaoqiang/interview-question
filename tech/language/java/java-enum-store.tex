\documentclass[../../../interview-questions.tex]{subfiles}

\begin{document}

\subsection{Java枚举存储}

ordinal 用于获取当前枚举在定义时的索引, 从 0 开始 依次累加,当做一些状态存储时,使用状态的索引顺序来代表某一个状态,例如 一个枚举中定义了 oneDay, twoDay, threedDay 使用 ordinal 获取时分别值为 0,1,2 即存储的 0,1,2 分别代表了上面的三个枚举。Java枚举存储时,可以存储值或者字符串。存储ordinal值时,一个缺点是如果调整了枚举的顺序,那么已经存储的枚举数据的映射关系会被破坏。存储ordinal值好处是更节省存储空间,在数据库性能方面可能会有优势。而存储枚举的名字带来的好处是可以调整枚举顺序,在数据库中更加容易理解,不用去翻字典搞明白1代表什么,2代表什么\footnote{\url{https://stackoverflow.com/questions/2801953/how-should-i-store-an-java-enum-in-javadb?noredirect=1&lq=1}}。具体存储什么数据可以根据具体需求综合考虑,推荐存储值,毕竟现在存储空间相对来说更廉价,可读性,容易理解更加重要。


\end{document}