\documentclass[../../../interview-questions.tex]{subfiles}

\begin{document}

\subsection{互斥锁(Mutex Lock)和自旋锁(Spin Lock)的区别}

最底层的两种就是会「互斥锁和自旋锁」,有很多高级的锁都是基于它们实现的,你可以认为它们是各种锁的地基,所以我们必须清楚它俩之间的区别和应用。互斥锁加锁失败后,线程会释放 CPU ,给其他线程;自旋锁加锁失败后,线程会忙等待,直到它拿到锁;自旋锁与互斥锁有点类似,只是自旋锁不会引起调用者睡眠,如果自旋锁已经被别的执行单元保持,调用者就一直循环在那里看是 否该自旋锁的保持者已经释放了锁,"自旋"一词就是因此而得名。其作用是为了解决某项资源的互斥使用。因为自旋锁不会引起调用者睡眠,所以自旋锁的效率远 高于互斥锁。虽然它的效率比互斥锁高,但是它也有些不足之处:1、自旋锁一直占用CPU,他在未获得锁的情况下,一直运行--自旋,所以占用着CPU,如果不能在很短的时 间内获得锁,这无疑会使CPU效率降低。2、在用自旋锁时有可能造成死锁,当递归调用时有可能造成死锁,调用有些其他函数也可能造成死锁,如 copy\_to\_user()、copy\_from\_user()、kmalloc()等。
因此我们要慎重使用自旋锁,自旋锁只有在内核可抢占式或SMP的情况下才真正需要,在单CPU且不可抢占式的内核下,自旋锁的操作为空操作。自旋锁适用于锁使用者保持锁时间比较短的情况下。

\end{document}

