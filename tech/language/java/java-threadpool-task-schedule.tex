\documentclass[../../../interview-questions.tex]{subfiles}

\begin{document}

\subsection{Java线程池原理}

J.U.C提供的线程池:ThreadPoolExecutor类,帮助开发人员管理线程并方便地执行并行任务。了解并合理使用线程池,是一个开发人员必修的基本功。线程过多会带来额外的开销,其中包括创建销毁线程的开销、调度线程的开销等等,同时也降低了计算机的整体性能。线程池维护多个线程,等待监督管理者分配可并发执行的任务。这种做法,一方面避免了处理任务时创建销毁线程开销的代价,另一方面避免了线程数量膨胀导致的过分调度问题,保证了对内核的充分利用\footnote{参见:\url{https://tech.meituan.com/2020/04/02/java-pooling-pratice-in-meituan.html}}。JUC是java.util.concurrent包的简称,在Java5.0添加,目的就是为了更好的支持高并发任务。

待补充。

\end{document}



