\documentclass[../../../interview-questions.tex]{subfiles}

\begin{document}

\subsection{线程状态}

Java中线程的状态分为6种。这6种状态定义在src/share/classes/java/lang包的Thread.java类的State枚举中。

\begin{enumerate}
    \item {初始(NEW):新创建了一个线程对象,但还没有调用start()方法。}
    \item {运行(RUNNABLE):Java线程中将就绪(ready)和运行中(running)两种状态笼统的称为“运行”。
    线程对象创建后,其他线程(比如main线程)调用了该对象的start()方法。该状态的线程位于可运行线程池中,等待被线程调度选中,获取CPU的使用权,此时处于就绪状态(ready)。就绪状态的线程在获得CPU时间片后变为运行中状态(running)。}
    \item {阻塞(BLOCKED):表示线程阻塞于锁。}
    \item {等待(WAITING):进入该状态的线程需要等待其他线程做出一些特定动作(通知或中断)。}
    \item {超时等待(TIMED\_WAITING):该状态不同于WAITING,它可以在指定的时间后自行返回。}
    \item {终止(TERMINATED):表示该线程已经执行完毕。}
\end{enumerate}

https://blog.csdn.net/pange1991/article/details/53860651

\end{document}