\documentclass[../../../interview-questions.tex]{subfiles}

\begin{document}

\subsection{垃圾回收历史(The History of Garbage Collection)}

很难想象垃圾回收这个概念在20世纪60年代都已经出现了,所提出来都处理方式过去了半个多世纪了还在沿用。John McCarthy身为Lisp之父和人工智能之父,同时,他也是GC之父。1960年,他在其论文\footnote{\url{http://www-formal.stanford.edu/jmc/recursive/node4.html\#tex2html8}}中首次发布了GC算法(作为对比,世界上第一台通用计算机“ENIAC”于1946年2月14日诞生,Lisp语言首次出现是在1958年,Unix操作系统首次发布的时间是1971年11月3日,C语言首次出现是在1972年)。而Java的前身Oak是在1990发布的,利用JVM实现了跨平台。目前有GC机制的语言有Lisp、Java、Ruby、Python、Perl、Haskell。

\begin{enumerate}
\item{1959: D. Edwards实现了GC} 
\item{1960: John McCarthy发布了初代GC算法即标记-清除(Mark Sweep)算法}
\item{1960: George E. Collins发布了引用计数算法}
\item{1963: Marvin L. Minsky发布了复制算法}
\item{1996: 首次出版了Garbage Collection一书}
\end{enumerate}


\end{document}