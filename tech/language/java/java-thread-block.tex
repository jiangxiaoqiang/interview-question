\documentclass[../../../interview-questions.tex]{subfiles}

\begin{document}

\subsection{线程阻塞}

除了抢占锁的时候会出现线程阻塞,另外还有一些方法也会产生线程阻塞,比如: Object.wait(), Thread.sleep(), ArrayBlockingQueue.put() 等等,他们都有一个共同特点:不消耗 CPU 时间片。最终会调用 LockSupport.park(this) 阻塞当前线程,同样的 ReentrantLock.unlock 方法会调用 LockSupport.unpark(thread) 来恢复阻塞的线程。

(1)线程睡眠:Thread.sleep (long millis)方法,使线程转到阻塞状态。millis参数设定睡眠的时间,以毫秒为单位。当睡眠结束后,就转为就绪(Runnable)状态。sleep()平台移植性好。

(2)线程等待:Object类中的wait()方法,导致当前的线程等待,直到其他线程调用此对象的 notify() 唤醒方法。这个两个唤醒方法也是Object类中的方法,行为等价于调用 wait() 一样。wait() 和 notify() 方法:两个方法配套使用,wait() 使得线程进入阻塞状态,它有两种形式,一种允许 指定以毫秒为单位的一段时间作为参数,另一种没有参数,前者当对应的 notify() 被调用或者超出指定时间时线程重新进入可执行状态,后者则必须对应的 notify() 被调用.

(3)线程礼让,Thread.yield() 方法,暂停当前正在执行的线程对象,把执行机会让给相同或者更高优先级的线程。yield() 使得线程放弃当前分得的 CPU 时间,但是不使线程阻塞,即线程仍处于可执行状态,随时可能再次分得 CPU 时间。调用 yield() 的效果等价于调度程序认为该线程已执行了足够的时间从而转到另一个线程.

(4)线程自闭,join()方法,等待其他线程终止。在当前线程中调用另一个线程的join()方法,则当前线程转入阻塞状态,直到另一个进程运行结束,当前线程再由阻塞转为就绪状态。

(5)suspend() 和 resume() 方法:两个方法配套使用,suspend()使得线程进入阻塞状态,并且不会自动恢复,必须其对应的resume() 被调用,才能使得线程重新进入可执行状态。典型地,suspend() 和 resume() 被用在等待另一个线程产生的结果的情形:测试发现结果还没有产生后,让线程阻塞,另一个线程产生了结果后,调用 resume() 使其恢复。Thread中suspend()和resume()两个方法在JDK1.5中已经废除,不再介绍。因为有死锁倾向。

\paragraph{线程阻塞原理}

阻塞,一般有两个特性很亮眼:1. 不耗 CPU 等待;2. 线程安全;Java层面的阻塞最终都是依赖了 AbstractQueuedSynchronizer 类(著名的AQS)的 await 方法,看起来像那么回事。那么这个AbstractQueuedSynchronizer同步器的阻塞又是如何实现的呢?AbstractQueuedSynchronizer真正的阻塞工作又转交给了另一个工具类: LockSupport 的 park 方法。那 park() 方法到底是如何实现的呢? 其实是继承的 os::PlatformParker 的功能,也就是平台相关的私有实现,以 Linux 平台实现为例(os\_linux.hpp)。在进行阻塞的时候,底层并没有(并不一定)要用 while 死循环来阻塞,更多的是借助于操作系统的实现来进行阻塞的。https://smartkeyerror.com/Linux-Blocking

\end{document}