\documentclass[../../../interview-questions.tex]{subfiles}

\begin{document}

\subsection{Java线程池}

\paragraph{线程池主要组件}

一个线程池包括以下四个基本组成部分:

线程池管理器(ThreadPool):用于创建并管理线程池,包括 创建线程池,销毁线程池,添加新任务;
工作线程(WorkThread):线程池中线程,在没有任务时处于等待状态,可以循环的执行任务;
任务接口(Task):每个任务必须实现的接口,以供工作线程调度任务的执行,它主要规定了任务的入口,任务执行完后的收尾工作,任务的执行状态等;
任务队列(taskQueue):用于存放没有处理的任务。提供一种缓冲机制。

\paragraph{创建一个线程池需要输入几个参数}

\textbf{corePoolSize(线程池的基本大小)}:当提交一个任务到线程池时,线程池会创建一个线程来执行任务,即使其他空闲的基本线程能够执行新任务也会创建线程,等到需要执行的任务数大于线程池基本大小时就不再创建。如果调用了线程池的prestartAllCoreThreads方法,线程池会提前创建并启动所有基本线程。

\textbf{maximumPoolSize(线程池最大大小)}:线程池允许创建的最大线程数。如果队列满了,并且已创建的线程数小于最大线程数,则线程池会再创建新的线程执行任务。值得注意的是如果使用了无界的任务队列这个参数就没什么效果。

\textbf{runnableTaskQueue(任务队列)}:用于保存等待执行的任务的阻塞队列。
ThreadFactory:用于设置创建线程的工厂,可以通过线程工厂给每个创建出来的线程设置更有意义的名字,Debug和定位问题时非常又帮助。

\textbf{RejectedExecutionHandler(拒绝策略)}:当队列和线程池都满了,说明线程池处于饱和状态,那么必须采取一种策略处理提交的新任务。这个策略默认情况下是AbortPolicy,表示无法处理新任务时抛出异常。以下是JDK1.5提供的四种策略。n  AbortPolicy:直接抛出异常。

\textbf{keepAliveTime(线程活动保持时间,线程池的​​工作线程空闲后​​​,​​保持存活的时间​​​)}:线程池的工作线程空闲后,保持存活的时间。所以如果任务很多,并且每个任务执行的时间比较短,可以调大这个时间,提高线程的利用率。

\textbf{TimeUnit(线程活动保持时间的单位)}:可选的单位有天(DAYS),小时(HOURS),分钟(MINUTES),毫秒(MILLISECONDS),微秒(MICROSECONDS, 千分之一毫秒)和毫微秒(NANOSECONDS, 千分之一微秒)。

\paragraph{线程池5个状态}

线程池的5种状态:Running、ShutDown、Stop、Tidying、Terminated。

\textbf{Running}(1) 状态说明:线程池处在RUNNING状态时,能够接收新任务,以及对已添加的任务进行处理。
(2) 状态切换:线程池的初始化状态是RUNNING。换句话说,线程池被一旦被创建,就处于RUNNING状态,并且线程池中的任务数为0!


\end{document}