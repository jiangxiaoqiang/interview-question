\documentclass[../../../interview-questions.tex]{subfiles}

\begin{document}

\subsection{\color{red}{synchronized的实现原理是什么}}

synchronized 是由一对 monitorenter/monitorexit 指令实现的,monitor对象是同步的基本实现单元。在 Java 6 之前,monitor 的实现完全是依靠操作系统内部的互斥锁,因为需要进行用户态到内核态的切换,所以同步操作是一个无差别的重量级操作,性能也很低。但在 Java 6 的时候,Java 虚拟机 对此进行了大刀阔斧地改进,提供了三种不同的 monitor 实现,也就是常说的三种不同的锁:偏向锁(Biased Locking)、轻量级锁和重量级锁,大大改进了其性能。在Java虚拟机(HotSpot)中,Monitor是由ObjectMonitor实现的,其主要数据结构如下(位于HotSpot虚拟机源码ObjectMonitor.hpp文件,C++实现的)。

\begin{lstlisting}[language=C++]
ObjectMonitor() {
    _header       = NULL;
    _count        = 0; // 记录个数
    _waiters      = 0,
    _recursions   = 0;
    _object       = NULL;
    _owner        = NULL;
    // 处于wait状态的线程,会被加入到WaitSet
    _WaitSet      = NULL; 
    _WaitSetLock  = 0 ;
    _Responsible  = NULL ;
    _succ         = NULL ;
    _cxq          = NULL ;
    FreeNext      = NULL ;
    // 处于等待锁block状态的线程,会被加入到该列表
    _EntryList    = NULL ; 
    _SpinFreq     = 0 ;
    _SpinClock    = 0 ;
    OwnerIsThread = 0 ;
}
\end{lstlisting}

Synchronized可以把任何一个非null对象作为"锁",在HotSpot JVM实现中,锁有个专门的名字:对象监视器(Object Monitor)。Synchronized的语义底层是通过一个monitor的对象来完成,其实wait/notify等方法也依赖于monitor对象,这就是为什么只有在同步的块或者方法中才能调用wait/notify等方法,否则会抛出java.lang.IllegalMonitorStateException的异常的原因。当方法调用时,调用指令将会检查方法的 ACC\_SYNCHRONIZED 访问标志是否被设置,如果设置了,执行线程将先获取monitor,获取成功之后才能执行方法体,方法执行完后再释放monitor。两种同步方式本质上没有区别,只是方法的同步是一种隐式的方式来实现,无需通过字节码来完成。两个指令的执行是JVM通过调用操作系统的互斥原语mutex来实现,被阻塞的线程会被挂起、等待重新调度,会导致“用户态和内核态”两个态之间来回切换,对性能有较大影响。从JDK6开始,就对synchronized的实现机制进行了较大调整,包括使用JDK5引进的CAS自旋之外,还增加了自适应的CAS自旋、锁消除、锁粗化、偏向锁、轻量级锁这些优化策略。由于此关键字的优化使得性能极大提高,同时语义清晰、操作简单、无需手动关闭,所以推荐在允许的情况下尽量使用此关键字,同时在性能上此关键字还有优化的空间。

锁主要存在四种状态,依次是:无锁状态、偏向锁\footnote{大多数情况下锁不仅不存在多线程竞争,而且总是由同一线程多次获得。偏向锁的目的是在某个线程获得锁之后,消除这个线程锁重入(CAS)的开销,看起来让这个线程得到了偏护。另外,JVM对那种会有多线程加锁,但不存在锁竞争的情况也做了优化,听起来比较拗口,但在现实应用中确实是可能出现这种情况,因为线程之前除了互斥之外也可能发生同步关系,被同步的两个线程(一前一后)对共享对象锁的竞争很可能是没有冲突的。对这种情况,JVM用一个epoch表示一个偏向锁的时间戳(真实地生成一个时间戳代价还是蛮大的,因此这里应当理解为一种类似时间戳的identifier)。}状态、轻量级锁状态、重量级锁状态,锁可以从偏向锁升级到轻量级锁,再升级的重量级锁。但是锁的升级是单向的,也就是说只能从低到高升级,不会出现锁的降级。

在 JDK 1.6 中默认是开启偏向锁和轻量级锁的,可以通过-XX:-UseBiasedLocking来禁用偏向锁。在方法执行期间,其他任何线程都无法再获得同一个monitor对象。其中同步代码块是通过使用 monitorenter 和 monitorexit 实现的,而同步方法却是使用ACC\_SYNCHRONIZED标记符隐示的实现,原理是通过方法调用指令检查该方法在常量池中是否包含 ACC\_SYNCHRONIZED 标记符。一段使用了synchronized同步代码块的片段:

\begin{lstlisting}[language=Java]
/**
 * @author jiangtingqiang@gmail.com
 * @create 2019-05-08-19:21
 */
public class SynchronizedTest {

    public void test1(){
        synchronized (this){

        }
    }
}
\end{lstlisting}

用javap查看JVM中间代码:

\begin{lstlisting}
public void test1();
    Code:
       0: aload_0
       1: dup
       2: astore_1
       3: monitorenter
       4: aload_1
       5: monitorexit
       6: goto          14
       9: astore_2
      10: aload_1
      11: monitorexit //再加一个monitorexit,确保释放了锁
      12: aload_2
      13: athrow
      14: return
\end{lstlisting}

Java虚拟机中的同步(Synchronization)基于进入和退出管程(Monitor)对象实现,无论是显式同步(有明确的monitorenter和 monitorexit指令,即同步代码块)还是隐式同步都是如此\footnote{依据来自Java虚拟机规范:\url{https://docs.oracle.com/javase/specs/jvms/se8/html/jvms-3.html\#jvms-3.14}}。同步方法(用synchronization关键字修饰的方法)并不是由monitorenter和monitorexit指令来实现同步的,而是由方法调用指令读取运行时常量池中方法的ACC\_SYNCHRONIZED(Access Flags)标志来隐式实现\footnote{依据来自Java虚拟机规范:\url{https://docs.oracle.com/javase/specs/jvms/se8/html/jvms-3.html\#jvms-3.14}}.一段同步方法的代码块:

\begin{lstlisting}[language=Java]
/**
 * @author jiangtingqiang@gmail.com
 * @create 2019-05-08-19:21
 */
public class SynchronizedTest {

    public synchronized void test1(){

    }
}
\end{lstlisting}

javap查看中间代码:

\begin{lstlisting}
public synchronized void test1();
    descriptor: ()V
    flags: ACC_PUBLIC, ACC_SYNCHRONIZED
    Code:
      stack=0, locals=1, args_size=1
         0: return
      LineNumberTable:
        line 11: 0
      LocalVariableTable:
        Start  Length  Slot  Name   Signature
            0       1     0  this   Ldolphinweb/SynchronizedTest;
\end{lstlisting}

同步方法通过ACC\_SYNCHRONIZED(Access Synchronized)关键字隐式的对方法进行加锁。当线程要执行的方法被标注上ACC\_SYNCHRONIZED时,需要先获得锁才能执行该方法。同步代码块通过monitorenter和monitorexit执行来进行加锁。当线程执行到monitorenter的时候要先获得所锁,才能执行后面的方法。当线程执行到monitorexit的时候则要释放锁。每个对象自身维护这一个被加锁次数的计数器,当计数器数字为0时表示可以被任意线程获得锁。当计数器不为0时,只有获得锁的线程才能再次获得锁。即可重入锁\footnote{\url{https://www.hollischuang.com/archives/1883}}。

\paragraph{Synchronized锁静态方法和非静态方法区别}

synchronized修饰不加static的方法,锁是加在单个对象上,不同的对象没有竞争关系;修饰加了static的方法,锁是加载类上,这个类所有的对象竞争一把锁.



\end{document}