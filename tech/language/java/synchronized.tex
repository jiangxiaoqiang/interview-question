\documentclass[../../../interview-questions.tex]{subfiles}

\begin{document}

\subsection{synchronized的实现原理是什么}

其中同步代码块是通过使用 monitorenter 和 monitorexit 实现的,而同步方法却是使用ACC\_SYNCHRONIZED标记符隐示的实现,原理是通过方法调用指令检查该方法在常量池中是否包含 ACC\_SYNCHRONIZED 标记符。一段使用了synchronized同步代码块的片段:

\begin{lstlisting}[language=Java]
/**
 * @author jiangtingqiang@gmail.com
 * @create 2019-05-08-19:21
 */
public class SynchronizedTest {

    public void test1(){
        synchronized (this){

        }
    }
}
\end{lstlisting}

用javap查看JVM中间代码:

\begin{lstlisting}
public void test1();
    Code:
       0: aload_0
       1: dup
       2: astore_1
       3: monitorenter
       4: aload_1
       5: monitorexit
       6: goto          14
       9: astore_2
      10: aload_1
      11: monitorexit //再加一个monitorexit,确保释放了锁
      12: aload_2
      13: athrow
      14: return
\end{lstlisting}

Java虚拟机中的同步(Synchronization)基于进入和退出管程(Monitor)对象实现,无论是显式同步(有明确的monitorenter和 monitorexit指令,即同步代码块)还是隐式同步都是如此\footnote{依据来自Java虚拟机规范:\url{https://docs.oracle.com/javase/specs/jvms/se8/html/jvms-3.html\#jvms-3.14}}。同步方法并不是由monitorenter和monitorexit指令来实现同步的,而是由方法调用指令读取运行时常量池中方法的ACC\_SYNCHRONIZED(Access Flags)标志来隐式实现\footnote{依据来自Java虚拟机规范:\url{https://docs.oracle.com/javase/specs/jvms/se8/html/jvms-3.html\#jvms-3.14}}.一段同步方法的代码块:

\begin{lstlisting}[language=Java]
/**
 * @author jiangtingqiang@gmail.com
 * @create 2019-05-08-19:21
 */
public class SynchronizedTest {

    public synchronized void test1(){

    }
}
\end{lstlisting}

javap查看中间代码:

\begin{lstlisting}
public synchronized void test1();
    descriptor: ()V
    flags: ACC_PUBLIC, ACC_SYNCHRONIZED
    Code:
      stack=0, locals=1, args_size=1
         0: return
      LineNumberTable:
        line 11: 0
      LocalVariableTable:
        Start  Length  Slot  Name   Signature
            0       1     0  this   Ldolphinweb/SynchronizedTest;
\end{lstlisting}

同步方法通过ACC\_SYNCHRONIZED关键字隐式的对方法进行加锁。当线程要执行的方法被标注上ACC\_SYNCHRONIZED时,需要先获得锁才能执行该方法。同步代码块通过monitorenter和monitorexit执行来进行加锁。当线程执行到monitorenter的时候要先获得所锁,才能执行后面的方法。当线程执行到monitorexit的时候则要释放锁。每个对象自身维护这一个被加锁次数的计数器,当计数器数字为0时表示可以被任意线程获得锁。当计数器不为0时,只有获得锁的线程才能再次获得锁。即可重入锁\footnote{\url{https://www.hollischuang.com/archives/1883}}。


\end{document}