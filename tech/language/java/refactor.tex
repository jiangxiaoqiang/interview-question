\documentclass[../../../interview-questions.tex]{subfiles}

\begin{document}

\subsection{\color{red}{Java反射的实现原理}}

Java的反射(Reflection)机制是指在程序的运行状态中,可以构造任意一个类的对象,可以了解任意一个对象所属的类,可以了解任意一个类的成员变量和方法,可以调用任意一个对象的属性和方法。这种动态获取程序信息以及动态调用对象的功能称为Java语言的反射机制。反射(Reflection)被视为动态语言的关键。Class对象是在加载类时由JVM构造的,JVM为每个类管理一个独一无二的Class对象,这份Class对象里维护着该类的所有Method,Field,Constructor的cache,这份cache也可以被称作根对象。Java反射的原理就是获取Class对象然后使用java.lang.reflect里提供的方法操作Class对象,Class与java.lang.reflect构成了java的反射技术。反射在Java中可以直接调用,不过最终调用的仍是Native方法(严谨的说法是在调用较少次数时,调用Native方法,超过一定次数时,调用字节码的实现方式)。Class.forName可以通过包名寻找Class对象,比如Class.forName("java.lang.String")。在JDK的源码实现中,可以发现最终调用的是native方法forName0(),它在JVM中调用的实际是findClassFromClassLoader()。

\begin{lstlisting}[language=Java]
private static native Class<?> forName0(String name, boolean initialize,
                                            ClassLoader loader,
                                            Class<?> caller)
        throws ClassNotFoundException;
\end{lstlisting}

实际的MethodAccessor实现有两个版本,一个是Java实现的,另一个是Native Code实现的。查看JDK(我这里看的是Azul 15.0.5版本的源代码,其他版本或者其他厂商的源代码可能会略有区别,截止到公园2022年04月份,OpenJDK还没有发布支持macOS ARM 64bit的版本,但最新版本的JDK 19可以在macOS M1类型芯片的机器上编译)的源码,MethodAccessor的实现有Natvie实现类NativeMethodAccessorImpl和MethodAccessorImpl实现类,在NativeMethodAccessorImpl类里注释明确说明了,开始一定次数调用Native实现,超过阈值后,调用Java字节码实现方式:Used only for the first few invocations of a Method; afterward, switches to bytecode-based implementation。Java字节码实现(Bytecode-based Implementations)的版本在初始化时需要较多时间,但长久来说性能较好;native版本正好相反,启动时相对较快,但运行时间长了之后速度就比不过Java版了。这是HotSpot的优化方式带来的性能特性,同时也是许多虚拟机的共同点:跨越native边界会对优化有阻碍作用,它就像个黑箱一样让虚拟机难以分析并将其内联,于是运行时间长了之后反而是托管版本的代码更快些。
为了权衡两个版本的性能,Sun的JDK使用了“inflation”的技巧:让Java方法在被反射调用时,开头若干次使用native版,等反射调用次数超过阈值时则生成一个专用的MethodAccessor实现类,生成其中的invoke()方法的字节码,以后对该Java方法的反射调用就会使用Java版\footnote{参见:\url{https://www.iteye.com/blog/rednaxelafx-548536}}。具体Java版和Native版本的实现分析参见这里\footnote{\url{https://www.iteye.com/blog/rednaxelafx-548536}}。网上许多分析文章写到这里就到此为止了,一般理解到这个程度可以应对一般的面试。但如果想要看看Native到底是怎么实现反射的,可以调试跟踪JVM虚拟机的C++代码,看看到底是怎么一步步的实现反射的。

\paragraph{反射的Native实现(Native Implementations)}

当Java调用反射时,默认调用Native的实现方式。Natvie是在JVM层面实现的,具体是怎么实现的呢?NativeMethodAccessorImpl类中关键的invoke0()方法是个native方法。它在HotSpot VM里是由JVM\_InvokeMethod()函数所支持的。在JVM的入口是/jdk/src/share/native/sun/reflect/目录下的NativeAccessors.c。

\begin{lstlisting}[language=C]
JNIEXPORT jobject JNICALL Java_jdk_internal_reflect_NativeMethodAccessorImpl_invoke0
(JNIEnv *env, jclass unused, jobject m, jobject obj, jobjectArray args)
{
        return JVM_InvokeMethod(env, m, obj, args);
}
\end{lstlisting}

为什么Java的调用会直接调用JVM的C代码呢?两种不同的语言。要理解这点,需要了解JNI(Java Native Interface)技术。JNI全称为Java Native Interface. 它可以简单理解为是本地方法的接口,即允许在Java虚拟机里面的Java代码可以和如C,C++等其他底层语言进行交互(即可互相调用)。一般情况下,当你无法用纯Java来实现需求的时候,就需要使用JNI来用底层语言编写的本地方法来满足这些该需求。
例如以下的几种场景可能需要用到JNI:

\begin{enumerate}
        \item {Java库无法提供基于平台系统相关特性的功能,如平台特有的功能或接口等。}
        \item{已经用其他语言写好的库,需要用Java调用,而不想重现编码,而想直接复用它们。}
        \item {希望实现一部分时间和性能要求都很高的逻辑,需要用底层语言来编写,如视频图片处理,游戏逻辑等。}
\end{enumerate}

使用JNI,你可以使用本地方法来:

\begin{enumerate}
        \item {创造,交互,更新Java对象(包括array和string)}
        \item {调用Java方法。}
        \item {catch或抛出异常。}
        \item {载入Java Class,获取class信息。}
        \item {执行运行时类型检查(runtime type checking)}
\end{enumerate}

当你使用JNI的 Invocation API 可以允许任何本地应用内嵌Java虚拟机,这允许程序员非常简单的可以使已经写好的程序变成 Java-enabled , 而不需要链接到虚拟机源码。当然你还可以使用JNI来使得Java方法可以直接调用本地方法。本地方法也可以调用Java方法。关键的语句是:

\begin{lstlisting}[language=C++]
// 获取Class的元信息
oop mirror             = java_lang_reflect_Method::clazz(method_mirror);
// 实例化Class
InstanceKlass* klass = InstanceKlass::cast(java_lang_Class::as_Klass(mirror));
// 反射Class的方法
Method* m = klass->method_with_idnum(slot);
\end{lstlisting}

由于 mirror 也是一个 instanceKlass,所以它包含了 instanceKlass 所包含的一切字段。
执行 new A() 的时候,JVM native 层里发生了什么。首先,如果这个类没有被加载过,JVM 就会进行类的加载,并在 JVM 内部创建一个 instanceKlass 对象表示这个类的运行时元数据(相当于 Java 层的 Class 对象)。到初始化的时候(执行 invokespecial A::),JVM 就会创建一个 instanceOopDesc 对象表示这个对象的实例,然后进行 Mark Word 的填充,将元数据指针指向 Klass 对象,并填充实例变量。根据对 JVM 的理解,我们可以想到,元数据 —— instanceKlass 对象会存在元空间(方法区),而对象实例 —— instanceOopDesc 会存在 Java 堆。Java 虚拟机栈中会存有这个对象实例的引用。oop 是 ordinary object pointer (普通对象指针), 它用来表示对象的实例信息 (Java 类实例对象中各个属性在运行期的值)。看起来像是一个指针,而实际上对象实例数据都藏在指针所指向的内存首地址后面的一篇内存区域中\footnote{内容来源:\url{https://generalthink.github.io/2021/04/27/view-class-in-jvm-with-HSDB/}}。klass包含元数据和方法信息,用来描述 Java 类或者 JVM 内部自带的 C++ 类型信息。比如 Java 类的继承信息、成员变量、静态变量、成员方法、构造函数等信息都在 klass 中保存,JVM 根据这个
可以在运行期反射出 Java 类的全部结构信息。跟踪反射方法的代码,发现直接是从类的属性中获取的,类的属性什么时候put进去的呢,跟踪可知在Define类的时候。具体是什么时候定义类的呢?需进一步分析。为什么开始几次使用Natvie方式来实现反射呢?动态字节码实现(Bytecode-based Implementations)和本地实现(Native Implementations)相比,其运行效率要快上 20 倍。这是因为动态实现无需经过 Java 到 C++ 再到 Java 的切换,但由于生成字节码十分耗时,仅调用一次的话,反而是本地实现要快上 3 到 4 倍\footnote{性能数据来源于JDK源码的注释:\url{http://hg.openjdk.java.net/jdk10/jdk10/jdk/file/777356696811/src/java.base/share/classes/jdk/internal/reflect/ReflectionFactory.java\#l80}}。Native方式在少量次数的反射时,性能要优于字节码方式实现。如果次数超过阈值,使用字节码方式来实现反射更具备优势。可以写一个Benchmark来验证,不过应该没有这个必要。

\paragraph{反射的Java字节码方式实现(Bytecode-based Implementations)}

除了Native方式实现(Native Implementations)反射,Java 的反射调用机制还设立了动态生成字节码的实现(Bytecode-based Implementations),直接使用 invoke 指令来调用目标方法。关键代码在NativeConstructorAccessorImpl类中:

\begin{lstlisting}[language=Java]
if (++numInvocations > ReflectionFactory.inflationThreshold()
        && !c.getDeclaringClass().isHidden()
        && !ReflectUtil.isVMAnonymousClass(c.getDeclaringClass())) {
    ConstructorAccessorImpl acc = (ConstructorAccessorImpl)
        new MethodAccessorGenerator().
            generateConstructor(c.getDeclaringClass(),
                                c.getParameterTypes(),
                                c.getExceptionTypes(),
                                c.getModifiers());
    parent.setDelegate(acc);
}
\end{lstlisting}

% -Dsun.reflect.inflationThreshold=自动换行的解决方案:https://tex.stackexchange.com/questions/596714/how-to-auto-start-a-new-line-when-text-too-long
调用次数超过边界(Threshold)的时候采用字节码方式生成,inflationThreshold的边界默认是15。考虑到许多反射调用仅会执行一次,Java 虚拟机设置了一个阈值 15(可以通过 \url{-Dsun.reflect.inflationThreshold=} 来调整),当某个反射调用的调用次数在 15 之下时,采用本地实现;当达到 15 时,便开始动态生成字节码,并将委派实现的委派对象切换至动态实现,这个过程我们称之为 Inflation。

\end{document}