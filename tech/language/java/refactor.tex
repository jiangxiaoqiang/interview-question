\documentclass[../../../interview-questions.tex]{subfiles}

\begin{document}

\subsection{\color{red}{Java反射的实现原理}}

Java的反射(Reflection)机制是指在程序的运行状态中,可以构造任意一个类的对象,可以了解任意一个对象所属的类,可以了解任意一个类的成员变量和方法,可以调用任意一个对象的属性和方法。这种动态获取程序信息以及动态调用对象的功能称为Java语言的反射机制。反射(Reflection)被视为动态语言的关键。Class对象是在加载类时由JVM构造的,JVM为每个类管理一个独一无二的Class对象,这份Class对象里维护着该类的所有Method,Field,Constructor的cache,这份cache也可以被称作根对象。Java反射的原理就是获取Class对象然后使用java.lang.reflect里提供的方法操作Class对象,Class与java.lang.reflect构成了java的反射技术。反射在Java中可以直接调用,不过最终调用的仍是Native方法(严谨的说法是在调用较少次数时,调用Native方法,超过一定次数时,调用字节码的实现方式)。Class.forName可以通过包名寻找Class对象,比如Class.forName("java.lang.String")。在JDK的源码实现中,可以发现最终调用的是native方法forName0(),它在JVM中调用的实际是findClassFromClassLoader()。

\begin{lstlisting}[language=Java]
private static native Class<?> forName0(String name, boolean initialize,
                                            ClassLoader loader,
                                            Class<?> caller)
        throws ClassNotFoundException;
\end{lstlisting}

实际的MethodAccessor实现有两个版本,一个是Java实现的,另一个是Native Code实现的。查看JDK(我这里看的是Azul 15.0.5版本的源代码,其他版本或者其他厂商的源代码可能会略有区别,截止到公园2022年04月份,OpenJDK还没有发布支持macOS ARM 64bit的版本,但最新版本的JDK 19可以在macOS M1类型芯片的机器上编译)的源码,MethodAccessor的实现有Natvie实现类NativeMethodAccessorImpl和MethodAccessorImpl实现类,在NativeMethodAccessorImpl类里注释明确说明了,开始一定次数调用Native实现,超过阈值后,调用Java字节码实现方式:Used only for the first few invocations of a Method; afterward, switches to bytecode-based implementation。Java实现的版本在初始化时需要较多时间,但长久来说性能较好;native版本正好相反,启动时相对较快,但运行时间长了之后速度就比不过Java版了。这是HotSpot的优化方式带来的性能特性,同时也是许多虚拟机的共同点:跨越native边界会对优化有阻碍作用,它就像个黑箱一样让虚拟机难以分析并将其内联,于是运行时间长了之后反而是托管版本的代码更快些。
为了权衡两个版本的性能,Sun的JDK使用了“inflation”的技巧:让Java方法在被反射调用时,开头若干次使用native版,等反射调用次数超过阈值时则生成一个专用的MethodAccessor实现类,生成其中的invoke()方法的字节码,以后对该Java方法的反射调用就会使用Java版\footnote{参见:\url{https://www.iteye.com/blog/rednaxelafx-548536}}。具体Java版和Native版本的实现分析参见这里\footnote{\url{https://www.iteye.com/blog/rednaxelafx-548536}}。网上许多分析文章写到这里就到此为止了,一般理解到这个程度可以应对一般的面试。但如果想要看看Native到底是怎么实现反射的,可以调试跟踪JVM虚拟机的C++代码,看看到底是怎么一步步的实现反射的。

\paragraph{反射的Native实现}

当Java调用反射时,默认调用Native的实现方式。Natvie是在JVM层面实现的,具体是怎么实现的呢?NativeMethodAccessorImpl类中关键的invoke0()方法是个native方法。它在HotSpot VM里是由JVM\_InvokeMethod()函数所支持的。在JVM的入口是/jdk/src/share/native/sun/reflect/目录下的NativeAccessors.c。

\begin{lstlisting}[language=C]
JNIEXPORT jobject JNICALL Java_jdk_internal_reflect_NativeMethodAccessorImpl_invoke0
(JNIEnv *env, jclass unused, jobject m, jobject obj, jobjectArray args)
{
        return JVM_InvokeMethod(env, m, obj, args);
}
\end{lstlisting}

为什么Java的调用会直接调用JVM的C代码呢?两种不同的语言。要理解这点,需要了解JNI(Java Native Interface)技术。JNI全称为Java Native Interface. 它可以简单理解为是本地方法的接口,即允许在Java虚拟机里面的Java代码可以和如C,C++等其他底层语言进行交互(即可互相调用)。一般情况下,当你无法用纯Java来实现需求的时候,就需要使用JNI来用底层语言编写的本地方法来满足这些该需求。
例如以下的几种场景可能需要用到JNI:

\begin{enumerate}
        \item {Java库无法提供基于平台系统相关特性的功能,如平台特有的功能或接口等。}
        \item{已经用其他语言写好的库,需要用Java调用,而不想重现编码,而想直接复用它们。}
        \item {希望实现一部分时间和性能要求都很高的逻辑,需要用底层语言来编写,如视频图片处理,游戏逻辑等。}
\end{enumerate}

使用JNI,你可以使用本地方法来:

\begin{enumerate}
        \item {创造,交互,更新Java对象(包括array和string)}
        \item {调用Java方法。}
        \item {catch或抛出异常。}
        \item {载入Java Class,获取class信息。}
        \item {执行运行时类型检查(runtime type checking)}
\end{enumerate}

当你使用JNI的 Invocation API 可以允许任何本地应用内嵌Java虚拟机,这允许程序员非常简单的可以使已经写好的程序变成 Java-enabled , 而不需要链接到虚拟机源码。当然你还可以使用JNI来使得Java方法可以直接调用本地方法。本地方法也可以调用Java方法。关键的语句是:

\begin{lstlisting}[language=C++]
// 获取Class的元信息
oop mirror             = java_lang_reflect_Method::clazz(method_mirror);
// 实例化Class
InstanceKlass* klass = InstanceKlass::cast(java_lang_Class::as_Klass(mirror));
// 反射Class的方法
Method* m = klass->method_with_idnum(slot);
\end{lstlisting}

跟踪反射方法的代码,发现直接是从类的属性中获取的,类的属性什么时候put进去的呢,跟踪可知在Define类的时候。具体是什么时候定义类的呢?需进一步分析。为什么开始几次使用Natvie方式来实现反射呢?应该是Native方式在少量次数的反射时,性能要优于字节码方式实现。如果次数超过阈值,使用字节码方式来实现反射更具备优势。有时间可以写一个Benchmark来验证。

\paragraph{反射的Java字节码方式实现}

\end{document}