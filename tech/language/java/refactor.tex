\documentclass[../../../interview-questions.tex]{subfiles}

\begin{document}

\subsection{反射的实现原理}

反射在Java中可以直接调用,不过最终调用的仍是native方法。Class.forName可以通过包名寻找Class对象,比如Class.forName("java.lang.String")。在JDK的源码实现中,可以发现最终调用的是native方法forName0(),它在JVM中调用的实际是findClassFromClassLoader()。

\begin{lstlisting}[language=Java]
private static native Class<?> forName0(String name, boolean initialize,
                                            ClassLoader loader,
                                            Class<?> caller)
        throws ClassNotFoundException;
\end{lstlisting}

网上许多分析文章写到这里就到此为止了。尝试调试Hotspot的原始码,看看反射到底是怎么实现的。首先下载JDK11 Hotspot的源代码:

\begin{lstlisting}[language=Bash]
wget -c https://hg.openjdk.java.net/jdk-updates/jdk11u/archive/tip.tar.gz    
\end{lstlisting}








\end{document}