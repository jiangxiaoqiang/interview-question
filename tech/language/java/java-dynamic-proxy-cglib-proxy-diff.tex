\documentclass[../../../interview-questions.tex]{subfiles}

\begin{document}

\subsection{Java的动态代理(Dynamic Proxy)和CGLib有什么差别?}

JDK代理只能对实现接口的类生成代理;CGLib是针对类实现代理,对指定的类生成一个子类,并覆盖其中的方法,这种通过继承类的实现方式,不能代理final修饰的类。

\textbf{JDK代理使用的是反射机制实现AOP的动态代理,CGLib代理使用字节码处理框架ASM,通过修改字节码生成子类。所以JDK动态代理的方式创建代理对象效率较高,执行效率较低,CGLib创建效率较低,执行效率高。}

JDK动态代理机制是委托机制,具体说动态实现接口类,在动态生成的实现类里面委托hanlder去调用原始实现类方法,CGLib则使用的继承机制,具体说被代理类和代理类是继承关系,所以代理类是可以赋值给被代理类的,如果被代理类有接口,那么代理类也可以赋值给接口。

\end{document}