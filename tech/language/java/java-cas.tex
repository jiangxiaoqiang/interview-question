\documentclass[../../../interview-questions.tex]{subfiles}

\begin{document}

\subsection{Java CAS原理}

CAS(比较与交换,Compare and Swap)是一种无锁算法。无锁编程,即不使用锁的情况下实现多线程之间的变量同步,也就是在没有线程被阻塞的情况下实现变量的同步,所以也叫非阻塞同步(Non-blocking Synchronization)。实现非阻塞同步的方案称为“无锁编程算法”( Non-blocking algorithm)\footnote{\url{https://www.cnblogs.com/zhuawang/p/4196904.html}},java.util.concurrent包全完建立在CAS之上,CAS又建立在unsafe之上,而compareAndSet利用JNI(Java Native Interface)来完成CPU指令的操作。JNI利用了CPU本身对原子操作的支持,现在几乎所有的CPU指令都支持CAS的原子操作,X86下对应的是 CMPXCHG 汇编指令。有了这个原子操作,我们就可以用其来实现各种无锁(lock free)的数据结构\footnote{\url{https://coolshell.cn/articles/8239.html}}。

\begin{lstlisting}[language=Java]
public final boolean compareAndSet(int expect, int update) {
    return unsafe.compareAndSwapInt(this, valueOffset, expect, update);
}
\end{lstlisting}

Java无法直接访问底层操作系统,而是通过本地(native)方法来访问。不过尽管如此,JVM还是开了一个后门,JDK中有一个类Unsafe,它提供了硬件级别的原子操作,CAS运用了这个特性。CAS的逻辑漏洞:如果一个变量V初次读取的时候是A值,并且在准备赋值的时候检查到它仍然是A值,那我们就能说明它的值没有被其他线程修改过了吗?如果在这段期间它的值曾经被改成了B,然后又改回A,那CAS操作就会误认为它从来没有被修改过。这个漏洞称为CAS操作的"ABA"问题。java.util.concurrent包为了解决这个问题,提供了一个带有标记的原子引用类"AtomicStampedReference",它可以通过控制变量值的版本来保证CAS的正确性。

\end{document}