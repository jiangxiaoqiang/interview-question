\documentclass[../../../interview-questions.tex]{subfiles}

\begin{document}

\subsection{Spring Cloud Hystrix}

Hystrix [hɪst'rɪks],中文含义是豪猪,因其背上长满棘刺,从而拥有了自我保护的能力。本文所说的Hystrix是Netflix开源的一款容错框架,同样具有自我保护能力。Netflix Hystrix是SOA/微服务架构中提供服务隔离、熔断、降级机制的工具/框架。Netflix Hystrix是断路器的一种实现,用于高微服务架构的可用性,是防止服务出现雪崩的利器。


\paragraph{资源隔离}

资源隔离主要指对线程的隔离。Hystrix提供了两种线程隔离方式:线程池和信号量。

\paragraph{熔断}

Hystrix中的熔断器(Circuit Breaker)也是起类似作用,Hystrix在运行过程中会向每个commandKey对应的熔断器报告成功、失败、超时和拒绝的状态,熔断器维护并统计这些数据,并根据这些统计信息来决策熔断开关是否打开。如果打开,熔断后续请求,快速返回。隔一段时间(默认是5s)之后熔断器尝试半开,放入一部分流量请求进来,相当于对依赖服务进行一次健康检查,如果请求成功,熔断器关闭。


\paragraph{降级}

Fail Fast 快速失败,快速失败是最普通的命令执行方法,命令没有重写降级逻辑。 如果命令执行发生任何类型的故障,它将直接抛出异常。

Fail Silent 无声失败,指在降级方法中通过返回null,空Map,空List或其他类似的响应来完成。

https://juejin.cn/post/6844903912869199885

\paragraph{降级方案}

\begin{itemize}
    \item {超时降级} 配置好超时时间和超时重试次数和机制,并使用异步机制探测恢复情况。
    \item {失败次数降级} 失败次数达到阈值自动降级,同样要使用异步机制探测回复情况。
    \item {故障降级} 如要调用的远程服务挂掉了(网络故障、DNS 故障、HTTP 服务返回错误的状态码和 RPC 服务抛出异常),则可以直接降级。
    和上面那种失败次数降级原理类似,只是需要区分错误类型。
    \item {限流降级} 单位时间内调用次数超过阈值,可以使用暂时屏蔽的方式来进行短暂的屏蔽。
\end{itemize}




\end{document}










