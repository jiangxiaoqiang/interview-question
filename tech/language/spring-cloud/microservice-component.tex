\documentclass[../../../interview-questions.tex]{subfiles}

\begin{document}

\subsection{微服务组件}

微服务架构下,服务调用主要依赖下面几个基本组件:

\begin{enumerate}
    \item {注册中心(Registry):注册并维护远程服务及服务提供者的地址,供服务消费者发现和调用,为保证可用性,通常基于分布式 kv 存储器来实现,比如 zookeeper、etcd、Eureka(已过时)、Consul 等;}
    \item {服务框架:用于实现微服务的 RPC 框架,包含服务接口描述及实现方案、向注册中心发布服务等功能,常见的 RPC 框架包括 Spring Cloud、Dubbo、gRPC、 Thrift、go-micro 等;}
    \item {服务网关(Gateway):介于客户端与微服务之间的网关层,可以理解为「门卫」的角色,以确保服务提供者对客户端的透明,这一层可以进行反向路由、安全认证、灰度发布、日志监控等前置动作;}如Zuul、Spring Cloud Gateway等。
    \item {服务监控:对服务消费者与提供者之间的调用情况进行监控和数据展示;}
    \item {服务追踪:记录对每个请求的微服务调用完整链路,以便进行问题定位和故障分析;}
    \item {服务治理:服务治理就是通过一系列的手段来保证在各种意外情况下,服务调用仍然能够正常进行,这些手段包括熔断、隔离、限流、降级、负载均衡等。}
    \item {基础设施:分布式消息队列、日志存储、数据库、缓存、文件服务器、搜索集群等,用以提供服务底层的基础数据服务,可以自建,也可以使用阿里云等公有云提供的服务。}
\end{enumerate}

\end{document}

