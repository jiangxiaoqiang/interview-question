\documentclass[../../../interview-questions.tex]{subfiles}

\begin{document}

\subsection{Docker容器间通信}

\paragraph{Docker的网络驱动模型}

Docker的网络驱动模型分类:

\begin{enumerate}
    \item {bridge:Docker中默认的网络驱动模型,在启动容器时如果不指定则默认为此驱动类型;}
    \item {host:打破Docker容器与宿主机之间的网络隔离,直接使用宿主机的网络环境,该模型仅适用于Docker17.6及以上版本;}
    \item {overlay:可以连接多个docker守护进程或者满足集群服务之间的通信;适用于不同宿主机上的docker容器之间的通信;}
    \item {macvlan:可以为docker容器分配MAC地址,使其像真实的物理机一样运行;}
    \item {none:即禁用了网络驱动,需要自己手动自定义网络驱动配置;}
    \item {plugins:使用第三方网络驱动插件;}
\end{enumerate}

以上是Docker支持的几种网络驱动模型,它们都具有独特的特点和应用范围,为了更加详细地了解Docker的网络运行原理,下面挑选几种较为重要的网络模型进行研究。

\paragraph{跨主机通信}

跨主机通信一般通过隧道(Tunnel)或者路由(Route)的方式,隧道方式代表性的是Fannel方案,路由方式是Calico方案。Calico通信。Calico的原理是通过修改每个主机节点上的iptables和路由表规则,实现容器间数据路由和访问控制,并通过Etcd协调节点配置信息的。因此Calico服务本身和许多分布式服务一样,需要运行在集群的每一个节点上。
在默认配置下每台宿主机的BGPClient需要和集群所有的BGPClient建立连接,进行路由信息交换,随着集群规模的扩大,集群的网络将会面临巨大的压力并且宿主机的路由表也会变的过大。所以在大规模的集群中,通常使用BGP Route Reflector充当BGP客户端连接的中心点,从而避免与互联网中的每个BGP客户端进行通信。Calico使用BGP Route Reflector是为了减少给定一个BGP客户端与集群其他BGP客户端的连接。用户也可以同时部署多个BGP Route Reflector服务实现高可用。Route Reflector仅仅是协助管理BGP网络,并没有工作负载的数据包经过它们。https://www.cnblogs.com/v-fan/p/14452770.html

\end{document}