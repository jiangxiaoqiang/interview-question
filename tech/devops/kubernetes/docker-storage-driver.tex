\documentclass[../../../interview-questions.tex]{subfiles}

\begin{document}

\subsection{Docker存储驱动}

docker提供了多种存储驱动来实现不同的方式存储镜像,下面是常用的几种存储驱动:

AUFS
OverlayFS
Devicemapper
Btrfs
ZFS

AUFS(AnotherUnionFS)是一种Union FS,是文件级的存储驱动。AUFS是一个能透明覆盖一个或多个现有文件系统的层状文件系统,把多层合并成文件系统的单层表示。简单来说就是支持将不同目录挂载到同一个虚拟文件系统下的文件系统。这种文件系统可以一层一层地叠加修改文件。无论底下有多少层都是只读的,只有最上层的文件系统是可写的。当需要修改一个文件时,AUFS创建该文件的一个副本,使用CoW将文件从只读层复制到可写层进行修改,结果也保存在可写层。在Docker中,底下的只读层就是image,可写层就是Container。


Overlay是Linux内核3.18后支持的,也是一种Union FS,和AUFS的多层不同的是Overlay只有两层:一个upper文件系统和一个lower文件系统,分别代表Docker的镜像层和容器层。当需要修改一个文件时,使用CoW将文件从只读的lower复制到可写的upper进行修改,结果也保存在upper层。在Docker中,底下的只读层就是image,可写层就是Container。目前最新的OverlayFS为Overlay2。

Device mapper是Linux内核2.6.9后支持的,提供的一种从逻辑设备到物理设备的映射框架机制,在该机制下,用户可以很方便的根据自己的需要制定实现存储资源的管理策略。前面讲的AUFS和OverlayFS都是文件级存储,而Device mapper是块级存储,所有的操作都是直接对块进行操作,而不是文件。Device mapper驱动会先在块设备上创建一个资源池,然后在资源池上创建一个带有文件系统的基本设备,所有镜像都是这个基本设备的快照,而容器则是镜像的快照。所以在容器里看到文件系统是资源池上基本设备的文件系统的快照,并没有为容器分配空间。当要写入一个新文件时,在容器的镜像内为其分配新的块并写入数据,这个叫用时分配。当要修改已有文件时,再使用CoW为容器快照分配块空间,将要修改的数据复制到在容器快照中新的块里再进行修改。Device mapper 驱动默认会创建一个100G的文件包含镜像和容器。每一个容器被限制在10G大小的卷内,可以自己配置调整。

AUFS VS OverlayFS
AUFS和Overlay都是联合文件系统,但AUFS有多层,而Overlay只有两层,所以在做写时复制操作时,如果文件比较大且存在比较低的层,则AUSF可能会慢一些。而且Overlay并入了linux kernel mainline,AUFS没有。目前AUFS已基本被淘汰

OverlayFS VS Device mapper
OverlayFS是文件级存储,Device mapper是块级存储,当文件特别大而修改的内容很小,Overlay不管修改的内容大小都会复制整个文件,对大文件进行修改显示要比小文件要消耗更多的时间,而块级无论是大文件还是小文件都只复制需要修改的块,并不是整个文件,在这种场景下,显然device mapper要快一些。因为块级的是直接访问逻辑盘,适合IO密集的场景。而对于程序内部复杂,大并发但少IO的场景,Overlay的性能相对要强一些。

\end{document}