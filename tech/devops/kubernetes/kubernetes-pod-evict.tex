\documentclass[../../../interview-questions.tex]{subfiles}

\begin{document}

\subsection{Kubernetes Pod驱逐分析}

通过kubectl命令查看Pod驱逐日志:

\begin{lstlisting}[language=Bash]
kubectl describe pod admin-service-6b684499f6-4xnkb -n reddwarf-pro    
\end{lstlisting}

从日志中可以看到驱逐的原因:Pod The node had condition: [DiskPressure],说明是由于磁盘空间不足导致驱逐。通过duc工具,找到占用磁盘较大的目录是:

\begin{lstlisting}[language=Bash]
/var/lib/kubelet/pods/8aafe99f-53c1-4bec-8cb8-abd09af1448f/volumes/kubernetes.io~nfs/nfs-reddwarf-postgresql-pv-general/
\end{lstlisting}

但是此目录是通过NFS挂载在NAS上的,为什么会占用空间?实际上此目录是Pod持久卷的挂载点,实际不占用空间。如何确定不占用空间呢?

\begin{lstlisting}[language=Bash]
[root@k8smasterone volumes]# df|grep "general"
xxx.cn-shanghai.nas.aliyuncs.com:/k8s/reddwarf-storage/postgresql          10995116277760  8651776 10995107625984   1% /var/lib/kubelet/pods/8aafe99f-53c1-4bec-8cb8-abd09af1448f/volumes/kubernetes.io~nfs/nfs-reddwarf-postgresql-pv-general
\end{lstlisting}

从df的命令可以看出,路径\url{/var/lib/kubelet/pods/8aafe99f-53c1-4bec-8cb8-abd09af1448f/volumes/kubernetes.io~nfs/nfs-reddwarf-postgresql-pv-general}实际是指向阿里云NAS实际的挂载路径\url{xxx.cn-shanghai.nas.aliyuncs.com:/k8s/reddwarf-storage/postgresql}。

\paragraph{优化磁盘空间}






\end{document}