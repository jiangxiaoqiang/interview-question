\documentclass[../../../interview-questions.tex]{subfiles}

\begin{document}

\subsection{Calico通信原理}

Calico是一个纯三层的协议,为OpenStack虚机和Docker容器提供多主机间通信。Calico不使用重叠网络比如flannel和libnetwork重叠网络驱动,使用虚拟路由代替虚拟交换,每一台虚拟路由通过BGP协议传播可达信息(路由)到剩余数据中心。Calico网络模型主要工作组件:

\begin{itemize}
    \item{Felix:calico的核心组件,运行在每个节点上。主要的功能有接口管理、路由规则、ACL规则和状态报告,Felix会监听ECTD中心的存储,从它获取事件,比如说用户在这台机器上加了一个IP,或者是创建了一个容器等。用户创建pod后,Felix负责将其网卡、IP、MAC都设置好,然后在内核的路由表里面写一条,注明这个IP应该到这张网卡。同样如果用户制定了隔离策略,Felix同样会将该策略创建到ACL中,以实现隔离。}
    \item {etcd:分布式键值存储,主要负责网络元数据一致性,确保Calico网络状态的准确性,可以与kubernetes共用}
    \item {BGP Client(BIRD):Calico 为每一台 Host 部署一个 BGP Client,它的作用是将Felix的路由信息读入内核,并通过BGP协议在集群中分发。当Felix将路由插入到Linux内核FIB中时,BGP客户端将获取这些路由并将它们分发到部署中的其他节点。这可以确保在部署时有效地路由流量}
    \item {BGP Router Reflector:大型网络仅仅使用 BGP client 形成 mesh 全网互联的方案就会导致规模限制,所有节点需要 $N^{2}$ 个连接,为了解决这个规模问题,可以采用 BGP 的 Router Reflector 的方法,使所有 BGP Client 仅与特定 RR 节点互联并做路由同步,从而大大减少连接数}
    \item {Calicoctl:calico 命令行管理工具}
\end{itemize}


\end{document}