\documentclass[../../../interview-questions.tex]{subfiles}

\begin{document}

\subsection{Kubernetes创建一个pod的详细流程}

(1) 客户端提交创建请求,可以通过 api-server 提供的 restful 接口,或者是通过 kubectl 命令行工具,支持的数据类型包括 JSON 和 YAML。

(2) api-server 处理用户请求,将 pod 信息存储至 etcd 中。

(3) kube-scheduler 通过 api-server 提供的接口监控到未绑定的 pod,尝试为 pod 分配 node 节点,主要分为两个阶段,预选阶段和优选阶段,其中预选阶段是遍历所有的 node 节点,根据策略筛选出候选节点,而优选阶段是在第一步的基础上,为每一个候选节点进行打分,分数最高者胜出。

(4) 选择分数最高的节点,进行 pod binding 操作,并将结果存储至 etcd 中。

(5) 随后目标节点的 kubelet 进程通过 api-server 提供的接口监测到 kube-scheduler 产生的 pod 绑定事件,然后从 etcd 获取 pod 清单,下载镜像并启动容器。

\end{document}