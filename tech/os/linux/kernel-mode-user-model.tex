\documentclass[../../../interview-questions.tex]{subfiles}

\begin{document}

\subsection{用户态与内核态}

在说用户态与内核态之前\footnote{参见:\url{https://segmentfault.com/a/1190000039774784}},有必要说一下 C P U 指令集,指令集是 C P U 实现软件指挥硬件执行的媒介,具体来说每一条汇编语句都对应了一条 C P U 指令,而非常非常多的 C P U 指令 在一起,可以组成一个、甚至多个集合,指令的集合叫 C P U 指令集。

同时 C P U 指令集 有权限分级,大家试想,C P U 指令集 可以直接操作硬件的,要是因为指令操作的不规范\`,造成的错误会影响整个计算机系统的。好比你写程序,因为对硬件操作不熟悉,导致操作系统内核、及其他所有正在运行的程序,都可能会因为操作失误而受到不可挽回的错误,最后只能重启计算机才行。

而对于硬件的操作是非常复杂的,参数众多,出问题的几率相当大,必须谨慎的进行操作,对开发人员来说是个艰巨的任务,还会增加负担,同时开发人员在这方面也不被信任,所以操作系统内核直接屏蔽开发人员对硬件操作的可能,都不让你碰到这些 C P U 指令集。


针对上面的需求,硬件设备商直接提供硬件级别的支持,做法就是对 C P U 指令集设置了权限,不同级别权限能使用的 C P U 指令集 是有限的,以 Inter C P U 为例,Inter把 C P U 指令集 操作的权限由高到低划为4级:

ring 0
ring 1
ring 2
ring 3

其中 ring 0 权限最高,可以使用所有 C P U 指令集,ring 3 权限最低,仅能使用常规 C P U 指令集,不能使用操作硬件资源的 C P U 指令集,比如 I O 读写、网卡访问、申请内存都不行,Linux系统仅采用ring 0 和 ring 3 这2个权限。

ring 0被叫做内核态,完全在操作系统内核中运行
ring 3被叫做用户态,在应用程序中运行

\end{document}