\documentclass[../../../interview-questions.tex]{subfiles}

\begin{document}

\subsection{进程程切换开销}


\subsection{线程切换开销}

在Linux下其实本并没有线程,只是为了迎合开发者口味,搞了个轻量级进程出来就叫做了线程。轻量级进程和进程一样,都有自己独立的task\_struct进程描述符,也都有自己独立的pid。从操作系统视角看,调度上和进程没有什么区别,都是在等待队列的双向链表里选择一个task\_struct切到运行态而已。只不过轻量级进程和普通进程的区别是可以共享同一内存地址空间、代码段、全局变量、同一打开文件集合而已。

我们都知道,线程切换会带来开销,如果频繁进行线程切换,所造成的开销是相当可观的。那么为什么线程切换会有开销呢,有哪些开销呢?这里涉及几个概念:CPU上下文切换、线程上下文切换、特权模式切换(内核态和用户态的互相转换)。

\paragraph{CPU上下文切换}

在多任务操作系统中,对于一个CPU而言,它并不是一直为一个任务服务直到任务结束的,而是在不同的任务之间切换,使得多个任务轮流使用CPU。而在每个任务运行前,CPU都需要知道任务从哪里加载、此时的状态、从哪里开始运行,也就是说,需要系统事先帮它设置好CPU寄存器和程序计数器,这些内容就是CPU上下文。

稍微详细描述一下,CPU上下文切换可以认为是内核在 CPU 上对于进程(包括线程)进行以下的活动:(1)挂起一个进程,将这个进程的CPU 上下文存储于内存中的某处,(2)在内存中检索下一个进程的上下文并将其在 CPU 的寄存器中恢复,(3)跳转到程序计数器所指向的位置(即跳转到进程被中断时的代码行),以恢复该进程。

CPU上下文切换的分为几个不同的场景:进程上下文切换,线程上下文切换,中断上下文切换

\paragraph{线程上下文切换}

当从一个线程切换到另一个线程时,不仅会发生线程上下文切换,还会发生特权模式切换。

首先,既然是线程切换,那么一定涉及线程状态的保存和恢复,包括寄存器、栈等私有数据。另外,线程的调度是需要内核级别的权限的(操作CPU和内存),也就是说线程的调度工作是在内核态完成的,因此会有一个从用户态到内核态的切换。而且,不管是线程本身的切换还是特权模式的切换,都要进行CPU的上下文切换。本质上都是从“一段程序”切换到了“另一段程序”,都要设置相应的CPU上下文。要明确一个问题,那就是内核也是有代码的,只是这些代码的机密性比较高,我们用户态无法访问。(要理清这几个概念的关系)

总结来说,线程切换过程包括:线程上下文的保存和恢复,用户态和内核态的转换,CPU上下文的切换,这些工作都需要CPU去完成,是一笔不小的开销。

\end{document}