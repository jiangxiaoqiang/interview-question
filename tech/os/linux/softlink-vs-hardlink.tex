\documentclass[../../../interview-questions.tex]{subfiles}

\begin{document}

\subsection{硬链接和软链接的区别}

链接为 Linux 系统解决了文件的共享使用,还带来了隐藏文件路径、增加权限安全及节省存储等好处。若一个 inode 号对应多个文件名,则称这些文件为硬链接。换言之,硬链接就是同一个文件使用了多个别名.软链接与硬链接不同,若文件用户数据块中存放的内容是另一文件的路径名的指向,则该文件就是软连接。软链接就是一个普通文件,只是数据块内容有点特殊。软链接有着自己的 inode 号以及用户数据块。

\textbf{硬连接(Hard Link)}。硬连接指通过索引节点来进行连接。在Linux的文件系统中,保存在磁盘分区中的文件不管是什么类型都给它分配一个编号,称为索引节点号(Node Index)。在Linux中,多个文件名指向同一索引节点是存在的。一般这种连接就是硬连接。硬连接的作用是允许一个文件拥有多个有效路径名,这样用户就可以建立硬连接到重要文件,以防止“误删”的功能。其原因如上所述,因为对应该目录的索引节点有一个以上的连接。只删除一个连接并不影响索引节点本身和其它的连接,只有当最后一个连接被删除后,文件的数据块及目录的连接才会被释放。也就是说,文件真正删除的条件是与之相关的所有硬连接文件均被删除。

\textbf{软链接(Soft Link)}。另外一种连接称之为符号连接(Symbolic Link),也叫软连接。软链接文件有类似于Windows的快捷方式。它实际上是一个特殊的文件。在符号连接中,文件实际上是一个文本文件,其中包含的有另一文件的位置信息。

\end{document}