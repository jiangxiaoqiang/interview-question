\documentclass[../../../interview-questions.tex]{subfiles}

\begin{document}

\subsection{\color{red}{容器三把斧之cgroup原理与实现}}

cgroup(Control Group)从2.6.4引入linux内核主线,目前默认已启用该特性。在cgroup出现之前,只能对一个进程做资源限制,比如通过sched\_setaffinity设置进程cpu亲和性,使用ulimit限制进程打开文件上限、栈大小等。cgroups是Linux下控制一个(或一组)进程的资源限制机制,全称是control groups,可以对cpu、内存等资源做精细化控制,比如目前很多的Docker在Linux下就是基于cgroups提供的资源限制机制来实现资源控制的;除此之外,开发者也可以指直接基于cgroups来进行进程资源控制,比如8核的机器上部署了一个web服务和一个计算服务,可以让web服务仅可使用其中6个核,把剩下的两个核留给计算服务。cgroups cpu限制除了可以限制使用多少/哪几个核心之外,还可以设置cpu占用比(注意占用比是各自都跑满情况下的使用比例,如果一个cgroup空闲而另一个繁忙,那么繁忙的cgroup是有可能占满整个cpu核心的)。

\paragraph{CGroup子系统}

cgroups为每种资源定义了一个子系统,典型的子系统如下:

cpu 子系统,主要限制进程的 cpu 使用率。
cpuacct 子系统,可以统计 cgroups 中的进程的 cpu 使用报告。
cpuset 子系统,可以为 cgroups 中的进程分配单独的 cpu 节点或者内存节点。
memory 子系统,可以限制进程的 memory 使用量。
blkio 子系统,可以限制进程的块设备 io。
devices 子系统,可以控制进程能够访问某些设备。
net\_cls 子系统,可以标记 cgroups 中进程的网络数据包,然后可以使用 tc 模块(traffic control)对数据包进行控制。
freezer 子系统,可以挂起或者恢复 cgroups 中的进程。
ns 子系统,可以使不同 cgroups 下面的进程使用不同的 namespace。
每个子系统都是定义了一套限制策略,它们需要与内核的其他模块配合来完成资源限制功能,比如对 cpu 资源的限制是通过进程调度模块根据 cpu 子系统的配置来完成的;对内存资源的限制则是内存模块根据 memory 子系统的配置来完成的,而对网络数据包的控制则需要 Traffic Control 子系统来配合完成。

\paragraph{CGroup的结构}

关于cgroups原理,可以从进程角度来剖析相关数据结构之间关系,Linux 下管理进程的数据结构是 task\_struct。

每个进程对应一个css\_set结构,css\_set存储了与进程相关的cgropus信息。cg\_list是一个嵌入的 list\_head 结构,用于将连到同一个 css\_set 的进程组织成一个链表。进程和css\_set的关系是多对一关系,tasks表示关联的多个进程。

在cgroups中一个task任务就是一个进程,一个进程可以加入到某个cgroup,也从一个进程组迁移到另一个cgroup。一个进程组的进程可以使用 cgroups 以控制族群为单位分配的资源,同时受到 cgroups 以控制族群为单位设定的限制。多个cgroup形成一个层级结构(树形结构),cgroup树上的子节点cgroup是父节点cgroup的孩子,继承父cgroup的特定的属性。注意:cgroups层级只会关联某个子系统之后才能进行对应的资源控制,一个子系统附加到某个层级以后,这个层级上的所有cgroup都受到这个子系统的控制。

\paragraph{限制进程使用资源}

当设置好 cgroup 的资源使用限制信息\footnote{\url{参见:https://jishuin.proginn.com/p/763bfbd62530}},并且把进程添加到这个 cgroup 的 tasks 列表后,进程的资源使用就会受到这个 cgroup 的限制。这里使用 内存子系统 作为例子,来分析一下内核是怎么通过 cgroup 来限制进程对资源的使用的。

当进程要使用内存时,会调用 do\_anonymous\_page() 来申请一些内存页,而 \url{do\_anonymous\_page()} 函数会调用 mem\_cgroup\_charge() 函数来检测进程是否超过了 cgroup 设置的资源限制。而 \url{mem\_cgroup\_charge()} 最终会调用 \url{mem\_cgroup\_charge\_common()} 函数进行检测,\url{mem\_cgroup\_charge\_common()} 函数实现如下:

\begin{lstlisting}[language=C]
static int mem_cgroup_charge_common(struct page *page, struct mm_struct *mm, gfp_t gfp_mask, enum charge_type ctype)
{
    struct mem_cgroup *mem;
    ...
    // 获取进程对应的内存限制对象
    mem = rcu_dereference(mm->mem_cgroup); 
    ...
    // 判断进程使用内存是否超出限制
    while (res_counter_charge(&mem->res, PAGE_SIZE)) { 
        if (!(gfp_mask & __GFP_WAIT))
            goto out;
        // 如果超出限制, 就释放一些不用的内存
        if (try_to_free_mem_cgroup_pages(mem, gfp_mask)) 
            continue;

        if (res_counter_check_under_limit(&mem->res))
            continue;

        if (!nr_retries--) {
            // 如果尝试过5次后还是超出限制, 那么发出oom信号
            mem_cgroup_out_of_memory(mem, gfp_mask); 
            goto out;
        }
        ...
    }
    ...
}
\end{lstlisting}

mem\_cgroup\_charge\_common() 函数会对进程内存使用情况进行检测,如果进程已经超过了 cgroup 设置的限制,那么就会尝试进行释放一些不用的内存,如果还是超过限制,那么就会发出 OOM (out of memory) 的信号。

\end{document}