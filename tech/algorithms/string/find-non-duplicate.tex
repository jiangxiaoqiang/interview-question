\documentclass[../../../interview-questions.tex]{subfiles}

\begin{document}

\subsection{找出不重复字符}

问题:给定一个字符串 s ,找到 它的第一个不重复的字符,并返回它的索引 。如果不存在,则返回 -1 。

可以使用哈希表来解决这个问题。首先,我们需要遍历字符串一次,将每个字符和它出现的次数存储在哈希表中。然后,我们再次遍历字符串,找到第一个出现次数为 1 的字符。

\begin{lstlisting}[language=Java]
public int firstUniqChar(String s) {
    // 创建哈希表
    Map<Character, Integer> charCount = new HashMap<>();
    for (char c : s.toCharArray()) {
        charCount.put(c, charCount.getOrDefault(c, 0) + 1);
    }
    
    // 找到第一个出现次数为 1 的字符
    for (int i = 0; i < s.length(); i++) {
        if (charCount.get(s.charAt(i)) == 1) {
            return i;
        }
    }
    
    return -1;
}    
\end{lstlisting}


该算法的时间复杂度为 O(n),其中 n 是字符串的长度。我们需要遍历字符串两次,每次遍历的时间复杂度都是 O(n)。空间复杂度为 O(k),其中 k 是不同字符的数量。在最坏情况下,k = n,即每个字符都不同,空间复杂度为 O(n)。


\end{document}