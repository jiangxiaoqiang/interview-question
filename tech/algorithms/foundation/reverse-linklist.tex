\documentclass[../../../interview-questions.tex]{subfiles}

\begin{document}

\subsubsection{倒转链表(Reverse Link List)}


虽然平时用不到,但是翻转链表之于程序员就像小丑的翻筋斗,是一项基础技能,面试的时候表演专用。不管会不会干活,先翻两个筋斗瞧瞧。Homebrew作者Max Howell就是不会翻转二叉树而失去了机会\footnote{\url{https://twitter.com/mxcl/status/608682016205344768}}。定义单向链表:


\begin{lstlisting}[language=Java]
/**单向链表定义**/
static class Node<T> {
    private T value;    //节点值
    private Node<T> next;   //后继节点

    public Node() {
    }
    public Node(T value, Node<T> next) {
        this.value = value;
        this.next = next;
    }
}
\end{lstlisting}

初始化单向链表:

\begin{lstlisting}[language=Java]
/**初始化链表**/
private Node initLinkedList(int num) {
    Node head = new Node(0, null);
    Node cur = head;
    for(int i=1; i<num;i++){
        cur.next = new Node(i, null);
        cur = cur.next;
    }
    return head;
}
\end{lstlisting}

翻转链表:

\begin{lstlisting}[language=Java]
/**反转链表**/
private Node reverseLinkedList(Node head) {
    if (head == null || head.next==null) {
        return head;
    }

    Node prev = null;
    Node next = null;
    while(head.next!=null){
        /**
         * 将下一个节点保存到next中
         * 左后指针要恢复指向下一个节点继续遍历
         */
        next = head.next;
        /**
         * 当前节点后跟上一次循环的倒转链表集合
         * 组成一个新的倒转集合
         * 新的倒转链表新增指针当前指向的节点
         */
        head.next = prev;
        /**
         * 保存当前节点
         * 注意prev保存时是保存的倒转之后的链表集合,不是单个节点
         * 此处的head包含指针当前的节点加上上一次循环倒转的链表节点
         */
        prev = head;    //保存当前节点
        /**
         * 链表头指针移到下一个节点
         * 进入下一个元素的循环
         */
        head = next;
    }
    head.next = prev;
    return head;
}
\end{lstlisting}

\end{document}






