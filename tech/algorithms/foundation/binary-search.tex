\documentclass[../../../interview-questions.tex]{subfiles}

\begin{document}

\subsection{二分查找算法(Binary Search)}

在计算机科学中,二分查找(英语:binary search),也称折半搜索(英语:half-interval search)、对数搜索(英语:logarithmic search),是一种在有序数组中查找某一特定元素的搜索算法。

搜索过程从数组的中间元素开始,如果中间元素正好是要查找的元素,则搜索过程结束;如果某一特定元素大于或者小于中间元素,则在数组大于或小于中间元素的那一半中查找,而且跟开始一样从中间元素开始比较。如果在某一步骤数组为空,则代表找不到。这种搜索算法每一次比较都使搜索范围缩小一半。

\begin{lstlisting}[language=Java]
public static int binary(int[] arr, int data) {
    int min = 0;
    int max = arr.length - 1;
    int mid;
    while (min <= max) {
        // 防止溢出
        mid =  min + (max - min) / 2;
        if (arr[mid] > data) {
            max = mid - 1;
        } else if (arr[mid] < data) {
            min = mid + 1;
        } else {
            return mid;
        }
    }
    return -1;
}
\end{lstlisting}

使用位运算:

\begin{lstlisting}[language=Java]
public static int binary(int[] arr, int data) {
    int min = 0;
    int max = arr.length - 1;
    int mid;
    while (min <= max) {
        // 无符号位运算符的优先级较低,先括起来
        mid =  min + ((max - min) >>> 1);
        if (arr[mid] > data) {
            max = mid - 1;
        } else if (arr[mid] < data) {
            min = mid + 1;
        } else {
            return mid;
        }
    }
    return -1;
}
\end{lstlisting}


\end{document}






