\documentclass[../../../interview-questions.tex]{subfiles}

\begin{document}

\subsection{设计模式应遵循的原则}

SOLID 原则是由 Robert C. Martin (Bob 大叔) 在 21 世纪初定义的。Bob 大叔阐述了几个并且确认了其它已经存在的原则。他说我们应该使用这些原则,让代码获得好的依赖管理。但是,SOLID 原则在最初并没有被大家熟知直到 Michael Feathers 观察到这些原则的首字母正好能拼成缩写 SOLID,这个非常具有代表性的名字。面向对象的设计模式有七大基本原则:

\begin{enumerate}
    \item {单一职责原则(Single Responsibility Principle, SRP)}一个类只负责一个功能领域中的相应职责
    \item {开闭原则(Open Closed Principle,OCP)}对扩展开放,对修改关闭
    \item {里氏代换原则(Liskov Substitution Principle,LSP)}所有引用基类的地方必须能透明地使用其子类的对象
    \item {接口隔离原则(Interface Segregation Principle,ISP)}类之间的依赖关系应该建立在最小的接口上
    \item {依赖倒转原则(Dependency Inversion Principle,DIP)}依赖于抽象,不能依赖于具体实现
    \item {合成/聚合复用原则(Composite/Aggregate Reuse Principle,CARP)}尽量使用合成/聚合,而不是通过继承达到复用的目的
    \item {最少知识原则(Least Knowledge Principle,LKP)或者迪米特法则(Law of  Demeter,LOD)}一个软件实体应当尽可能少的与其他实体发生相互作用
\end{enumerate}

https://cloud.tencent.com/developer/article/1650116

\end{document}